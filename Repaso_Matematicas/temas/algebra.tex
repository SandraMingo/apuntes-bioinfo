\section{Álgebra lineal}
\subsection{Vectores}
Un vector es una cantidad que tiene tanto magnitud como dirección. En el contexto de álgebra lineal, un vector es una lista ordenada de números.
Un vector se puede representar como:
$$\vec{v} = \begin{pmatrix}
v_1 \\ v_2 \\ \ldots \\ v_n
\end{pmatrix}$$

\paragraph{Suma de vectores} Dos vectores del mismo tamaño se pueden sumar componente a componente.
$$\vec{u} + \vec{v} = \begin{pmatrix}
u_1 + v_1 \\ u_2 + v_2 \\ \ldots \\ u_n + v_n
\end{pmatrix}$$

\paragraph{Multiplicación por un escalar} Un vector se puede multiplicar por un número (escalar) multiplicando cada componente por ese número.
$$c\vec{v} = \begin{pmatrix}
cv_1 \\ cv_2 \\ \ldots \\ cv_n
\end{pmatrix}$$

\paragraph{Producto punto (o producto escalar)} Dados dos vectores $\vec{u}$ y $\vec{v}$, su producto punto es:
$$\vec{u} \cdot \vec{v} = u_1v_1 + u_2v_2 + \ldots + u_nv_n$$
El producto punto es útil para calcular ángulos entre vectores y para proyectar un vector sobre otro.

\subsection{Matrices}
Una matriz es una disposición rectangular de números en filas y columnas. Las matrices son fundamentales en álgebra lineal y se utilizan para representar transformaciones lineales, sistemas de ecuaciones lineales, y más.
Una matriz $\vec{A}$ de tamaño $m \times n$ se representa como:
$$\vec{A} = \begin{pmatrix}
a_{11} & a_{12} & \ldots & a_{1n} \\
a_{21} & a_{22} & \ldots & a_{2n} \\
\ldots & \ldots & \ldots & \ldots \\
a_{m1} & a_{m2} & \ldots & a_{mn} 
\end{pmatrix}$$
donde $a_{ij}$ es el elemento en la fila $i$ y la columna $j$.

\paragraph{Suma de matrices} Dos matrices del mismo tamaño se pueden sumar elemento a elemento.
$$\vec{A} + \vec{B} = \begin{pmatrix}
a_{11} + b_{11} & a_{12} + b_{12} & \ldots & a_{1n} + b_{1n} \\
a_{21} + b_{21} & a_{22} + b_{22} & \ldots & a_{2n} + b_{2n} \\
\ldots & \ldots & \ldots & \ldots \\
a_{m1} + b_{m1} & a_{m2} + b_{m2} & \ldots & a_{mn} + b_{mn}
\end{pmatrix}$$

\paragraph{Multiplicación por un escalar} Una matriz se puede multiplicar por un escalar multiplicando cada elemento de la matriz por ese escalar.
$$c\vec{A} = \begin{pmatrix}
ca_{11} & ca_{12} & \ldots & ca_{1n} \\
ca_{21} & ca_{22} & \ldots & ca_{2n} \\
\ldots & \ldots & \ldots & \ldots \\
ca_{m1} & ca_{m2} & \ldots & ca_{mn} 
\end{pmatrix}$$

\paragraph{Multiplicación de matrices} Dadas dos matrices $\vec{A}$ de tamaño $m \times n$ y $\vec{B}$ de tamaño $n \times p$, su producto $\vec{C} = \vec{AB}$ es una matriz de tamaño $m \times p$ donde cada elemento $c_{ij}$ se calcula como:
$$c_{ij} = \sum^n_{k=1} a_{ik}b_{kj}$$
La multiplicación de matrices no es conmutativa, es decir, $\vec{AB} \neq \vec{BA}$ en general.

\paragraph{Transpuesta de una matriz}
La transpuesta de una matriz $\vec{A}$, denotada como $\vec{A}^T$, se obtiene intercambiando filas por columnas.
$$\vec{A} = \begin{pmatrix}
a_{11} & a_{12} & \ldots & a_{1n} \\
a_{21} & a_{22} & \ldots & a_{2n} \\
\ldots & \ldots & \ldots & \ldots \\
a_{m1} & a_{m2} & \ldots & a_{mn} 
\end{pmatrix} \rightarrow \vec{A}^T = \begin{pmatrix}
a_{11} & a_{21} & \ldots & a_{m1} \\
a_{12} & a_{22} & \ldots & a_{m2} \\
\ldots & \ldots & \ldots & \ldots \\
a_{1n} & a_{2n} & \ldots & a_{mn} 
\end{pmatrix}$$

\paragraph{Determinante de una matriz}
El determinante es un valor escalar que se puede calcular a partir de los elementos de una matriz cuadrada. Es útil para determinar si una matriz tiene inversa y en la resolución de sistemas de ecuaciones lineales.

Para una matriz $2 \times 2$:
\begin{align}
\det \vec{A}
=
\begin{vmatrix}
a & b \\ 
c & d
\end{vmatrix}
=ad - bc\notag
\end{align}
Para matrices más grandes, el cálculo del determinante es más complejo y se puede hacer mediante expansión por cofactores.

\paragraph{Inversa de una matriz}
La inversa de una matriz cuadrada $\vec{A}$, denotada como $\vec{A}^{-1}$, es una matriz tal que:
$$\vec{AA}^{-1} = \vec{A}^{-1} \vec{A} = \vec{I}$$
donde $\vec{I}$ es la matriz de identidad. No todas las matrices tienen inversa; una matriz tiene inversa si y solo si su determinante es distinto de cero.

\paragraph{Norma de una matriz}
Las normas de una matriz son herramientas matemáticas que nos permiten cuantificar el "tamaño" o la "magnitud" de una matriz. En el contexto del aprendizaje automático (Machine Learning, ML), las normas de matrices son útiles para regularizar modelos, medir errores, y entender la estabilidad de los algoritmos.

Una norma de matriz es una función que asigna un número real no negativo a una matriz, cumpliendo ciertas propiedades:
\begin{enumerate}
\item \textbf{No negatividad:} $||\vec{A}|| \geq 0$ y $||\vec{A}|| = 0$ si y solo si $\vec{A} = 0$
\item \textbf{Homogeneidad:} $||c\vec{A}|| = |c| \cdot ||\vec{A}||$ para cualquier escalar $c$
\item \textbf{Desigualdad triangular:} $||\vec{A} + \vec{B}|| \leq ||\vec{A}|| + ||\vec{B}||$
\item \textbf{Submultiplicatividad} (para normas de matrices): $||\vec{AB}|| \leq ||\vec{A}|| \cdot ||\vec{B}||$
\end{enumerate}

En ML, las normas más utilizadas son las \textbf{normas vectoriales} (aplicadas a matrices) y las \textbf{normas matriciales específicas}.
\begin{itemize}
\item \textbf{Norma $L_1$ (Norma de Manhattan)} En ML se utiliza en la regularización $L_1$ (Lasso) para obtener modelos dispersos (sparse model).

Para un vector $\vec{v} = (v_1, v_2, \ldots, v_n)$, la norma $L_1$ se define como:
$$||\vec{v}||_1 = \sum^n_{i=1} |v_i|$$

Para una matriz $\vec{A}$, la norma $L_1$ es la máxima suma absoluta de las columnas:
$$||\vec{A}||_1 = max_{1 \leq j \leq n} \sum^m_{i=1} |a_{ij}|$$

\end{itemize}