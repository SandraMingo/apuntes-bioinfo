\documentclass{config/apuntes}

\title{Caracterización de Redes y Topologías Biológicas}
\author{Sandra Mingo Ramírez}
\date{2024/25}
\acronym{CRRTT}

\usepackage[all]{nowidow}
\usepackage{listing}
\usepackage{color}
\usepackage{tabularx}
\usepackage{multirow}
\usepackage[normalem]{ulem}
\useunder{\uline}{\ul}{}\usepackage{makecell}
\usepackage{amsmath}
\usepackage{array}


\definecolor{dkgreen}{rgb}{0,0.6,0}
\definecolor{gray}{rgb}{0.5,0.5,0.5}
\definecolor{mauve}{rgb}{0.58,0,0.82}

\lstset{
  language=Python,
  frame=tb,
  aboveskip=3mm,
  belowskip=3mm,
  showstringspaces=false,
  columns=flexible,
  basicstyle={\small\ttfamily},
  numbers=none,
  numberstyle=\tiny\color{gray},
  keywordstyle=\color{blue},
  commentstyle=\color{dkgreen},
  stringstyle=\color{mauve},
  breaklines=true,
  breakatwhitespace=true,
  tabsize=3
}

\begin{document}

\begin{abstract}
En esta asignatura se estudian los principales tipos de conectividad que se pueden dar en una red biológica. Se describen además cuales puede ser la mejor estrategia de conexión entre los elementos de una red sujetos a una determinada dinámica. También se proporcionan métodos para calcular los principales parámetros topológicos y de rendimiento de una red dada. Además se estudian redes resistentes a una determinada estrategia de ataque o frente a errores en la red.
\end{abstract}

\pagestyle{plain}

\maketitle

\tableofcontents

%40% presentación de proyecto
%60% examen tipo test que resta 33% (nos va a decir lo que entra en el examen)
%Índice de clusterización, camino característico (si no entra uno, entra el otro, pero esos dos seguro)

%18/03 - Carlos Aguirre B322
\chapter{Introducción y descripción de algunas redes reales}
Aunque lo vayamos a utilizar como sinónimos, un grafo y una red no es lo mismo; el grafo es la representación matemática de la red. En una red aleatoria, no hay que medir nada; si una red biológica sale aleatoria, se ha medido mal. Las redes biológicas son todas de mundo pequeño. Además, casi todas son libres de escala. 

\section{Qué es una red}
Una red es un conjunto de elementos (personas, ciudades, proteínas, especies animales, productos químicos, etc) de las cuales algunas están conectadas con otras y otras no. Se puede representar en bolas que se unen con líneas con otras líneas. Las bolitas se denominan como nodos. 

Las redes se estudian con NetworkX y Cytoscape.

\section{Algunos ejemplos de redes y algunas de sus propiedades}
\subsection{World Wide Web}
La World Wide Web es la mayor red para la cual existe información topológica. Los nodos de la red son los documentos, y las ramas son los enlaces (hyperlinks) entre documentos. El tamaño actual de esta red es de más de 1000 millones de nodos. Esta red es dirigida: la página A apunta a la página B, pero sin que la página B apunte a la página A. 
El grupo CAIDA se dedica a analizar la red. Esta red es enorme, pudiendo dibujar solo a un nivel muy alto.

La distribución del grado de las páginas web tiene una distribución libre de escala tanto en los enlaces de salida como en los enlaces de entrada. Esto es una distribución de probabilidad. Por ejemplo, en el queso de Gruyere, los agujeros son de distinto tamaño, los cuales tienen una distribución de tamaño. Se llama libre de escala porque se pueden encontrar diez veces más los agujeros de un tamaño mayor y diez veces menos los agujeros de tamaño pequeño. Así, no hay una escala fija de la distribución (no se puede representar con ninguna escala, ni logarítmica ni nada). Esto con el queso manchego no pasa. Si en la WWW vemos cuántas páginas web tienen 100 enlaces de salida, 10, 1000, etc, y se dibuja en escala logarítmica logarítmica, sale una recta. Esto pasa también con los enlaces de entrada. 

La distancia entre dos páginas de la WWW es pequeña (entre 11 y 16). Los nodos de la WWW están muy clusterizados.

La cola de la derecha parece que rompe la recta. Las redes libres de escala se pueden producir por muchas razones, pero al utilizar un proceso evolutivo en el que cada tiempo se generan nuevos nodos y tienen mayor preferencia para conectarse a otros nodos, e incluso pueden desaparecer algunos nodos antiguos. Esto produce las colas residuales.

\begin{figure}[h]
\centering
\includegraphics[width = 0.6\textwidth]{figs/www.png}
\end{figure}

Las redes libres de escala son muy resistentes a ataques aleatorios (fallos en la red) en cuanto a la conectividad, por lo que hay una razón evolutiva por la que las redes biológicas son libres de escala. La red regulatoria de P53 está muy estudiada y caracterizada. Uno de los elementos más importantes es MDM5.

\subsection{Internet}
Internet es la red de enlaces físicos entre ordenadores u otros servicios de comunicación. La topología de internet se suele estudiar a dos niveles: Enrutadores y Sistemas autónomos. Los enrutadores son las máquinas que mandan los "paquetes" a otros enrutadores. Hay algoritmos de enrutación que deciden hacia dónde enviar las cosas. Los sistemas autónomos son conjuntos de máquinas que organizan y gestionan otras máquinas.

Para ambos tipos de red (enrutadores y sistemas autónomos) el grado de cada nodo seguía una distribución libre de escala. De nuevo la red está altamente clusterizada (coeficiente de clustering entre 0,18 y 0,3) y los caminos entre nodos son cortos (aproximadamente 9). 

El \textbf{índice de clusterización} es una medida de la probabilidad de que los dos vecinos de un nodo sean vecinos entre sí, favoreciendo la creación de triángulos. Es decir, en redes sociales, que mis amigos también sean amigos entre sí. En biología, si dos proteínas son expresadas por una tercera proteína, las dos mantienen una relación entre sí (aunque puede no pasar). Los vecinos de un mismo nodo tienen una probabilidad alta de ser vecinos entre sí. En una red aleatoria, los vecinos de un nodo dependen de la probabilidad de rama de que esos nodos también sean vecinos entre sí (como cualquier otro).

La métrica de caminos cortos o largos se hace en comparación con el grafo aleatorio con el mismo número de nodos y ramas. En biología, los caminos también suelen ser cortos, y si son largos se puede deber a una enfermedad o patología.

\subsection{Red de actores}
Los nodos son actores, y dos de ellos están conectados si han participado juntos en alguna película. Actualmente, la red consta de unos 450.000 actores.
La distancia media entre actores es 3,65. La red está altamente clusterizada (100 veces más que un grafo aleatorio). La distribución de grados sigue una ley de potencias (libre de escala).

\subsection{Red de colaboración científica}
Los nodos están constituidos por científicos. Dos nodos están conectados si alguna vez publicaron un trabajo en común. La red de nuevo presenta una distribución libre de escala, caminos cortos entre los nodos y una alta clusterización. 

El \textbf{centro de la red} es el nodo que está a una menor distancia promedio del resto de nodos de la red. Este centro lo tiene un científico húngaro llamado Paul Erdös que trabajaba en teoría de grafos. 

Para una red de citaciones científicas, los nodos de la red son artículos científicos. Las ramas son citaciones entre artículos. Se tiene una base de datos de unos 750.000 artículos. Tanto los grados de entrada como los de salida siguen una distribución libre de escala.

\subsection{Red de contactos sexuales}
Los nodos y las ramas tienen una definición obvia. Tiene interés por la difusión de enfermedades (especialmente aquellas de transmisión sexual como el SIDA). Presenta una distribución libre de escala. Se sospecha que los datos de esta red no son totalmente fiables (es defectuosa al tener muchos datos falsos). Entre un 10-15\% es falsa.

Se define como k-core un grafo no dirigido creado a partir de un grafo más grande en el que se crean jerarquías o grupos en el que los nodos están separados por k vértices.

\subsection{Red de llamadas telefónicas}
Los nodos son números de teléfono. Las ramas son llamadas de larga distancia entre nodos. De nuevo la red presenta una distribución libre de escala.

\subsection{Redes lingüísticas}
Los nodos son palabras. Dos nodos están conectados si están juntas en alguna frase y hay solamente una palabra entre ambas. Un estudio realizado en inglés sobre 440.902 palabras presentó una distancia media de 2,62 y un índice de clusterización de 0,43.

Otra red linguística considera de nuevo los nodos como palabras. Dos nodos están conectados si se considera que ambas palabras son sinónimas (de acuerdo con el Merrian Webster Dictionary). El camino medio es de 4,7, el índice de clusterización es de 0,7 y los nodos presentan una distribución libre de escala.

En la red semántica, cada nodo es un objeto o un concepto. Dos nodos se relacionan entre sí, si existe una relación de la forma "es un" o "tiene un" entre ambos nodos. Se ha estudiado poco, pero parece presentar un camino medio corto, alta clusterización y una distribución de nodos libre de escala.

\subsection{Redes eléctricas}
La red eléctrica del Oeste de los Estados Unidos está compuesta por nodos (generadores, transformadores y subestaciones) y ramas (cables físicos entre nodos). La red tiene 4.941 nodos y un grado medio por nodo de 2,41. Esta red se aparta del patrón habitual teniendo una estructura muy jerárquita y en forma de estrella. Esto hace que sea muy frágil y condicionada a cuestiones económicas y políticas. Ocurre de forma similar con las redes de internet. No se utiliza el camino más rápido o corto, si no el camino más barato (como a la hora de buscar vuelos).

\section{Algunos ejemplos de redes biológicas y algunas de sus propiedades}
\subsection{Redes de ecología}
En las redes alimentarias, los nodos de la red son especies, y las ramas relaciones predador-presa entre especies. Las distancias son cortas entre los elementos de la red. En general, son redes con pocos nodos. 

Al ser redes pequeñas es difícil dibujar la distribución del grado de los nodos. Parecen presentar una distribución libre de escala, con un exponente inusualmente pequeño. Esta red es dirigida (aunque pueda haber dobles ramas). 

\subsection{Redes celulares}
Se presentan al estudiar el metabolismo de organismos. Los nodos son sustratos químicos (ATP, ADP, etc), y las ramas presentan reacciones químicas entre los sustratos. Esta red va de arriba a abajo, empezando con unos productos de entrada de la célula y terminando con productos de salida que la célula no puede descomponer más. 

\subsection{Redes neuronales}
Cada nodo es una neurona (biológica o artificial), y las ramas son conexiones sinápticas entre neuronas. La primera red estudiada de este tipo es la del gusano \textit{Caenorhabditis elegans}, del cual se tiene el mapa neuronal completo.

Las redes neuronales artificiales están ahora en auge para las inteligencias artificiales al utilizarse para el aprendizaje profundo.

\subsection{Redes de interacción de proteínas}
Cada nodo es una proteína. Las ramas representan relaciones de expresión entre las proteínas. Una de las redes más importantes es la red p53 de control de crecimiento del cáncer. Un paper muy bueno es \href{https://www.nature.com/articles/35042675}{Surfing the p53 network (DOI 10.1038/35042675)}.

Esta es la red en la que más se trabaja en biología. Se buscan los efectos entre los nodos (aumenta la expresión, inhibe), los componentes clave, los parámetros, etc. 

\subsection{Redes genéticas}
Cada nodo es expresión genética (nucleótidos). Las ramas conectan los nucleótidos que presentan un alto índice de similitud entre ambas. Una vez representada la red, se buscan familias o grupos de genes similares. Hay que diferenciar identidad con similitud (sobre todo con desajuste de fase). Se utiliza programación dinámica para calcular la mayor longitud de subsecuencia idéntica, como por ejemplo con el algoritmo Soldier's Walk. 

Si clusterizamos y obtenemos 2 cluster, cada cluster indica un gen con errores, o dos individuos distintos. Luego hay que interpretar por qué hay ese número de cluster. Normalmente hay muchos clusters que se quieren clasificar, y en cada cluster suele aparecer el mismo gen que se ha mutado. 

Las máquinas de microarrays ahora dan un conjunto de nucleótidos muy grandes, pero antes se obtenían fragmentos que había que unir. Para ello, se debían utilizar algoritmos sobre grafos para calcular cadenas largas a partir de las cadenas cortas, pero ahora ya no se usa por las mejoras tecnológicas. 

%20/03 - Carlos Aguirre
\chapter{Teoría de grafos y métricas}
\section{Introducción a la teoría de grafos}
La teoría de grafos ha sido utilizada recientemente para:
\begin{itemize}
\item Clasificación automática de secuencias de proteínas.
\item Detección de jerarquías de proteínas.
\item Análisis de redes genéticas.
\item Reconstrucción de redes genéticas grandes obtenidas mediante modificación de genes.
\end{itemize}

Un grafo G es un par de conjuntos (V,E) donde $V = \{v_1, v_2, \ldots v_n\}$ es el conjunto de vértices o nodos y $E = \{(v_i, v_j), (v_{i'}, v_{j'}), \ldots \}$ es un conjunto de pares no ordenados de elementos de V y se denomina conjunto de ramas del grafo. 
\marginpar[\footnotesize Pregunta de test: define orden y tamaño, dado un grafo dar el orden y tamaño, etc. ]  \
El número de nodos se denomina \textbf{orden} del grafo, y el número de ramas es el \textbf{tamaño} del grafo. 

\begin{figure}[h]
\centering
\includegraphics[width = 0.6\textwidth]{figs/grafo.png}
\caption{Ejemplo de grafo de orden 8 y tamaño 11.}
\end{figure}

Para una red de proteínas, cada proteína sería un nodo del grafo, y una rama indicaría interacción entre ambas proteínas. 

Una disposición (layout) es una posible colocación de los nodos y las ramas en un espacio 2D o 3D. Un mismo gráfo puede tener múltiples colocaciones. Ejemplo, consideremos el grafo G=(V,E).

\begin{figure}[h]
\centering
\includegraphics[width = 0.6\textwidth]{figs/layout.png}
\end{figure}

Existen programas de ordenador que nos permiten obtener colocaciones predefinidas (Gephy, Pajek). Cuando no se especifica ninguna colocación, se entiende que los nodos se sitúan aleatoriamente sobre el plano o espacio. Algunos de los tipos más habituales de colocaciones son:
\begin{itemize}
\item Colocaciones regulares
\item Basadas en la física (atracción-repulsión)
\item Basadas en propiedades topológicas (jerarquías, número de vecinos, etc)
\end{itemize}

Un hipergrafo H es un también par de conjuntos (V,E) donde $V=\{v_1, v_2, \ldots v_n\}$ es el conjunto de vértices o nodos y $E=\{(v_{i1}, v_{i2}, \ldots),(v_{i'1},v_{i'2}, \ldots), \ldots\}$ es una familia de
subconjuntos no ordenados de elementos de V. E se denomina conjunto de hiperramas o hiperaristas del hipergrafo. El número de hiperramas $|E|$ se denomina cardinalidad del hipergrafo. El valor $|E|*|V|$ se denomina tamaño o volumen del grafo.
\marginpar[\footnotesize Pregunta examen ]  \
Si tenemos un grafo de n nodos, ¿cuántas parejas podemos tener como máximo? $(n \cdot n-1)/2$ Por tanto, en un grafo con n nodos, ¿cuántas ramas puede tener? Igual, $(n \cdot n-1)/2$

\begin{figure}[h]
\centering
\includegraphics[width = 0.6\textwidth]{figs/hipergrafo.png}
\caption{Ejemplo de hipergrafo de cardinalidad 4 y tamaño 32.}
\end{figure}

Un hipergrafo H se dice que es \textbf{propio} si no es vacío ($V\neq \varnothing$) y no contiene ninguna arista vacía. Un hipergrafo H se dice que tiene \textbf{dominio completo} si todos los nodos están en al menos una arista, en caso contrario se dice que tiene \textbf{dominio parcial}. Si en un hipergrafo todas las hiperramas tienen el mismo número de nodos, entonces se denomina \textbf{hipergrafo k-uniforme}. 

\textit{Ejercicio: Indicar si el hipergrafo del ejemplo anterior es propio, tiene dominio completo y si es k uniforme}. Es propio (el conjunto de vértices tiene 8 elementos y todas las ramas e tienen vértices dentro), es de dominio parcial (v5 no está en ninguna rama) y no es k-uniforme (e1 tiene 3 elementos, e2 tiene 2, e3 tiene 1 y e4 tiene 4).

\section{Bucles y ramas paralelas}
Un bucle es una rama que empieza y termina en el mismo nodo $(v_i, v_i)$. Cuando dos ramas conectan el mismo par de vértices se denominan paralelas. Un grafo con bucles se denomina pseudografo. Un grafo con ramas paralelas pero sin bucles se denomina multigrafos. Un grafo sin bucles ni ramas paralelas se denomina grafo simple.

\begin{figure}[h]
\centering
\includegraphics[width = 0.6\textwidth]{figs/bucle-rama-paralela.png}
\end{figure}

%25/03 - Carlos Aguirre
\section{Grafos dirigidos y ponderados}
Se puede considerar que los enlaces entre nodos son dirigidos $(v_i, v_j) = (v_j, v_i)$. Los grafos dirigidos se denominan también \textbf{digrafos}.

\begin{figure}[h]
\centering
\includegraphics[width = 0.6\textwidth]{figs/grafo-dirigido.png}
\end{figure}

En los grafos ponderados, a cada rama del grafo se le puede asociar un número. El número asociado a cada rama puede indicar entre otras cosas una distancia, una capacidad, un valor temporal, etc.

\begin{figure}[h]
\centering
\includegraphics[width = 0.6\textwidth]{figs/grafo-ponderado.png}
\end{figure}

\newpage

Los grafos dirigidos y ponderados poseen ramas dirigidas a las que se asocia un número.

\begin{figure}[h]
\centering
\includegraphics[width = 0.6\textwidth]{figs/grafo-dirigido-ponderado.png}
\end{figure}

\section{Grado de un nodo}
Dos nodos de un grafo son \textbf{vecinos o adyacentes} si existe una rama que los conecta. El \textbf{grado} de un nodo es el número vecinos que tiene dicho nodo. En los grafos dirigidos se calcula el \textbf{grado de entrada} y el \textbf{grado de salida}. En los grafos ponderados, el grado se puede promediar por el número asociado a las ramas. Un grafo se dice que es \textbf{regular} si todos los nodos tienen el mismo grado.

\begin{figure}[h]
\centering
\includegraphics[width = 0.6\textwidth]{figs/grafo-grados.png}
\end{figure}

\section{Subgrafos}
Un grafo G’=(V’,E’) es un subgrafo de un grafo G=(V,E) si V’ es un subconjunto de V y E’ es un subconjunto de E. En otras palabras, un subgrafo es un trozo de un grafo más grande. 

\begin{figure}[h]
\centering
\includegraphics[width = 0.6\textwidth]{figs/subgrafo.png}
\end{figure}

Un subgrafo G’=(V’,E’) de un grafo G=(V,E) se dice que es \textbf{abarcador} si V=V’, es decir, si están todos los nodos, pero faltan algunas ramas.

Un grafo es un subgrafo de sí mismo. Además, un grafo vacío es un subgrafo de cualquier grafo.

\section{Paseos, caminos, circuitos y ciclos}
Un \textbf{paseo} de un nodo u a un nodo v es una secuencia de vértices $\{v_0, v_1, \ldots, v_k\}$ con $v_1 = u v_k = v$ y $(v_{i-1}, v_i)$ rama del grafo. El número de ramas del paseo es su \textbf{longitud}. Un paseo en el cual no se repiten ramas se denomina \textbf{rastro}. Un paseo en el cual todos los vértices $\{v_0, v_1, \ldots, v_k\}$ son distintos se denomina \textbf{camino}. 
Un camino siempre debe ser un rastro y un paseo. Si algo no es rastro, no puede ser camino, y si no es paseo, no puede ser ni rastro ni camino. Cada uno es cada vez más restrictivo.

Entre dos nodos, puede haber varios caminos posibles. Un \textbf{camino mínimo} entre dos nodos es aquel de menor longitud de entre todos los posibles caminos entre ambos nodos. La \textbf{distancia} entre dos nodos del grafo se define como la longitud de cualquier camino mínimo que los una.

\begin{figure}[h]
\centering
\includegraphics[width = 0.5\textwidth]{figs/paseo-rastro.png}
\end{figure}

\begin{figure}[h]
\centering
\includegraphics[width = 0.5\textwidth]{figs/camino-minimo.png}
\end{figure}

Un \textbf{paseo cerrado} es un paseo $\{v_0, v_1, \ldots, v_k\}$ tal que $v_0 = v_k$. Un paseo cerrado en el que no se repiten ramas es un \textbf{circuito}. Un \textbf{ciclo} es un circuito en el que no se repiten vértices. Los ciclos son importantes, porque las redes biológicas tienen ciclos (que suelen ser largos), pero en las redes aleatorias no aparecen ciclos, o éstos son muy pequeños.

\begin{figure}[h]
\centering
\includegraphics[width = 0.5\textwidth]{figs/ciclo.png}
\end{figure}

El nodo con menor distancia entre los demás es muy importante, denominándose como \textbf{centro del grafo}.

Para un grafo con excesivos nodos, los caminos mínimos y las distancias se calculan con un algoritmo. Si el grafo es no ponderado, se utiliza el algoritmo búsqueda en anchura, mientras que si es ponderado, utiliza Dijkstra.

\section{Medidas de centralidad, betweeness y closeness}
\marginpar[\footnotesize Pregunta examen: Betweeness/Closeness/Farness se define como... ]  \
Dado un nodo $v_i$ se define su \textbf{betweeness} $C_B (v_i)$ como la fracción de caminos mínimos que hay entre el resto de nodos del grafo y que pasan por el nodo $v_i$. Es decir, se hacen parejas de todos los nodos del grafo excluyendo el nodo de interés, y se calculan los caminos mínimos. Algunos pasarán por el nodo de interés, que son los que nos quedamos. Con eso se evalúa el cociente (los que pasan por ese nodo entre todos), que será el betweeness (un valor entre 0 y 1).
La centralidad de un nodo es muy costosa de calcular, usualmente se emplean algoritmos aproximados. 

Dado un nodo $v_i$ se define su \textbf{lejanía o farness} $C_F (v_i)$ como la suma de las distancias de $v_i$ al resto de nodos del grafo. 

Dado un nodo $v_i$ se define su \textbf{cercanía o closeness} $C_C (v_i)$ como la inversa de su lejanía $C_C (v_i) = 1/C_F (v_i)$.

\begin{figure}[h]
\centering
\includegraphics[width = 0.5\textwidth]{figs/medidas-centralidad.png}
\caption{Respuesta al ejercicio: Cogiendo v4, la lejanía será 2+1+2+1+1+2+2 = 11, y la cercanía 1/11.}
\end{figure}

La cercanía y lejanía tiene un problema: su valor numérico depende del orden del grafo. Por tanto, sirve para comparar dentro del mismo grafo, pero no entre grafos. 
\marginpar[\footnotesize Pregunta examen: Calcular camino característico ] \
Para eso, habría que normalizar dividiendo por el número total de nodos. A esto se le conoce como \textbf{camino característico}.

\section{Conexidad}
Un grafo es \textbf{conexo} si para cada par de nodos del grafo existe al menos un camino que los une. En otras palabras, que no esté separado en distintos trozos. 

\begin{figure}[h]
\centering
\includegraphics[width = 0.6\textwidth]{figs/conexidad.png}
\end{figure}

Hay un algoritmo muy rápido y eficiente que calcula si un grafo es conexo o no. 

Una \textbf{componente conexa} de un grafo es cada uno de los subgrafos
maximales conexos. Esto quiere decir que el subgrafo no puede ser más grande, que no se le puede añadir más nodos.

\begin{figure}[h]
\centering
\includegraphics[width = 0.5\textwidth]{figs/componentes-conexas.png}
\end{figure}

Un \textbf{punto de articulación} es un nodo que desconecta un grafo conexo. Un \textbf{corte} es un conjunto de ramas que desconecta un grafo conexo. Si un corte esta compuesto por una única rama, se denomina \textbf{puente}. Un \textbf{corte mínimo} de un grafo es el mínimo número de ramas que al ser eliminadas desconectan el grafo.

El algoritmo CLICK (CLuster Identification via Connectivity Kernels) calcula una aproximación al corte mínimo. Esto lo hacían cogiendo los dos nodos más lejanos. Los puentes suelen ser muy malos para la conexidad de los grafos. 

\begin{figure}[h]
\centering
\includegraphics[width = 0.5\textwidth]{figs/conexidad2.png}
\end{figure}

La máxima distancia entre cualquier par de nodos se denomina como diámetro.

El corte mínimo entre dos nodos es siempre mayor que el corte mínimo de todo el grafo.  

\section{Bosques y árboles}
Un grafo sin ciclos (acíclico) se denomina bosque. Un árbol es un grafo acíclico conexo. Cada componente conexa de un bosque es un árbol.

\begin{figure}[h]
\centering
\includegraphics[width = 0.6\textwidth]{figs/arbol-bosque.png}
\end{figure}

Un subgrafo abarcador acíclico de un grafo G se denomina un \textbf{bosque abarcador}. Un subgrafo abarcador conexo acíclico de un grafo G se denomina un \textbf{árbol abarcador}.

\begin{figure}[h]
\centering
\includegraphics[width = 0.6\textwidth]{figs/arbol-abarcador.png}
\end{figure}



%06/05 - Carlos Aguirre
\chapter{Grafos aleatorios, grafos regulares y redes de mundo pequeño}
\section{Grafos aleatorios}
\subsection{Modelo de Erdös y Rényi}
En las redes reales muy raramente suele aparecer una topología aletoria. Sin embargo los grafos aleatorios han sido muy estudiados por las siguientes razones. Si encontramos una propiedad que ocurre con probabilidad 1 para ciertos parámetros de la red, podemos saber si nuestra red tiene esa propiedad solo con mirar sus parámetros. Para predecir el comportamiento de ciertas propiedades en función de parámetros de la red. Para comprobar si nuestra red tiene sesgos estructurales.

El modelo de Erdös y Rényi  se estudió en los años 50-60. Cada rama del grafo existe con una probabilidad p, que suele seguir una distribución uniforme.

Otra forma de definir los grafos aleatorios es seleccionar parejas de nodos aleatoriamente. Se seleccionan exactamente $pN(N-1)/2$ parejas. Ambos tipos de grafos aleatorios son equivalentes.

\subsection{Propiedades}
El estudio de grafos aleatorios se centra sobre todo en averiguar para que probabilidad p aparece cierta propiedad Q:
\begin{itemize}
\item Cuando el grafo es conexo.
\item Cuando la distancia media es menor que cierto número.
\item Cuando el índice de clusterización es mayor que cierto número.
\end{itemize}

Erdös y Rényi descubrieron que las propiedades Q aparecían de forma repentina según crecía p. Para muchas propiedades Q se verifica que existe una \textbf{probabilidad crítica} $p_c(N)$ tal que:
\begin{itemize}
\item Con probabilidad 0 el grafo no tiene Q si $p(N) < p_c(N)$
\item Con probabilidad 1 el grafo tiene Q si $p(N) > p_c(N)$
\end{itemize}

\subsection{Subgrafos}
La primera propiedad que estudiaron fue la aparición de subgrafos. Por ejemplo, a que probabilidad crítica p casi todo grafo G contiene un árbol de orden 3. 
La probabilidad crítica $p_c(N)$ de encontrar algunos subgrafos es:
\begin{figure}
\centering
\includegraphics[width = 0.7\textwidth]{figs/prob-critica.png}
\end{figure}

Ejercicio: Calcular la probabilidad crítica p para que una red aleatoria de 1000 nodos contenga tanto un ciclo de orden 5 como un cliqué de orden 5 (cuidado, tiene truco).
\marginpar[\footnotesize Pregunta examen antigua] \
El ciclo de orden 5 aparece con una Z de -1
Un cliqué de 5 aparece con una Z de -0.5. 
Para que haya un cliqué de orden 5 es obligatorio que haya un ciclo de orden 5. Por tanto, teniendo N = 1000 nodos, solo habría que calcular la probabilidad de un cliqué: $1000^{-1/2} = 0.031$.

\subsection{Clusters}
Un subgrafo aislado y conexo es un
cluster. Erdös y Rényi demostraron que la estructura de clusters de un grafo
cambia abruptamente cuando el grado medio (número de ramas dividido entre el número de nodos) se acerca a 1.
$$\langle k \rangle = \frac{\frac{N \cdot (N-1)}{2} \cdot p}{N} = \frac{p \cdot (N-1)}{2} \xrightarrow{N \to \infty} p\frac{N}{2}$$

Si el grado medio <k> está entre 0 y 1, casi todos los clusters son árboles (en su mayor parte) o clusters que contienen un solo ciclo. El número de clusters es de orden N-n (número de nodos menos número de ramas), y el cluster mayor es un árbol de tamaño proporcional a N. 

Si el grado medio <k> es mayor que 1, la estructura anterior cambia completamente. Aparece un cluster gigante con [1-f(<k>)]N nodos donde f es una función que decae exponencialmente de 1 a 0 cuando x va a infinito. Los demás clusters pertenecen a árboles con Nf(<k>) nodos.

\subsection{Distribución de grado}
El grado de cada nodo sigue una distribución binomial. La distribución del grado de los nodos sigue una distribución de Poisson. 

\subsection{Conexidad y diámetro}
El diámetro de un grafo es la máxima distancia entre cualquier par de nodos. Si p no es demasiado pequeño los grafos aleatorios tienden a tener poco diámetro. Casi todos los grafos aleatorios tienen el mismo diámetro (más o menos) para la mayor parte de los valores de p.

En general se tiene:
\begin{itemize}
\item Si $\langle k \rangle < 1$: el grafo tiene árboles aislados.
\item Si $\langle k \rangle > 1$: aparece el cluster gigante y el diámetro del grafo es el del cluster gigante. Para $\langle k \rangle > 3.5$, el diámetro es proporcional a $\ln(N)/\ln(<k>)$.
\item Si $\langle k \rangle \geq ln(N)$ : el grafo es conexo y su diámetro es próximo a $\ln(N)/\ln(\langle k \rangle)$
\end{itemize}

El camino característico se comporta de forma similar al diámetro. En particular:
 $$\ell_{rand} \sim \frac{\ln(N)}{\ln(\langle k \rangle)}$$
 
\subsection{Índice de clusterización}
En un grafo aleatorio la probabilidad de que dos vecinos de un nodo dado están conectados es igual a la que dos nodos elegidos al azar estén conectados.
\marginpar[\footnotesize Pregunta típica de examen] \
El índice de clusterización de un grafo alteatorio es p, la probabilidad de existencia de cada rama.

\section{Grafos regulares}
Son los mejor conocidos de forma analítica debido a sus simetrías. Existen expresiones cerradas para todas las métricas. Son los más usados en modelos de redes neuronales artificiales. También se usan como sustrato inicial para la generación de otros tipos de redes.

Un caso particular de grafos regulares son las mallas (grids) donde cada nodo se conecta a sus $2^d \cdot k$ vecinos más próximos, donde d es la dimensión del grid.

En las mallas monodimensionales, el índice de clusterización es de 0.75, el camino característico es $|V|/\langle k \rangle$, y la distribución del grado de los nodos es una delta en el valor $\langle k \rangle$. Es decir tanto L como C son más grandes de lo que correspondería en el grafo aleatorio equivalente y la distribución de grado se puede ver como el límite cuando la varianza tiende a 0.

%08/05 - Carlos Aguirre
\section{Redes de mundo pequeño}
En 1998 Watts y Strogatz (Nature, 1998) proponen un modelo de red dependiente de un parámetro p, que es distinto al parámetro p de las redes aleatorias. 
\marginpar[\footnotesize Buena pregunta de 1 punto] \
En redes aleatorias, significa la existencia de las ramas, pero aquí es la probabilidad de reasignación o recolocación de cada una de las ramas que ya tiene el grafo. 
 Este modelo interpola entre un grafo regular y un grafo aleatorio. Se colocan inicialmente los nodos en un anillo y cada nodo se conecta con los 2k vecinos a izquierda y derecha.

Para cada rama de este grafo, con probabilidad p se decide si la rama se modifica o no.
Si la rama se modifica, se elige un nuevo nodo al azar con probabilidad uniforme. Se evitan ramas dobles y autoconexiones. Este proceso produce pNk atajos en el grafo.

\begin{figure}[h]
\centering
\includegraphics[width = 0.5\textwidth]{figs/grafos.png}
\end{figure}

El método de Watts y Strogatz puede producir grafos no conexos. Un método alternativo propuesto por Newman consiste en añadir de forma aleatoria un número pequeño de ramas sobre el sustrato inicial (duplicación y manteniendo la rama original).

\subsection{Sustrato inicial}
Como sustrato inicial se suele tomar un grid monodimensional cumpliendo las siguientes condiciones:
$$N \gg k \gg \log(N)$$
Cada nodo esta conectado con sus 2k vecinos a izquierda y derecha.

Así, el grafo siempre será conexo ($k \gg \log(N)$) y disperso ($N \gg k$). 

\subsection{Sustratos bi-conexos}
Un grafo conexo tiene un camino que une cualquier par de nodos. Una red biconexa es aquella que para cualquier par de nodos tiene al menos dos caminos que los unen. Por ejemplo, un anillo es una red biconexa porque se puede llegar por la izquierda y por la derecha. Las redes biológicas suelen ser biconexas, pero no todas lo son. Calcular tres o más caminos es un problema NP completo - buscar uno es fácil mediante un algoritmo en anchura y buscar dos se puede hacer con el algoritmo de Tarjan.

El modelo de red original de Watts y Strogatz tiene el mismo número de componentes biconexas que los grafos regulares. Las redes reales suelen tener un número mayor de componentes biconexas (p. ej. redes de comunicaciones o muchas redes biológicas).

Los grafos de mundo pequeño tienen el mismo índice de clusterización que un grafo aleatorio regular con el mismo número de ramas y nodos. Camino característico de un grafo aleatorio con el mismo número de nodos y ramas.

El modelo anterior permite crear redes de tipo Mundo-pequeño con un número elevado de componentes bi-conexas. El sustrato inicial es un grafo regular pero con un numero elevado de componentes biconexas. Presenta una transición a aleatorio similar a la de los anillos.

\subsection{Sustratos dirigidos y ponderados}
El modelo original de Watts y Strogatz no contempla la posibilidad de grafos con dirección y peso. Existen redes en la naturaleza donde aparecen la direccionalidad y el peso (redes neuronales, redes sociales, redes de interacción, etc).

\subsection{Parámetros}
El procedimiento de Watts y Strogatz sobre el sustrato inicial produce una transición en el comportamiento del camino característico y del índice de clusterización. Ambos dos han de pasar de valores altos propios del grafo regular a valores pequeños propios del grafo aleatorio.

En una malla monodimensional $L = O(N)$ y $C = O(1)$. En un grafo aleatorio: $L = O(\log(N))$ y $C = 0 = p$.

Watts y Strogatz observaron que la transición en el comportamiento de los valores era diferente para el
caso de camino característico L y para el índice de clusterización C. El camino característico presentaba la transición de régimen mucho antes que el índice de clusterización.

Cuando se consideran otros sustratos tales como grafos bi-conexos o grafos con dirección peso también se puede observar el mismo fenómeno en el comportamiento de C y L. Además las propiedades iniciales del grafo (bi-conectividad, dirección, peso) no se pierden durante la transición

L no empieza a decrecer hasta que $p > 1/Nk$ (es decir hasta que no aparece al menos un atajo). Por tanto, el valor de p para el cual se entra en la zona de mundo pequeño es dependiente de N y k. Para una probabilidad fija p, existe un valor N' tal que $L = O(N)$ si$ N < N'$ y $L=O(\log(N))$ sin $N > N'$. Se puede demostrar el valor de p que para un valor fijo de N y k produce la transición de L es $p = 1/kN$.

La distribución del grado de los nodos debe pasar de una delta (vale 1 en 0 y 0 en todos los demás) en el grado medio de cada nodo (2k) a una distribución de Poisson.

La distribución espectral (de la matriz de autovalores) también sufre una transición en función de p.
\end{document}
