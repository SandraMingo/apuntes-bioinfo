%06/05 - Carlos Aguirre
\chapter{Grafos aleatorios, grafos regulares y redes de mundo pequeño}
\section{Grafos aleatorios}
\subsection{Modelo de Erdös y Rényi}
En las redes reales muy raramente suele aparecer una topología aletoria. Sin embargo los grafos aleatorios han sido muy estudiados por las siguientes razones. Si encontramos una propiedad que ocurre con probabilidad 1 para ciertos parámetros de la red, podemos saber si nuestra red tiene esa propiedad solo con mirar sus parámetros. Para predecir el comportamiento de ciertas propiedades en función de parámetros de la red. Para comprobar si nuestra red tiene sesgos estructurales.

El modelo de Erdös y Rényi  se estudió en los años 50-60. Cada rama del grafo existe con una probabilidad p, que suele seguir una distribución uniforme.

Otra forma de definir los grafos aleatorios es seleccionar parejas de nodos aleatoriamente. Se seleccionan exactamente $pN(N-1)/2$ parejas. Ambos tipos de grafos aleatorios son equivalentes.

\subsection{Propiedades}
El estudio de grafos aleatorios se centra sobre todo en averiguar para que probabilidad p aparece cierta propiedad Q:
\begin{itemize}
\item Cuando el grafo es conexo.
\item Cuando la distancia media es menor que cierto número.
\item Cuando el índice de clusterización es mayor que cierto número.
\end{itemize}

Erdös y Rényi descubrieron que las propiedades Q aparecían de forma repentina según crecía p. Para muchas propiedades Q se verifica que existe una \textbf{probabilidad crítica} $p_c(N)$ tal que:
\begin{itemize}
\item Con probabilidad 0 el grafo no tiene Q si $p(N) < p_c(N)$
\item Con probabilidad 1 el grafo tiene Q si $p(N) > p_c(N)$
\end{itemize}

\subsection{Subgrafos}
La primera propiedad que estudiaron fue la aparición de subgrafos. Por ejemplo, a que probabilidad crítica p casi todo grafo G contiene un árbol de orden 3. 
La probabilidad crítica $p_c(N)$ de encontrar algunos subgrafos es:
\begin{figure}
\centering
\includegraphics[width = 0.7\textwidth]{figs/prob-critica.png}
\end{figure}

Ejercicio: Calcular la probabilidad crítica p para que una red aleatoria de 1000 nodos contenga tanto un ciclo de orden 5 como un cliqué de orden 5 (cuidado, tiene truco).
\marginpar[\footnotesize Pregunta examen antigua] \
El ciclo de orden 5 aparece con una Z de -1
Un cliqué de 5 aparece con una Z de -0.5. 
Para que haya un cliqué de orden 5 es obligatorio que haya un ciclo de orden 5. Por tanto, teniendo N = 1000 nodos, solo habría que calcular la probabilidad de un cliqué: $1000^{-1/2} = 0.031$.

\subsection{Clusters}
Un subgrafo aislado y conexo es un
cluster. Erdös y Rényi demostraron que la estructura de clusters de un grafo
cambia abruptamente cuando el grado medio (número de ramas dividido entre el número de nodos) se acerca a 1.
$$\langle k \rangle = \frac{\frac{N \cdot (N-1)}{2} \cdot p}{N} = \frac{p \cdot (N-1)}{2} \xrightarrow{N \to \infty} p\frac{N}{2}$$

Si el grado medio <k> está entre 0 y 1, casi todos los clusters son árboles (en su mayor parte) o clusters que contienen un solo ciclo. El número de clusters es de orden N-n (número de nodos menos número de ramas), y el cluster mayor es un árbol de tamaño proporcional a N. 

Si el grado medio <k> es mayor que 1, la estructura anterior cambia completamente. Aparece un cluster gigante con [1-f(<k>)]N nodos donde f es una función que decae exponencialmente de 1 a 0 cuando x va a infinito. Los demás clusters pertenecen a árboles con Nf(<k>) nodos.

\subsection{Distribución de grado}
El grado de cada nodo sigue una distribución binomial. La distribución del grado de los nodos sigue una distribución de Poisson. 

\subsection{Conexidad y diámetro}
El diámetro de un grafo es la máxima distancia entre cualquier par de nodos. Si p no es demasiado pequeño los grafos aleatorios tienden a tener poco diámetro. Casi todos los grafos aleatorios tienen el mismo diámetro (más o menos) para la mayor parte de los valores de p.

En general se tiene:
\begin{itemize}
\item Si $\langle k \rangle < 1$: el grafo tiene árboles aislados.
\item Si $\langle k \rangle > 1$: aparece el cluster gigante y el diámetro del grafo es el del cluster gigante. Para $\langle k \rangle > 3.5$, el diámetro es proporcional a $\ln(N)/\ln(<k>)$.
\item Si $\langle k \rangle \geq ln(N)$ : el grafo es conexo y su diámetro es próximo a $\ln(N)/\ln(\langle k \rangle)$
\end{itemize}

El camino característico se comporta de forma similar al diámetro. En particular:
 $$\ell_{rand} \sim \frac{\ln(N)}{\ln(\langle k \rangle)}$$
 
\subsection{Índice de clusterización}
En un grafo aleatorio la probabilidad de que dos vecinos de un nodo dado están conectados es igual a la que dos nodos elegidos al azar estén conectados.
\marginpar[\footnotesize Pregunta típica de examen] \
El índice de clusterización de un grafo alteatorio es p, la probabilidad de existencia de cada rama.

\section{Grafos regulares}
Son los mejor conocidos de forma analítica debido a sus simetrías. Existen expresiones cerradas para todas las métricas. Son los más usados en modelos de redes neuronales artificiales. También se usan como sustrato inicial para la generación de otros tipos de redes.

Un caso particular de grafos regulares son las mallas (grids) donde cada nodo se conecta a sus $2^d \cdot k$ vecinos más próximos, donde d es la dimensión del grid.

En las mallas monodimensionales, el índice de clusterización es de 0.75, el camino característico es $|V|/\langle k \rangle$, y la distribución del grado de los nodos es una delta en el valor $\langle k \rangle$. Es decir tanto L como C son más grandes de lo que correspondería en el grafo aleatorio equivalente y la distribución de grado se puede ver como el límite cuando la varianza tiende a 0.

\section{Redes de mundo pequeño}
En 1998 Watts y Strogatz (Nature, 1998) proponen un modelo de red dependiente de un parámetro p. Este modelo interpola entre un grafo regular y un grafo aleatorio. Se colocan inicialmente los nodos en un anillo y cada nodo se conecta con los 2k vecinos a izquierda y derecha.

Para cada rama de este grafo, con probabilidad p se decide si la rama se modifica o no.
Si la rama se modifica, se elige un nuevo nodo al azar con probabilidad uniforme. Se evitan ramas dobles y
autoconexiones. Este proceso produce pNk atajos en el grafo.

\begin{figure}[h]
\centering
\includegraphics[width = 0.5\textwidth]{figs/grafos.png}
\end{figure}

El método de Watts y Strogatz puede producir grafos no conexos. Un método alternativo propuesto por Newman consiste en añadir de forma aleatoria un número pequeño de ramas sobre el sustrato inicial.

\subsection{Sustrato inicial}
