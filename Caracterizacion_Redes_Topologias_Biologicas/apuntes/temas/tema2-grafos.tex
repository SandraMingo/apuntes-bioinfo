%20/03 - Carlos Aguirre
\chapter{Teoría de grafos y métricas}
\section{Introducción a la teoría de grafos}
La teoría de grafos ha sido utilizada recientemente para:
\begin{itemize}
\item Clasificación automática de secuencias de proteínas.
\item Detección de jerarquías de proteínas.
\item Análisis de redes genéticas.
\item Reconstrucción de redes genéticas grandes obtenidas mediante modificación de genes.
\end{itemize}

Un grafo G es un par de conjuntos (V,E) donde $V = \{v_1, v_2, \ldots v_n\}$ es el conjunto de vértices o nodos y $E = \{(v_i, v_j), (v_{i'}, v_{j'}), \ldots \}$ es un conjunto de pares no ordenados de elementos de V y se denomina conjunto de ramas del grafo. 
\marginpar[\footnotesize Pregunta de test: define orden y tamaño, dado un grafo dar el orden y tamaño, etc. ]  \
El número de nodos se denomina \textbf{orden} del grafo, y el número de ramas es el \textbf{tamaño} del grafo. 

\begin{figure}[h]
\centering
\includegraphics[width = 0.6\textwidth]{figs/grafo.png}
\caption{Ejemplo de grafo de orden 8 y tamaño 11.}
\end{figure}

Para una red de proteínas, cada proteína sería un nodo del grafo, y una rama indicaría interacción entre ambas proteínas. 

Una disposición (layout) es una posible colocación de los nodos y las ramas en un espacio 2D o 3D. Un mismo gráfo puede tener múltiples colocaciones. Ejemplo, consideremos el grafo G=(V,E).

\begin{figure}[h]
\centering
\includegraphics[width = 0.6\textwidth]{figs/layout.png}
\end{figure}

Existen programas de ordenador que nos permiten obtener colocaciones predefinidas (Gephy, Pajek). Cuando no se especifica ninguna colocación, se entiende que los nodos se sitúan aleatoriamente sobre el plano o espacio. Algunos de los tipos más habituales de colocaciones son:
\begin{itemize}
\item Colocaciones regulares
\item Basadas en la física (atracción-repulsión)
\item Basadas en propiedades topológicas (jerarquías, número de vecinos, etc)
\end{itemize}

Un hipergrafo H es un también par de conjuntos (V,E) donde $V=\{v_1, v_2, \ldots v_n\}$ es el conjunto de vértices o nodos y $E=\{(v_{i1}, v_{i2}, \ldots),(v_{i'1},v_{i'2}, \ldots), \ldots\}$ es una familia de
subconjuntos no ordenados de elementos de V. E se denomina conjunto de hiperramas o hiperaristas del hipergrafo. El número de hiperramas $|E|$ se denomina cardinalidad del hipergrafo. El valor $|E|*|V|$ se denomina tamaño o volumen del grafo.
\marginpar[\footnotesize Pregunta examen ]  \
Si tenemos un grafo de n nodos, ¿cuántas parejas podemos tener como máximo? $(n \cdot n-1)/2$ Por tanto, en un grafo con n nodos, ¿cuántas ramas puede tener? Igual, $(n \cdot n-1)/2$

\begin{figure}[h]
\centering
\includegraphics[width = 0.6\textwidth]{figs/hipergrafo.png}
\caption{Ejemplo de hipergrafo de cardinalidad 4 y tamaño 32.}
\end{figure}

Un hipergrafo H se dice que es \textbf{propio} si no es vacío ($V\neq \varnothing$) y no contiene ninguna arista vacía. Un hipergrafo H se dice que tiene \textbf{dominio completo} si todos los nodos están en al menos una arista, en caso contrario se dice que tiene \textbf{dominio parcial}. Si en un hipergrafo todas las hiperramas tienen el mismo número de nodos, entonces se denomina \textbf{hipergrafo k-uniforme}. 

\textit{Ejercicio: Indicar si el hipergrafo del ejemplo anterior es propio, tiene dominio completo y si es k uniforme}. Es propio (el conjunto de vértices tiene 8 elementos y todas las ramas e tienen vértices dentro), es de dominio parcial (v5 no está en ninguna rama) y no es k-uniforme (e1 tiene 3 elementos, e2 tiene 2, e3 tiene 1 y e4 tiene 4).

\section{Bucles y ramas paralelas}
Un bucle es una rama que empieza y termina en el mismo nodo $(v_i, v_i)$. Cuando dos ramas conectan el mismo par de vértices se denominan paralelas. Un grafo con bucles se denomina pseudografo. Un grafo con ramas paralelas pero sin bucles se denomina multigrafos. Un grafo sin bucles ni ramas paralelas se denomina grafo simple.

\begin{figure}[h]
\centering
\includegraphics[width = 0.6\textwidth]{figs/bucle-rama-paralela.png}
\end{figure}

%25/03 - Carlos Aguirre
\section{Grafos dirigidos y ponderados}
Se puede considerar que los enlaces entre nodos son dirigidos $(v_i, v_j) = (v_j, v_i)$. Los grafos dirigidos se denominan también \textbf{digrafos}.

\begin{figure}[h]
\centering
\includegraphics[width = 0.6\textwidth]{figs/grafo-dirigido.png}
\end{figure}

En los grafos ponderados, a cada rama del grafo se le puede asociar un número. El número asociado a cada rama puede indicar entre otras cosas una distancia, una capacidad, un valor temporal, etc.

\begin{figure}[h]
\centering
\includegraphics[width = 0.6\textwidth]{figs/grafo-ponderado.png}
\end{figure}

\newpage

Los grafos dirigidos y ponderados poseen ramas dirigidas a las que se asocia un número.

\begin{figure}[h]
\centering
\includegraphics[width = 0.6\textwidth]{figs/grafo-dirigido-ponderado.png}
\end{figure}

\section{Grado de un nodo}
Dos nodos de un grafo son \textbf{vecinos o adyacentes} si existe una rama que los conecta. El \textbf{grado} de un nodo es el número vecinos que tiene dicho nodo. En los grafos dirigidos se calcula el \textbf{grado de entrada} y el \textbf{grado de salida}. En los grafos ponderados, el grado se puede promediar por el número asociado a las ramas. Un grafo se dice que es \textbf{regular} si todos los nodos tienen el mismo grado.

\begin{figure}[h]
\centering
\includegraphics[width = 0.6\textwidth]{figs/grafo-grados.png}
\end{figure}

\section{Subgrafos}
Un grafo G’=(V’,E’) es un subgrafo de un grafo G=(V,E) si V’ es un subconjunto de V y E’ es un subconjunto de E. En otras palabras, un subgrafo es un trozo de un grafo más grande. 

\begin{figure}[h]
\centering
\includegraphics[width = 0.6\textwidth]{figs/subgrafo.png}
\end{figure}

Un subgrafo G’=(V’,E’) de un grafo G=(V,E) se dice que es \textbf{abarcador} si V=V’, es decir, si están todos los nodos, pero faltan algunas ramas.

Un grafo es un subgrafo de sí mismo. Además, un grafo vacío es un subgrafo de cualquier grafo.

\section{Paseos, caminos, circuitos y ciclos}
Un \textbf{paseo} de un nodo u a un nodo v es una secuencia de vértices $\{v_0, v_1, \ldots, v_k\}$ con $v_1 = u v_k = v$ y $(v_{i-1}, v_i)$ rama del grafo. El número de ramas del paseo es su \textbf{longitud}. Un paseo en el cual no se repiten ramas se denomina \textbf{rastro}. Un paseo en el cual todos los vértices $\{v_0, v_1, \ldots, v_k\}$ son distintos se denomina \textbf{camino}. 
Un camino siempre debe ser un rastro y un paseo. Si algo no es rastro, no puede ser camino, y si no es paseo, no puede ser ni rastro ni camino. Cada uno es cada vez más restrictivo.

Entre dos nodos, puede haber varios caminos posibles. Un \textbf{camino mínimo} entre dos nodos es aquel de menor longitud de entre todos los posibles caminos entre ambos nodos. La \textbf{distancia} entre dos nodos del grafo se define como la longitud de cualquier camino mínimo que los una.

\begin{figure}[h]
\centering
\includegraphics[width = 0.5\textwidth]{figs/paseo-rastro.png}
\end{figure}

\begin{figure}[h]
\centering
\includegraphics[width = 0.5\textwidth]{figs/camino-minimo.png}
\end{figure}

Un \textbf{paseo cerrado} es un paseo $\{v_0, v_1, \ldots, v_k\}$ tal que $v_0 = v_k$. Un paseo cerrado en el que no se repiten ramas es un \textbf{circuito}. Un \textbf{ciclo} es un circuito en el que no se repiten vértices. Los ciclos son importantes, porque las redes biológicas tienen ciclos (que suelen ser largos), pero en las redes aleatorias no aparecen ciclos, o éstos son muy pequeños.

\begin{figure}[h]
\centering
\includegraphics[width = 0.5\textwidth]{figs/ciclo.png}
\end{figure}

El nodo con menor distancia entre los demás es muy importante, denominándose como \textbf{centro del grafo}.

Para un grafo con excesivos nodos, los caminos mínimos y las distancias se calculan con un algoritmo. Si el grafo es no ponderado, se utiliza el algoritmo búsqueda en anchura, mientras que si es ponderado, utiliza Dijkstra.

\section{Medidas de centralidad, betweeness y closeness}
\marginpar[\footnotesize Pregunta examen: Betweeness/Closeness/Farness se define como... ]  \
Dado un nodo $v_i$ se define su \textbf{betweeness} $C_B (v_i)$ como la fracción de caminos mínimos que hay entre el resto de nodos del grafo y que pasan por el nodo $v_i$. Es decir, se hacen parejas de todos los nodos del grafo excluyendo el nodo de interés, y se calculan los caminos mínimos. Algunos pasarán por el nodo de interés, que son los que nos quedamos. Con eso se evalúa el cociente (los que pasan por ese nodo entre todos), que será el betweeness (un valor entre 0 y 1).
La centralidad de un nodo es muy costosa de calcular, usualmente se emplean algoritmos aproximados. 

Dado un nodo $v_i$ se define su \textbf{lejanía o farness} $C_F (v_i)$ como la suma de las distancias de $v_i$ al resto de nodos del grafo. 

Dado un nodo $v_i$ se define su \textbf{cercanía o closeness} $C_C (v_i)$ como la inversa de su lejanía $C_C (v_i) = 1/C_F (v_i)$.

\begin{figure}[h]
\centering
\includegraphics[width = 0.5\textwidth]{figs/medidas-centralidad.png}
\caption{Respuesta al ejercicio: Cogiendo v4, la lejanía será 2+1+2+1+1+2+2 = 11, y la cercanía 1/11.}
\end{figure}

La cercanía y lejanía tiene un problema: su valor numérico depende del orden del grafo. Por tanto, sirve para comparar dentro del mismo grafo, pero no entre grafos. 
\marginpar[\footnotesize Pregunta examen: Calcular camino característico ] \
Para eso, habría que normalizar dividiendo por el número total de nodos. A esto se le conoce como \textbf{camino característico}.

\section{Conexidad}
Un grafo es \textbf{conexo} si para cada par de nodos del grafo existe al menos un camino que los une. En otras palabras, que no esté separado en distintos trozos. 

\begin{figure}[h]
\centering
\includegraphics[width = 0.6\textwidth]{figs/conexidad.png}
\end{figure}

Hay un algoritmo muy rápido y eficiente que calcula si un grafo es conexo o no. 

Una \textbf{componente conexa} de un grafo es cada uno de los subgrafos
maximales conexos. Esto quiere decir que el subgrafo no puede ser más grande, que no se le puede añadir más nodos.

\begin{figure}[h]
\centering
\includegraphics[width = 0.5\textwidth]{figs/componentes-conexas.png}
\end{figure}

Un \textbf{punto de articulación} es un nodo que desconecta un grafo conexo. Un \textbf{corte} es un conjunto de ramas que desconecta un grafo conexo. Si un corte esta compuesto por una única rama, se denomina \textbf{puente}. Un \textbf{corte mínimo} de un grafo es el mínimo número de ramas que al ser eliminadas desconectan el grafo.

El algoritmo CLICK (CLuster Identification via Connectivity Kernels) calcula una aproximación al corte mínimo. Esto lo hacían cogiendo los dos nodos más lejanos. Los puentes suelen ser muy malos para la conexidad de los grafos. 

\begin{figure}[h]
\centering
\includegraphics[width = 0.5\textwidth]{figs/conexidad2.png}
\end{figure}

La máxima distancia entre cualquier par de nodos se denomina como diámetro.

El corte mínimo entre dos nodos es siempre mayor que el corte mínimo de todo el grafo.  

\section{Bosques y árboles}
Un grafo sin ciclos (acíclico) se denomina bosque. Un árbol es un grafo acíclico conexo. Cada componente conexa de un bosque es un árbol.

\begin{figure}[h]
\centering
\includegraphics[width = 0.6\textwidth]{figs/arbol-bosque.png}
\end{figure}

Un subgrafo abarcador acíclico de un grafo G se denomina un \textbf{bosque abarcador}. Un subgrafo abarcador conexo acíclico de un grafo G se denomina un \textbf{árbol abarcador}.

\begin{figure}[h]
\centering
\includegraphics[width = 0.6\textwidth]{figs/arbol-abarcador.png}
\end{figure}

%27/03 - Carlos Aguirre
\section{Grafos bipartitos}
Un grafo se dice que es bipartito si:
\begin{itemize}
\item El conjunto de vértices V se puede romper en dos subconjuntos disjuntos V1 y V2.
\item El vértice inicial de cada rama de E pertenece a V1 y el vértice final a V2
\end{itemize}

Ejercicio: el grafo anterior (el del árbol abarcador), ¿es bipartito? No, porque no es posible realizar la partición. Si V1 pertenece al conjunto 1, V2 y V3 deben estar en el conjunto 2, por lo que V4 y V5 tienen que estar en V1, pero esto no es posible porque están conectados entre sí. La condición necesaria para que un grafo sea bipartito es que no tenga triángulos. Pero esto no es suficiente; se puede construir un grafo sin triángulos, pero que tampoco sea bipartito. Si cogemos solo el cuadrado V4-V7, sí se podría generar un grafo bipartito: V4 y V7 en un conjunto y V5 y V6 en otro. 

\section{Representación de grafos}
Hay dos formas estándar de representar un grafo en un ordenador:
\begin{itemize}
\item \textbf{Matriz de adyacencia}: consume mucha memoria, pero es fácil de añadir o eliminar ramas. Es fácil saber si existe una rama, pero es lento enumerar los vecinos de un nodo. Se pueden calcular los autovalores y autovectores.
\item \textbf{Lista de adyacencia}: tiene un consumo limitado de memoria, pero es costoso añadir o eliminar ramas. También es costoso saber si existe una rama, pero rápido enumerar los vecinos de un nodo.
\end{itemize}

\section{Métricas sobre grafos}
Los grafos se clasifican en función de unas determinadas métricas topológicas. Las métricas más empleadas son:
\begin{itemize}
\item Tamaño |E| y orden |V|
\item Dispersión: $\frac{2 |E|}{|V| (|V| - 1)}$ para un grafo no dirigido y $\frac{|E|}{|V| (|V| - 1)}$ para un grafo dirigido. Si el coeficiente es pequeño (0), el grafo es disperso, si es cercano a 1, es denso. En redes biológicas, los grafos suelen ser dispersos.
\item Distribución del grado de los nodos: división del grado de todos los nodos entre el número de nodos. El resultado es una distribución de probabilidad. En un grafo aleatorio, la distribución es de Poisson (como la gaussiana, pero sin valores negativos). En las redes biológicas, la distribución no será de Poisson, por lo que este será el primer test que se haga a los datos. 
\item Grado medio (<k>): media del grado de todos los nodos.
\item Coeficiente de agrupamiento (C)
\marginpar[\footnotesize Pregunta examen: Calcular C o L de un grafo ] \
\item Camino característico (L)
\end{itemize}

%01/04 - Carlos
\subsection{Coeficiente de agrupamiento C}
El coeficiente de agrupamiento (C) es un valor métrico \textbf{local} que mide el nivel de agrupamiento de los nodos. Es decir, mira un nodo y sus vecinos y mira el índice de clusterización. En redes biológicas, el índice de clusterización es alto. Para cada nodo v del grafo se obtiene su vecindario, es decir, el conjunto de nodos que son vecinos de v, el tamaño del vecindario coincide con el grado de v (kv). Se calcula el coeficiente
$$Cv = \frac{Nv}{kv(kv-1)/2} = \frac{\text{número real de ramas entre vecinos sin incluir el nodo v}}{\text{número máximo de ramas entre vecinos}}$$
donde Nv es el numero de ramas que hay entre los vecinos de v. El valor anterior se promedia entre todos los nodos del grafo. 

\begin{figure}[h]
\centering
\includegraphics[width = 0.8\textwidth]{figs/coeficiente-agrupamiento.png}
\caption{Figura de la izquierda: todos los vecinos del nodo v son vecinos entre sí. Figura de la derecha: la mitad de mis amigos son amigos entre sí (0,5).}
\end{figure}

Para un grafo, se calcula el coeficiente de agrupación es el valor de cada nodo dividido por el número de nodos - es decir, el promedio. Para simetría, se calcula el valor de una fracción de los nodos y se divide por esa fracción.

Ejercicio: calcular el coeficiente de agrupamiento C del siguiente grafo. Se sugiere utilizar simetrías para reducir el trabajo. Esto será igual en el examen. 

\begin{figure}[h]
\centering
\includegraphics[width = 0.3\textwidth]{figs/ejercicio-calculo-c.png}
\caption{Se deben calcular dos índices de clusterización: nodo superior (igual al nodo inferior) y nodo izquierdo (igual al nodo derecho). Empezando por el nodo superior, la fórmula quedaría como $2/(3*2/2) = 2/3 = 0,66$. Para el nodo izquierdo, quedaría $3/(3*2/2) = 1$. Ahora hay que sumar esos dos valores y dividir entre el número de nodos calculados: $(1 + 0,66)/ 2 = 1,66/2 = 0,83$.}
\end{figure}

\subsection{Camino característico L}
El camino característico (L) es un valor métrico \textbf{global} que mide el nivel grado de separación de los nodos. Para cada nodo v se calcula la distancia promedio a todos los demás nodos del grafo:
$$Lv = \sum^{|v|}_{k=1} d(v, v_k) / (|v| - 1)$$

Se calcula el promedio del valor anterior entre todos los nodos del grafo. Es como un farness normalizado al número de nodos del grafo.
$$L = \sum^{|v|}_{v=1} L_v / (|v| - 1)$$

En las redes biológicas, el camino característico suele ser cortito. 

Ejercicio: Calcular el camino característico L del grafo de la figura anterior. Se puede (y debe) volver a utilizar simetrías. 
$$L_{v1} = L_{v3} = 1/3 + 1/3 + 1/3 = 1$$
$$L_{v2} = L_{v4} = 1/3 + 1/3 + 2/3 = 4/3 = 1,33$$
$$L = 1/4 + 1,33/4 + 1/4 + 1,33/4 = 7/6 = 1,165$$

\section{Topologías}
\subsection{Grafos aleatorios}
Fueron estudiados principalmente por Erdos y Renyi en los años 50. Cada rama del grafo existe con una determinada probabilidad p. Erdos y Renyi estudiaron los valores de las métricas topológicas para diferentes valores de $p$. Para la grafos dispersos (p pequeña) se puede comprobar que tanto C (aproximadamente 0) como L (aproximadamemte Ln(|V|) son pequeños.

\subsection{Grafos regulares}
Son los mejor conocidos de forma analítica. Existen expresiones cerradas para todas las métricas. Para la grafos dispersos se puede comprobar que tanto C (aproximadamente 0.75) como L (aproximadamente |V|/<k>) son grandes. 

\subsection{Mundo pequeño}
Son grafos que presentan altos valores de C (aprox .8) y bajos valores de L (aprox ln(|V|). Se obtienen introduciendo un pequeño número de “atajos” en un grafo regular. Representan bien un gran número de redes tales como redes sociales.

Las redes biológicas son todas de mundo pequeño. 

Un grafo de mundo pequeño es aquel cuyo índice de clusterización es el mismo de un grafo regular con el mismo número de nodos y ramas, y cuyo camino característico es el mismo de un grafo aleatorio con el mismo número de nodos y ramas. 

\subsection{Grafos libres de escala}
Son grafos que presentan bajos valores de C (aprox 0) y bajos valores de L (aprox ln(|V|). Se obtienen mediante crecimiento de la red y enlace preferencial. Cuando la distribución de los nodos se dibuja en escala log-log aparece una línea recta. Representan bien un gran número de redes tales como internet o redes de reacciones químicas.

Para diferenciar estas redes de un grafo aleatorio en la escala log-log, ya que los grafos libre de escala son una recta mientras que los aleatorios siguen una distribución de Poisson.

\begin{figure}[h]
\centering
\includegraphics[width = 0.8\textwidth]{figs/metricas-estandar.png}
\end{figure}

Un grafo con 2000 vecinos en el que cada nodo tiene 8 vecinos, ¿cuántas ramas tiene el grafo? 8000. De cada nodo entran y salen 8 ramas. $8 \cdot 2000/2$ porque cada rama se cuenta dos veces, una en cada vecino.

\section{Algoritmos sobre grafos}
El algoritmo de búsqueda en anchura permite calcular un camino mínimo entre dos nodos de un grafo. Dijkstra es una versión del algoritmo anterior para grafos ponderados. Ambos algoritmos funcionan tanto en grafos dirigidos como no dirigidos. Los algoritmos nos permiten calcular las métricas sobre el grafo.

Si los grafos tienen bucles y ramas paralelas, el algoritmo no funciona. 