\section{Estimación del tiempo de divergencia}
Para estimar una datación, se pueden emplear tres métodos: tasa universas de sustitución, incluir secuencias heterocrónicas y mediante eventos biogeográficos. Las secuencias heterocrónicas son secuencias en las que la fecha de muestreo es informativa de las edades que se infieren sobre los ancestros. Los ancestros que se esperan inferir están más o menos a la misma escala temporal que las mediciones. Esto se suele aplicar a patógenos de rápida evolución (virus y bacterias) o ADN ancestral.

BEAST es actualmente único en su capacidad para estimar el árbol filogenético y los tiempos de divergencia simultáneamente. BEAST es un programa multiplataforma para el análisis bayesiano MCMC de secuencias moleculares. Está totalmente orientado a filogenias enraizadas, medidas en el tiempo, inferidas usando modelos de reloj molecular estrictos o relajados. Puede utilizarse como método de reconstrucción de filogenias, pero también es un marco para comprobar hipótesis evolutivas sin condicionarlas a la topología de un único árbol. BEAST utiliza MCMC para promediar espacio arbóreo, de modo que cada árbol se pondera proporcionalmente a su probabilidad posterior.

BEAST es un programa de entorno bayesiano computacionalmente potente. Está muy optimizado para ejecutar el análisis (pero no para preparar el análisis). El input file debería estar escrito en XML con toda la configuración del análisis. Por tanto, se hizo un asistente que ayudase en la configuración previa y crease de un archivo Nexus el XML: BEAUTi. Los mismos desarrolladores también crearon el programa TRACER que analiza el fichero generado por MCMC (como el resultado de BEAST) y FigTree. Las versiones más modernas incluyen con BEAST el programa de TreeAnnotator que resume la información de los árboles de BEAST en un solo árbol que incluye la probabilidad posterior de los nodos en el árbol final. 

En BEAUTi, se deben importar los ficheros (File > Import Data, no Open). Se puede guardar la sesión (no el XML) para poder posteriormente cargarla (con Open) y cambiar algún parámetro y ver el resultado. Se puede elegir el modelo de sustitución, el modelo del reloj y el árbol de partición. Si solo hay un árbol, solo aparece una partición, mientras que en caso de haber dos genes concatenados, aparecerían como dos particiones (porque se especificaría en el Nexus). 

\textbf{Definición de las fechas de las puntas}
Por defecto, se supone que todos los taxones tienen una fecha de cero (es decir, se supone que las secuencias se muestrearon al mismo tiempo). En este caso, las secuencias RSVA han sido muestreadas en varias fechas que se remontan a la década de 1950. Podemos ver las fechas de los taxones/consejos en la pestaña Consejos en la parte superior de la ventana principal. Para activar la edición, haga clic en el botón «use tip dates». El año real de muestreo se indica en el nombre de cada taxón (separado por el símbolo de subrayado) y podríamos simplemente editar el valor en la columna Fecha de la tabla para reflejar las fechas. Sin embargo, si los nombres de los taxones contienen la información de calibración (como en nuestro caso), entonces una forma conveniente de especificar las fechas de las secuencias en BEAUti es utilizar el botón «Adivinar fechas» en la parte superior del panel. Al hacer clic, aparecerá un cuadro de diálogo. Esta operación intenta adivinar las fechas a partir de la información contenida en el nombre del taxón. Funciona intentando encontrar un campo numérico dentro de cada nombre. Si los nombres de los taxones contienen más de un campo numérico (como las secuencias RSVA, más arriba), puede especificar cómo encontrar el que corresponde a la fecha de muestreo. Puede especificar el orden en que aparece el campo de fecha (por ejemplo, primero, último o varias posiciones intermedias) o especificar un prefijo (algunos caracteres que aparecen inmediatamente antes del campo de fecha en cada nombre). Para las secuencias RSVA puede seleccionar «Definido por un prefijo y su orden» y luego «último» en el menú desplegable para el orden y especificar «\_» como prefijo. En este cuadro de diálogo, también puede hacer que BEAUti añada un valor fijo a cada fecha adivinada. En este caso, el valor «1900» se ha añadido para convertir las fechas de 2 dígitos a 4 dígitos. De este modo, todas las fechas de los nombres de taxones indicadas como «00» se convertirían en «1900». Algunas de las secuencias en el archivo de ejemplo tienen fechas posteriores al año 2000, por lo que al seleccionar la opción correctamente, añadiendo 2000 a cualquier fecha inferior a 09. Cuando pulse OK, las fechas aparecerán en la columna correspondiente de la ventana principal. Puede comprobarlas y editarlas manualmente si es necesario. En la parte superior de la ventana puede establecer las unidades en que se indican las fechas (años, meses, días) y si se especifican en relación con un punto del pasado (como en el caso de años como 1984) o hacia atrás en el tiempo desde el presente (como en el caso de las edades de radiocarbono).

\textbf{Configuración del modelo de sustitución}
A continuación, haz clic en la pestaña Sitios de la parte superior de la ventana principal. Esto mostrará la configuración evolutivos de BEAST. Las opciones que aparecen dependen de si los datos son nucleótidos o aminoácidos. Los ajustes que aparecerán después de cargar el conjunto de datos serán los valores por defecto, así que tenemos que hacer algunos cambios. La mayoría de los modelos deberían resultar familiares. Para este ejercicio, haremos varios cambios. En primer lugar selecciona el modelo HKY, luego Estimated en el menú Base frequencies y, finalmente, selecciona Gamma en el menú Modelo de heterogeneidad de sitios, que permitirá la variación de la tasa entre los sitios de la alineación. Selecciona las 3 particiones: posiciones de codón 1, 2 y 3 para que cada posición de codón tenga su propia tasa de evolución.

\textbf{Configurar el modelo de reloj}
La tercera cosa que haremos es hacer clic en la pestaña Relojes en la parte superior de la ventana principal, y cambiar el modelo de reloj molecular a Reloj relajado Lognormal (No correlacionado) para tener en cuenta la heterogeneidad específica de cada linaje. La casilla de verificación Estimar aparece marcada porque deseamos estimar la velocidad del reloj (y al hacerlo los tiempos de divergencia).

\textbf{Árboles}
La pestaña Árboles permite especificar priors para cada parámetro del modelo. Lo primero que hay que hacer es especificar que deseamos utilizar el modelo Epidemiology: Birth-Death Basic Reproductive Number model como árbol a priori. Este es un \href{https://academic.oup.com/mbe/article/29/1/347/1750040}{modelo epidemiológico de Stadler et al. (2011, doi: 10.1093/molbev/msr217)} adecuado para datos de secuencias virales. Desde la versión siguiente al 1.8 de BEAST, esta opción deja de estar presente y se deberá utilizar BEAST2 para ello.

\textbf{Prioridades}
La pestaña Prioridades permite especificar prioridades para cada parámetro del modelo. Necesita especificar una distribución a priori para los parámetros de tasa relativa para las posiciones de codón 1, 2 y 3 (CP1.mu, CP2.mu, CP3.mu), así como para la media del reloj relajado lognormal no correlacionado (ucld.mean). Haz clic en el botón de la tabla situado junto a «CP1.mu». Aparecerá un cuadro de diálogo que permitirá especificar una distribución a priori. Selecciona la distribución Normal. Vamos a suponer una distribución normal con una media de 0 y una desviación típica de 1 (valor inicial = 1,0). Siguiendo el mismo procedimiento, selecciona la misma distribución para «CP2.mu» y «CP3.mu». En el caso de ucld.mean, configura el prior como Uniforme (Inferior = 0,0, Superior = 1,0E100, Valor inicial = 1.0).

\textbf{Configuración de las opciones MCMC}
Ignora la pestaña Operadores, ya que sólo contiene ajustes técnicos que afectan a la eficacia del programa MCMC. La siguiente pestaña, MCMC, proporciona ajustes más generales para controlar la longitud del MCMC y los nombres de los archivos. En primer lugar, tenemos la Longitud de la cadena. Este es el número de pasos que el MCMC hará en la cadena antes de terminar. La longitud depende del tamaño del conjunto de datos, la complejidad del modelo y la calidad de la respuesta requerida. El valor por defecto de 10.000.000 es totalmente arbitrario y debe ajustarse en función del tamaño del conjunto de datos. Para este ejercicio (por falta de tiempo), lo fijaremos en 1.000.000 \footnote{NOTA: Una configuración adecuada para este conjunto de datos sería una longitud de 10.000.000 y una frecuencia de muestreo de 1000 (tanto para los registros de pantalla como para los de archivo). tanto para los registros de pantalla como para los de archivo).}. Las siguientes opciones especifican la frecuencia con la que los valores de los parámetros de la cadena de Markov deben mostrarse en la pantalla y registrarse en el archivo de registro. La salida en pantalla es simplemente para monitorizar el progreso del programa, por lo que puede establecerse cualquier valor (aunque si se establece demasiado pequeño, la cantidad de información que se muestra en la pantalla ralentizará el programa). Para el archivo de registro, el valor debe establecerse en relación con la longitud total de la cadena. Si se muestrea con demasiada frecuencia, se obtendrán archivos muy grandes con poco beneficio adicional en términos de precisión del análisis. Si el muestreo es demasiado infrecuente, el archivo de registro no contendrá mucha información sobre las distribuciones de los parámetros. Es probable que se desee almacenar no más de 10.000 muestras, por lo que debe establecerse en un valor no inferior a longitud de cadena / 10.000. Para este ejercicio, estableceremos el registro de pantalla a 100 y el registro de archivo a 10. Las dos últimas opciones dan los nombres de los archivos de registro para los parámetros muestreados y los árboles. Estos se establecerán por defecto basándose en el nombre del archivo NEXUS importado. Si se utiliza Windows, se sugiere que añada el sufijo .txt a ambos (HRSV-A.log.txt y HRSV-A.trees.txt) para que Windows los reconozca como archivos de texto. También puede crear un archivo de análisis de operador.

\textbf{Creación del archivo XML BEAST}
Ahora estamos listos para crear el archivo XML BEAST. Para ello, selecciona la opción Generar archivo BEAST... del menú Archivo o haz clic en el botón del mismo nombre situado en la parte inferior de la ventana. Guarda el archivo con un nombre apropiado (normalmente terminamos el nombre del archivo con .xml, es decir, HRSV-A.xml). Ahora estamos listos para ejecutar el archivo a través de BEAST.

\textbf{Ejecutar BEAST}
Ahora ejecuta BEAST y cuando pida un archivo de entrada, introduce el archivo XML que se acaba de crear. BEAST se ejecutará hasta que termine de mostrar la información en pantalla. Los archivos de resultados se guardan en el disco en la misma ubicación que el archivo de entrada.

\textbf{Análisis de los resultados}
Ejecuta el programa TRACER para analizar los resultados de BEAST. Cuando se haya abierto la ventana principal, selecciona Import Trace File... en el menú File y selecciona el archivo que BEAST ha creado llamado HRSV-A.log.txt. Recuerda que MCMC es un algoritmo estocástico, por lo que los números reales no serán exactamente los mismos entre dos ejecuciones. En la parte izquierda, hay una lista de las diferentes cantidades que BEAST ha registrado. Hay trazas para la probabilidad posterior (que es el logaritmo del producto de la probabilidad del árbol y las probabilidades a priori) y los parámetros continuos. Al seleccionar una traza a la izquierda, aparecen análisis para esta traza a la derecha, dependiendo de la pestaña seleccionada. Cuando se abre por primera vez, se selecciona la traza «posterior» y se muestran varios estadísticos de esta traza en la pestaña Estimaciones. En la parte superior derecha de la ventana aparece una tabla con los estadísticos calculados para la traza seleccionada. Tracer trazará una distribución (marginal posterior) para el parámetro seleccionado y también le proporcionará estadísticas como la media y la mediana. El 95\% HPD significa intervalo de máxima densidad posterior y representa el intervalo más compacto sobre el parámetro seleccionado que contiene el 95\% de la probabilidad posterior. Puede considerarse como un análogo bayesiano de un intervalo de confianza. Tracer también proporciona el Tamaño Muestral Efectivo (ESS de Effective Sample Size) de un parámetro. En una ejecución MCMC, el ESS es el número de extracciones efectivamente independientes de la distribución posterior a la que equivale la cadena de Markov. Si el ESS de un parámetro es pequeño, la estimación de la distribución posterior de ese parámetro será pobre. Por lo tanto, ¿qué tamaño de ESS es adecuado? Cuanto más grande, mejor. Tracer marca ESSs < 100 pero esto puede ser liberal y > 200 sería mejor. Por otro lado, perseguir ESSs > 10000 puede ser un desperdicio de recursos computacionales. Tracer le permite explorar los datos de sus resultados en detalle. Por ejemplo, puede superponer los gráficos de densidad de varias trazas para compararlas (el usuario debe determinar si son comparables en el mismo eje o no). En nuestro caso, seleccione las tasas de sustitución relativas para las tres posiciones de codón en la tabla de la izquierda (etiquetadas CP1.mu, CP2.mu y CP3.mu), y verá las densidades de probabilidad posterior para la tasa de sustitución relativa en las tres posiciones de codón superpuestas.

\textbf{Uso de TreeAnnotator para generar el cronograma de consenso}
BEAST produce una muestra de árboles plausibles junto con su muestra de estimaciones de parámetros. Estos deben ser resumidos utilizando el programa TreeAnnotator. Este programa tomará el conjunto de árboles y encontrará el que mejor soporte tenga. A continuación, anotarás este árbol resumen con las edades medias de todos los nodos y los rangos HPD. También calcularás la probabilidad de clado posterior para cada nodo. Haz doble clic en el icono del programa. Selecciona el burnin apropiado (normalmente el 10\% de los árboles guardados). Puede especificarse como el número de estados o el número de árboles. En nuestro caso, especificaremos Burnin como árboles, escribiendo 1000 en el campo burnin. Selecciona el árbol de entrada (HRSV-A.trees) y el nombre del archivo de salida (utiliza la extensión .tre). La opción Límite de probabilidad posterior especifica un límite tal que si un nodo se encuentra con menos de esta frecuencia en la muestra de árboles (es decir, tiene una probabilidad posterior menor que este límite), no se anotará. El valor predeterminado de 0,0 significa que se anotarán todos los nodos. Para Árbol objetivo deja Árbol de máxima credibilidad de clado, que encuentra el árbol con el producto más alto de la probabilidad posterior de todos sus nodos. Elige Alturas medias para las alturas de los nodos. Esto establece las alturas (edades) de cada nodo en el árbol a la altura media a través de toda la muestra de árboles para ese clado. Ahora pulsa «Ejecutar» y espera a que el programa termine.

\textbf{Visualización del resumen del Árbol}
Puedes ver el cronograma resumido (archivo de salida de TreeAnnotator) con FigTree. Puedes probar a seleccionar algunas de las opciones del panel de control de la izquierda. Intenta seleccionar Barras de nodos para obtener barras de error de la edad de los nodos. También activa y selecciona posterior para que muestre la probabilidad posterior de cada nodo. En Apariencia también puedes decirle a FigTree que coloree las ramas según el índice.