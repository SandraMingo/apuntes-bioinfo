%03/10 - Anna Laura Mastrangelo
\section{Gut microbiota-derived metabolites: discovery of biomarkers and therapeutic targets in CVDs}
\textbf{Annalaura Mastrangelo - Immunology Lab CNIC}

Atherosclerosis (AT) is a silent precursor of cardiovascular diseases (CVDs). Traditional cardiovascular risk factor-based scores fail to identify individuals at risk at early stages. The rate of CV events remains high despite good cholesterol control. The only treatments available today are lipid-lowering. However, microbiota-host crosstalk has been suggested as a contributor to AT. Microbial imidazole propionate (ImP) is associated with complex diseases and all-causes mortality like Alzheimer's.

How does the gut microbiota affect AT? A murine model was used with different diets and antibiotics after 4 weeks. Antibiotics were able to decrease the progression of atherosclerosis. Plasma metabolome is altered by diet and antibiotics, and the microbiome diversity is reduced in HC diet. The metabolite TMAO was already associated to AT in the literature, but ImP, a microbial metabolite, was also associated in mice. 

Higher plasmatic ImP is independently and strongly associated with subclinical AT, particularly active AT. ImP shows additive value when included to established AT biomarkers. To validate the biomarker, the patophysiology of AT and the role of ImP must be studied, as well as if there is a causal role. 

Two murine models with AT were used to see if the metabolite alone was able to induce the disease. ImP in drinking water was able to induce the disease in both models. So, ImP induces AT in proAT mice fed chow diet without affecting cholesterol levels in plasma. In the blood, ImP administration expanded proinflammatory Ly6C high monocytes, T-helper 17 (Th17) and Th1 cells. ImP administration induced an increase in fibroblasts, endotelial cells and immune cells, particularly T and B cells. 

ImP exerts its role on its targets cell by acting on the imidazoline 1 receptor (I1R Nisch), which is blocked by AGN192403. In vivo, ImP drived atherosclerosis via I1R in myeloid cells. AGN was able to prevent AT progression upon high-cholesterol disease.

ImP is associated with AT in mice and humans, possibly serving as a biomarker of early and active AT. 
ImP alone induces AT by activating proatherogenic systemic innate and adaptative responses, without influencing bloodstream cholesterol. 

The ongoing project is to test synergistic therapy with lipid-lowering treatments and generation of new molecules blocking the ImP/I1R axis. In addition, the development of a MS-based diagnostic tool to establish robust, standardized MS-based methodologies and ready-to-use kits for the reliable and reproducible quantification of ImP in biofluids for clinical applications. In clinics, it is important to define the use of high ImP as a marker for prognosis of future cardiovascular events and define physiological and pathological ranges of ImP in biofluids, together with testing the novel diagnostic devices for ImP quantification in clinical samples.

%You spoke about the diet and different diets that were used with the mice, but was the microbiota also analyzed? Like, if specific genus are more likely to produce ImP, if there are different "gut-types" that are more susceptible to ImP production or something like that. Or is it completely irrelevant?

