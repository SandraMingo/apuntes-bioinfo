%26/09 - Val Fernández
\section{Bioinformática en la investigación dentro de institutos de investigación sanitaria}
\textbf{Val Fernández - investigador Instituto Ramón y Cajal de Investigación Sanitaria}

Hay dos grandes fuentes en ciencia para dar financiación: ministerio de ciencia (agencia estatal de investigación) y ministerio de sanidad (Instituto de Salud Carlos III). Este último financia sólo proyectos relacionados con la sanidad, el otro no. La agencia estatal financia todo, denominado como "plan nacional". El ISCIII tiene una cogobernanza con el ministerio de ciencia. Pueden pedir financiación IIS y CNIO/CNIC/CNE/etc y todos aquellos proyectos de universidades y CSIC siempre y cuando estén asociados a un IIS (FIS). 

Dentro del hospital están los IIS: IRYCIS, IdiPaz, etc. Pero también hay algunos grupos concretos dentro del IIS que están asociados a universidades. Dentro de los hospitales y los IIS están las fundaciones, que son los encargados de gestionar el dinero de investigación y que no se mezcle con el dinero destinado a tratar pacientes. 

Puede haber bioinformáticos clínicos contratados por el hospital y bioinformáticos contratados por la fundación. Esto hace que los bioinformáticos contratados no deberían hacer labor asistencial. 
En general se divide:
\begin{itemize}
\item Bioinformático en IIS: realiza investigación en un área temática determinada en un grupo de investigación.

\item Bioinformático clínico: realiza análisis bioinformáticos destinados a la actividad asistencial. Un bioinformático puede trabajar en la genética médica, oncología, hematología, microbiología, enfermedades infecciosas, inmunología, unidad de data science, etc. En la parte asistencial, la realidad es que depende de cada hospital. Normalmente los bioinformáticos clínicos están asociados a la genética médica, y la oncología está algo atrasada en cuanto a la secuenciación masiva. 

\item Bioinformático servicio: proporciona soporte bioinformático a los investigadores del instituto. Esto se cobra de los proyectos, pero es una unidad central a la que poder pedir soporte y ayuda. Entre todas las labores que se hacen es apoyo a las ómicas, asesoramiento y ayuda en el diseño de proyectos que incluyan análisis bioinformático, data mining, desarrollo de herramientas de análisis, formación del personal del IIS, etc. Se suele dar servicio a todos los grupos que no tengan un bioinformático. Cuando un grupo se va adentrando en un tema, suelen contratar a un bioinformático y no acuden al servicio. En algunos campos, todos los investigadores tendrán que aprender algo de bioinformática para poder al menos analizar sus datos. 
\end{itemize} 

Las ómicas siempre son distintas en los grupos de ciencia y los del hospital. No se puede calcular un tamaño muestral, se realizan análisis multivariante en donde se suele buscar qué variable nos sirve, y lo más importante, trabajamos con lo que podemos. Pero existen unos mínimos. 

Se debe tener en cuenta la variabilidad técnica-muestra. El problema de la muestra humana es que hay muchas variables distintas entre todos, y se debe tener muy en cuenta. Por ello, la complejidad del estudio en términos de grupos a comparar debe ser un balance entre ambición y realidad. Los grupos control son esenciales, y en ocasiones se necesita más de un grupo control. Algunas ómicas son dependientes de referencia y otras independientes, y algunas tienen referencias fácilmente buscables en el NCBI o bases de datos similares. 

En el análisis, se debe saber si existe un consenso en la comunidad, si existen pipelines estandarizados (en nf-core o similares) y si hay infraestructura para analizarlo. 

Un bioinformático de servicio debe saber de todo: análisis de secuencias, anotaciones de genomas, análisis de la expresión génica, análisis de la regulación, análisis de mutaciones, predicción de la estructura de las proteínas, genómica comparativa, modelado de sistemas biológicos, análisis de imagen de alto rendimiento, acoplamiento proteína-proteína, etc.

