\documentclass{config/apuntes}

\title{Fundamentos de Secuenciación de Alto Rendimiento y Genómica Traslacional}
\author{Sandra Mingo Ramírez}
\date{2024/25}
\acronym{HTSTRG}

\usepackage[all]{nowidow}
\usepackage{listing}
\usepackage{color}
\usepackage{tabularx}
\usepackage{multirow}
\usepackage{makecell}
\usepackage{amsmath}
\usepackage{array}
\usepackage{soul}

\definecolor{dkgreen}{rgb}{0,0.6,0}
\definecolor{gray}{rgb}{0.5,0.5,0.5}
\definecolor{mauve}{rgb}{0.58,0,0.82}

\lstset{
  frame=tb,
  aboveskip=3mm,
  belowskip=3mm,
  showstringspaces=false,
  columns=flexible,
  basicstyle={\small\ttfamily},
  numbers=none,
  numberstyle=\tiny\color{gray},
  keywordstyle=\color{blue},
  commentstyle=\color{dkgreen},
  stringstyle=\color{mauve},
  breaklines=true,
  breakatwhitespace=true,
  tabsize=3
}

\usepackage{tocloft}

\advance\cftchapnumwidth 0.5em\relax
\advance\cftsecnumwidth 0.5em\relax
\advance\cftsubsecindent 0.5em\relax
\advance\cftsubsecnumwidth 0.5em\relax
\begin{document}

\begin{abstract}
La asignatura aborda las tecnologías y metodologías actuales para la generación y análisis de datos multi-ómicos en biología y biomedicina, con un enfoque en técnicas de secuenciación masiva (NGS). Se estudian variantes genómicas puntuales y estructurales - como mutaciones, polimorfismos de nucleótido único (SNPs) y variantes en el número de copias génicas (CNVs) - que explican la variabilidad genética poblacional y su relación con enfermedades humanas.

A lo largo del curso, se aplican herramientas bioinformáticas para el análisis de datos reales, profundizando en la epidemiología molecular y los estudios de asociación genómica (GWAS) para explorar los factores de riesgo asociados a variantes genómicas en contextos clínicos y poblacionales. También se revisa el uso de datos genómicos en tratamientos personalizados, considerando su aplicación actual y potencial en la clínica.
\end{abstract}

\pagestyle{plain}

\maketitle

\tableofcontents

%Bloque 1: caracterización del genoma por NGS - introducción a genómica traslacional, métodos de secuenciación, aligners and NGS formats, avances tecnológicos en NGS, NGS for genome assembly, variantes estructurales
%Bloque 2: Genomic variants: techniques, variant calling and annotation - identification of genetic variants, practical exercise, annotation of genetic variants, filtering and prioritization of genetic variants
%Bloque 3: Genome-Wide Association Analysis (GWAS) - technologies, QC, statistical methods, practical exercises
%Bloque 4: Molecular epidemiology - biobanks, mendelian, oligogenic and complex traits, genetic susceptibility and PRS
%Examen tipo test 80% 
%Presentación 20%

%28/10 - Fátima Sánchez Cabo
\chapter{Introducción a la genómica: caracterización del genoma mediante NGS}
\section{Introducción a la genómica traslacional}
\subsection{Definición e importancia de la genómica}
La genómica, el estudio integral del ADN y de la estructura, función y dinámica de los genomas, representa un pilar fundamental en la biología moderna. Marcó un cambio de paradigma, pasando de un enfoque reduccionista en biología - donde se estudiaban componentes individuales y de manera aislada - a una perspectiva integradora que analiza las interacciones y relaciones entre los distintos elementos biológicos. Esta transición permitió evolucionar de la genética clásica, basada en hipótesis concretas, hacia la genómica, que integra análisis de datos masivos sin necesidad de preguntas iniciales específicas, aunque sí en constante búsqueda de respuestas biológicas complejas.

En el marco del dogma central de la biología, las “ómicas” representan tres niveles de estudio: la genómica (centrada en el ADN), la transcriptómica (ARN) y la proteómica (proteínas). Este curso se enfoca en la genómica, ya que la información genética determina las funciones bioquímicas y, por ende, los fenotipos de los organismos. Gracias a avances recientes, ahora es posible inferir la función bioquímica de las proteínas directamente a partir de la secuencia de ADN, sin necesidad de técnicas complejas como la cristalización. Además, herramientas de inteligencia artificial pueden predecir la estructura de las proteínas con precisión, acelerando la interpretación de funciones biológicas.

Las proteínas, incluyendo enzimas esenciales, son los elementos funcionales clave en la biología. La secuencia de aminoácidos en una cadena polipeptídica define sus propiedades funcionales, y, por tanto, conocer la secuencia genética subyacente (el ADN) facilita predecir la función de una proteína. Aunque determinar experimentalmente las propiedades de una proteína es complejo, la secuenciación genómica ha simplificado enormemente este proceso.

La mejora en tecnologías de secuenciación impulsó el \textbf{Proyecto Genoma Humano}, que logró identificar entre 20,000 y 25,000 genes y determinar la secuencia de los aproximadamente 3 mil millones de pares de bases del genoma humano. Este proyecto también fomentó la creación de bases de datos y herramientas para el análisis de datos genómicos, además de abrir el debate sobre los aspectos éticos, legales y sociales (conocidos como ELSI, por sus siglas en inglés), que siguen siendo temas vigentes y complejos en la actualidad.

\paragraph{Evolución de la bioinformática en la genómica}
La bioinformática ha crecido a la par de la genómica en múltiples niveles. Inicialmente, era una \textbf{disciplina} incipiente y se desarrollaba como apoyo experimental; sin embargo, ha evolucionado hasta convertirse en un campo esencial que impulsa la investigación. En cuanto a su \textbf{material, los datos,} la bioinformática ha tenido que adaptarse al fenómeno del big data, pasando de manejar cantidades limitadas de datos a enfrentar volúmenes masivos, propios de la genómica actual. Paralelamente, el \textbf{rol de los bioinformáticos} se transformó, pasando de ser técnicos a científicos de datos y académicos altamente reconocidos en la industria y en la investigación.

\begin{figure}[htbp]
\centering
\includegraphics[width = 0.5\textwidth]{figs/history-bioinfo.png}
\caption{Breve historia de la bioinformática en tres niveles: como disciolina, como material que utiliza y como las personas que trabajan en ella. Evolución desde 1980 hasta 2020.}
\end{figure}

\subsection{Avances tecnológicos en secuenciación}
Existen distintos tipos de tecnologías de secuenciación, comúnmente clasificadas en tres generaciones: la primera generación (first generation), la segunda o Next Generation Sequencing (NGS) y la tercera generación. Las dos primeras generaciones se enfocan en la secuenciación de fragmentos cortos de ADN, mientras que la tercera generación permite la lectura de fragmentos largos, facilitando el ensamblaje completo de genomas. Actualmente, uno de los mayores desafíos tecnológicos es detectar variantes de baja frecuencia y realizar secuenciaciones de ADN en células individuales (single-cell sequencing), lo cual tradicionalmente se hacía de forma masiva (“bulk”).

\begin{figure}[htbp]
\centering
\includegraphics[width = 0.7\textwidth]{figs/sequencing-generations.png}
\caption{Las tres generaciones de secuenciación y su forma de actuar.}
\end{figure}

A medida que el costo de la secuenciación ha disminuido y la capacidad de almacenamiento ha mejorado desde 1990, los datos generados también han crecido exponencialmente. En un experimento de secuenciación, los costos abarcan tanto la secuenciación en sí como el procesamiento bioinformático, el reporte y el almacenamiento de los datos. La comunidad científica y muchos journals requieren que los datos de proyectos financiados públicamente estén disponibles en bases de datos accesibles, lo que asegura la transparencia y el acceso a esta información valiosa. Para obtener una cobertura de calidad, el ADN suele secuenciarse al menos 30 veces, lo que genera archivos de gran tamaño, como los archivos FastQ, que almacenan información de secuencia y calidad para cada base.

\subsection{Procesos de llamada y priorización de variantes}
Los datos de secuenciación se procesan en pipelines bioinformáticas que comienzan con archivos FastQ normalmente comprimidos y pasan por varias etapas: control de calidad, alineamiento y llamada de variantes (variant calling). Las variantes identificadas pueden incluir cambios de nucleótidos, variaciones en el número de copias de segmentos genómicos (copy number variation) o reordenamientos estructurales.

\begin{figure}[htbp]
\centering
\includegraphics[width = 0.7\textwidth]{figs/bioinfo-pipeline.png}
\caption{Esquema de la pipeline que se sigue en bioinformática para la llamada de variantes.}
\end{figure}

La priorización de variantes se basa en factores como el impacto funcional, la frecuencia alélica en la población y la asociación con enfermedades. Sin embargo, muchas variantes requieren validación experimental, frecuentemente en modelos animales como ratones, para corroborar su relevancia funcional. El proceso de filtrado inicial se enfoca en variantes en exones de genes candidatos, analizando su frecuencia, patogenicidad y modelo de herencia; en caso de no hallarse variantes relevantes, se amplía el análisis a variantes oligogénicas o no codificantes.

\begin{figure}[htbp]
\centering
\includegraphics[width = 0.5\textwidth]{figs/variant-priorization.png}
\caption{Ejemplo de la priorización de variantes.}
\end{figure}

\subsection{Genómica en medicina de precisión}
La genómica ha transformado el enfoque de la medicina de precisión, permitiendo identificar enfermedades con bases genéticas, ambientales o una combinación de ambas. Algunas variantes genéticas confieren una predisposición a enfermedades sin ser causantes directas, lo cual es crucial para inferir relaciones causales y acelerar ensayos clínicos mediante la integración de grandes volúmenes de datos. Estas variantes pueden clasificarse en germinales (heredadas) o somáticas (adquiridas).

\begin{figure}[htbp]
\centering
\includegraphics[width = 0.5\textwidth]{figs/genetic-environment.png}
\caption{Representación gráfica de la relación entre enfermedades con base genética, ambientales o una mezcla de ambas.}
\end{figure}

En medicina de precisión, la genómica es solo una capa de datos entre muchas. Para una comprensión holística de la salud y la enfermedad, es necesario combinarla con información de otras “ómicas” como la transcriptómica, epigenómica, proteómica, metabolómica, y datos de microbioma. Además, los datos clínicos y epidemiológicos también forman parte del ecosistema de \textbf{Big Data Biomédico}, que actualmente se maneja mediante técnicas avanzadas de computación en clusters HPC, computación en la nube y algoritmos de GPU.

\begin{figure}[htbp]
\centering
\includegraphics[width = 0.7\textwidth]{figs/bigger-picture-bioinfo.jpg}
\caption{Esquema representando el dibujo general de la bioinformática.}
\end{figure}

Varias bases de datos públicas permiten estudiar la transición entre salud y enfermedad. El estudio de Farmingham, por ejemplo, lleva más de 70 años recolectando datos de factores de riesgo cardiovascular en más de 15,000 participantes. En Reino Unido, el Biobank y, en Estados Unidos, la iniciativa All of Us, también representan recursos de gran envergadura. En España, el CNIC (Centro Nacional de Investigaciones Cardiovasculares) realiza el estudio PESA (Progression of Early Subclinical Atherosclerosis), que ha contribuido a identificar factores predictivos de aterosclerosis subclínica mediante el estudio multiómico, generando nuevos indicadores con un mayor poder predictivo de la formación de placas de colesterol.

\paragraph{Epigenética y la medición de la edad biológica}
El perfil de metilación del ADN es un factor epigenético que puede modificar la expresión genética y se ha utilizado para calcular la “edad biológica” o epigenética de una persona, lo que puede servir como predictor de esperanza de vida y salud. Al comparar estos perfiles con la edad cronológica, sexo y otros factores, se obtiene información sobre el envejecimiento y el riesgo de enfermedades, facilitando el desarrollo de estrategias de salud personalizadas.

\subsection{Resumen}
La genómica ha liderado una revolución científica en el siglo XX, evolucionando desde el estudio de componentes individuales hasta una perspectiva integral de sistemas biológicos y de investigación basada en datos masivos. La bioinformática se ha convertido en una disciplina central en el análisis genómico y predicción de estructuras proteicas, impulsada por el Proyecto Genoma Humano y el desarrollo de tecnologías de secuenciación. Los avances actuales buscan no solo la secuenciación del ADN, sino también la integración de estos datos con datos epidemiológicos y moleculares para obtener una comprensión más profunda de la salud y la enfermedad. Así, el Proyecto Genoma Humano fue decisivo para sentar las bases de tecnologías de secuenciación, el desarrollo de la bioinformática en sí y el uso social e industrial de los datos ómicos.

La identificación de características genómicas relevantes causales de rasgos/enfermedades se basa en la anotación de variantes en bases de datos y en estudios poblacionales: hay margen de mejora y un gran éxito de la ciencia colaborativa. Hoy en día, los principales proyectos tratan no sólo de secuenciar el ADN, sino de integrar esta información con datos epidemiológicos y otros datos moleculares para comprender mejor la salud y la enfermedad.
Las enfermedades, en función de su base genética, pueden clasificarse en monogénicas (mendelianas), oligogénicas (ej., cardiopatías familiares) y complejas (evaluadas mediante puntuaciones de riesgo poligénicas). Esta clasificación permite avanzar en la medicina de precisión, abordando enfermedades desde su origen genético para ofrecer intervenciones de salud más efectivas y personalizadas.

%18/11 - Dido Carrero
\part{Variantes genómicas: técnicas, llamada de variantes y anotación}
\chapter{Introducción a las variantes germinales}
\section{Análisis genómico}
El análisis genómico incluye varios pasos: primero, la extracción de muestras y la preparación de las librerías; luego, la secuenciación, el control de calidad de los archivos FastQ (donde se descartan las lecturas con errores, ya que una mayor refinación del pipeline implica un control de calidad más estricto); el alineamiento de las lecturas; la identificación o llamada de variantes (SNP, INDEL, CNV, SV); la anotación de los archivos VCF; la visualización de las variantes candidatas y, finalmente, los pasos de priorización y filtrado.

En general, en un análisis de genoma, pueden encontrarse muchas variantes en comparación con el genoma de referencia, por lo que es necesario aplicar filtros para identificar aquellas que sean realmente relevantes a nivel clínico. La validación final se realiza en el laboratorio mediante PCR.

Las variantes germinales se originan en la línea germinal, es decir, en los gametos, lo que las hace heredables y presentes en todo el organismo. En cambio, las variantes somáticas ocurren en células que no pertenecen a los gametos, son mutaciones adquiridas durante la vida y afectan solo a un linaje celular específico.

En la práctica, ya sea que se realice un análisis somático o germinal, se extraen muestras tanto del tejido tumoral como del tejido sano. Si se sospecha de una enfermedad genética germinal, también se debe extraer una muestra de un tejido germinal.

La frecuencia alélica es la proporción de moléculas de ADN en la muestra que contienen una mutación específica. Se calcula mediante la siguiente fórmula:
$$VAF = \frac{sequence.reads.with.a.DNA.variant}{overall.coverage.at.that.locus}$$
Este número es clave para diferenciar una variante somática de una germinal. En un organismo diploide, un locus heterocigoto debería mostrar un VAF cercano a 0,5, un locus homocigoto tendrá un VAF de 1 y un locus de referencia tendrá un VAF de 0. Las variantes somáticas presentan una frecuencia alélica muy variable, mientras que las variantes germinales suelen tener valores de VAF de 0, 0,5 o 1, dependiendo de si están presentes en uno, ambos o ninguno de los alelos.

\subsection{GATK}
GATK es un conjunto de herramientas desarrollado por el Broad Institute para el análisis de variantes genómicas. A partir de archivos BAM, estas herramientas permiten realizar un análisis completo de variantes. El paquete incluye buenas prácticas y un flujo de trabajo (workflow) que varía dependiendo de si se analizan variantes germinales o somáticas.

\begin{figure}[h!]
\centering
\includegraphics[width = 0.8\textwidth]{figs/gatk-pipelines.png}
\end{figure}

HaplotypeCaller es una herramienta utilizada para la llamada de variantes, basándose en el cálculo de la probabilidad de los genotipos. Utiliza un archivo BAM como entrada y produce un archivo de salida en formato VCF o GVCF con los genotipos 1/1, 0/1 y 0/0. Este archivo VCF debe ser filtrado mediante recalibración de bases (una práctica recomendada) o mediante hard-filtering. Si el archivo de salida es un GVCF, será necesario realizar un paso intermedio antes de aplicar el filtro y continuar con el análisis posterior. Además, con la opción -ploidy, se puede especificar la ploidía del organismo.

El comando básico para la herramienta es:
\begin{lstlisting}[language = bash]
gatk HaplotypeCaller \
	-R reference.fasta \
	-I preprocessed_reads.bam \
	-O germline_variants.vcf
\end{lstlisting}

MuTect2 es una herramienta diseñada para la llamada de variantes somáticas. Permite detectar SNVs e INDELs, con frecuencias alélicas variables, y es capaz de diferenciar entre variantes somáticas y germinales. MuTect2 ofrece varios modos: tumor con normal emparejado, solo tumor o modo mitocondrial.

\section{Práctica: análisis de datos}
Vamos a recibir los datos de cáncer de mama. Se ha secuenciado todo el exoma con Illumina. Primero creamos el entorno conda \texttt{OVCA\_case}. 
\begin{lstlisting}[language=bash]
conda create -n OVCA_case
conda activate OVCA_case
conda install bioconda::gatk4
conda install bioconda::samtools
\end{lstlisting}

Con los datos descargados, utilizamos la herramienta HaplotypeCaller. Lo primero que debemos hacer es realizar los índices de la referencia y del fichero bam:
\begin{lstlisting}[language=bash]
samtools dict  REFERENCE/hg19_chr17.fa -o  REFERENCE/hg19_chr17.dict
samtools faidx REFERENCE/hg19_chr17.fa
samtools index bams/normal_refined.bam
\end{lstlisting}

A continuación utilizamos la herramienta:
\begin{lstlisting}[language=bash]
gatk HaplotypeCaller -R REFERENCE/hg19_chr17.fa -I bams/normal_refined.bam -O out/normal_refined_out.vcf
\end{lstlisting}

El fichero resultante empieza con una cabecera con dos almohadillas y el cromosoma de referencia. Se muestra la información acerca de la generación del fichero y los filtros. Con una almohadilla se muestra el significado de cada columna: cromosoma en el que está la variante, posición genómica, ID, alelo de referencia, alelo alternativo con la mutación encontrada, score de calidad, filtros, información adicional con la anotación, formato del siguiente campo y normal. Dentro del formato, se distinguen: GT indica el genotipo, AD el número de lecturas que soporta la variante (en formato referencia,variante) y DP el total de lecturas.

El siguiente paso es la recalibración de variantes. Muchas veces, la calidad de las variantes que aparece de manera directa (columna QUAL) se debe recalibrar. Este modelo puntúa las calidades de las variantes y filtrar aquellas que no pasen los filtros. Se comprueba que una variante sea efectivamente verdadera. Para ello, se da un archivo de referencia de variantes y se estima si la variante es un artefacto de la secuenciación o una variante de verdad. El resultado es VQSLOD, que se añade al campo de información. Esto en general se realiza para SNPs e INDELs por separado debido a que las bases de datos de variantes son diferentes. 

El paso siguiente es aplicar los filtros VQSR. En la columna FILTER anota si la variante pasa filtros o no, pero no descarta aquellos que no pasen los filtros; si se quiere eso se debe especificar.
\begin{lstlisting}[language=bash]
tabix -p vcf Annotations/dbsnp_138.hg19_chr17.vcf.gz

gatk VariantRecalibrator -R REFERENCE/hg19_chr17.fa -V out/normal_refined_out.vcf
--resource:dbsnp,known=true,training=true,truth=true,prior=15.0
Annotations/dbsnp_138.hg19_chr17.vcf.gz -an QD -an ReadPosRankSum -an FS -an SOR -mode BOTH -O out/output_normal_refined.recal --tranches-file out/output_normal_refined.tranches

gatk ApplyVQSR -R REFERENCE/hg19_chr17.fa -V out/normal_refined_out.vcf -O out/output_normal_refined.recalibrated --truth-sensitivity-filter-level 99.0 --tranches-file out/output_normal_refined.tranches --recal-file out/output_normal_refined.recal -mode BOTH
\end{lstlisting}

El último paso es el filtrado de las variantes. 
\begin{lstlisting}[language=bash]
awk -F '\t' '{if ($0 ~ /#/ || $7 == "PASS") print}' out/output_normal_refined.recalibrated > out/output_normal_refined.onlypass

#Contar el número de líneas resultantes
grep "^chr17" out/output_normal_refined.onlypass | wc -l
\end{lstlisting}

%21/11 - Dido
\chapter{Introducción a las variantes somáticas}
\section{Control de calidad y refinamiento de alineamientos}
\subsection{Control de calidad}
Los controles de calidad se suelen hacer en varios puntos de todo el proceso. Hay varios puntos clave, y después de cada uno de ellos se realiza el control de calidad. El primero y más importante parte de los archivos FastQ de secuenciación, ya que si éstos están mal, el resto del análisis carece de sentido.

Se utiliza el programa FastQC que da un informe en HTML con varias estadísticas de los archivos. Hay otros programas que se pueden utilizar, como samtools, que indica si hay secuencias duplicadas y otras estadísticas. MultiQC combina varias herramientas para sacar un informe completo.

En los controles de calidad, se mira si hay un número de lecturas dentro del rango esperado, si las bases tienen una buena calidad mediante el Phred score, y si hay contaminación en las muestras. Una secuencia que aparezca repetidamente puede ser muy repetitiva en el genoma que se secuencie o una contaminación.

FastQC permite visualizar la calidad por base por secuencia, el contenido GC y secuencias sobrerrepresentadas entre otras métricas, pero esas son las que habitualmente están mal. En cuanto a la evaluación de la calidad por base, se ve la distribución de los scores en las lecturas. En secuenciación de Illumina, es habitual que al final de las lecturas la calidad decaiga un poco, pero debería seguir en un rango elevado. Si la calidad decae mucho, se trata de un error en la secuenciación. En los scores por secuencia, se espera que la mayoría de las secuencias tengan una puntuación muy alta. Si hay varias secuencias con una calidad baja, eso indica que algo está mal, pero es difícil indicar la causa (problema del secuenciador, purificación de las muestras, contaminación, librería, etc). La distribución esperada del contenido en GC debería seguir una distribución normal, aunque hay veces que se puede desviar un poco.

\begin{lstlisting}[language=bash]
conda install bioconda::fastqc
fastqc -o out/ Raw_data/*.fastq
\end{lstlisting}

El resultado es un fichero html y zip por cada FastQ. En cuanto a Normal R1, el contenido GC muestra un mensaje de error. Hay un pico muy grande alrededor del 60\%, cuando en humanos debería rondar el 50\%. Además, hay varias lecturas con un contenido GC bajo, sobre el 35\%. Como estos resultados son malos, se puede valorar descartarlos y repetir la secuenciación, pero en laboratorios pequeños puede ser un problema. Además, tenemos una secuencia sobrerrepresentada de N, que no nos preocupa porque se va a descargar. Todas las demás métricas han salido bien. En Normal R2, las secuencias sobrerrepresentadas dan error, pero sigue siendo una secuencia de todo N, por lo que se le puede dar poca importancia. El contenido GC también da error. En cuanto a las muestras de tumor, son muy similares: tienen un contenido GC más bajo y tiene una secuencia de N sobrerrepresentada. 
Hay que tener en cuenta que estamos trabajando solo con el cromosoma 17, pero que la distribución esperada está calculada sobre todo el genoma. Por tanto, si ese cromosoma tiene muchas secuencias repetitivas en AT, ya de por sí habrá un sesgo en la comparación con la distribución esperada. 

\subsection{Alineamiento}
En cuanto al alineamiento, se utiliza BWA. Se puede utilizar la siguiente chuleta para la indexación en base al fichero que se tenga:
\begin{lstlisting}[language=bash]
#Fasta
bwa index reference.fasta
samtools dict reference.fasta -o reference.dict
samtools faidx reference.fasta

#BAM
samtools index bam.file

#VCF
tabix -p vcf vcf.file
\end{lstlisting}

El siguiente paso es la alineación.
\begin{lstlisting}[language=bash]
bwa mem -R '@RG\tID:OVCA\tSM:Normal' REFRENCE/hg19_chr17.fa Raw_data/WEx_Normal_R1.fastq Raw_data/WEx_Normal_R2.fastq > alignment/normal.sam
bwa mem -R '@RG\tID:OVCA\tSM:Tumour' REFRENCE/hg19_chr17.fa Raw_data/WEx_Tumour_R1.fastq Raw_data/WEx_Tumour_R2.fastq > alignment/tumour.sam
\end{lstlisting}

Se puede utilizar \texttt{samtools flagstat normal.sam} para ver unas estadísticas como alineamientos mapeados, primarios, secundarios, duplicados, etc. 

\subsection{Refinamiento del alineamiento}
En este paso, queremos convertir los SAM en BAM para que los ficheros estén comprimidos y binarizados. Además, BWA a veces omite alguna información en los ficheros SAM que queremos rellenar.
\begin{lstlisting}[language=bash]
samtools fixmate -O bam alignment/normal.sam alignment/normal_fixmate.bam
samtools fixmate -O bam alignment/tumour.sam alignment/tumour_fixmate.bam
\end{lstlisting}

Los duplicados vienen de la amplificación por PCR durante la preparación de la librería. Un error al principio de la PCR se propaga. Los duplicados en secuenciación híbrida no aportan nada al análisis posterior, pudiendo dar lugar a falsos positivos y redundancia. Por ello, lo mejor es descartar estas duplicaciones. Además, para la llamada de variantes es necesario que el alineamiento esté ordenado por posición genómica, y aprovechamos para indexar el BAM.
\begin{lstlisting}[language=bash]
samtools sort -O bam -o alignment/normal_sorted.bam alignment/normal_fixmate.bam
samtools sort -O bam -o alignment/tumour_sorted.bam alignment/tumour_fixmate.bam

samtools rmdup -S alignment/normal_sorted.bam alignment/normal_refined.bam
samtools rmdup -S alignment/tumour_sorted.bam alignment/tumour_refined.bam

samtools index alignment/normal_refined.bam
samtools index alignment/tumour_refined.bam
\end{lstlisting}

Con esto hemos creado los bams que utilizamos en la parte anterior de las variantes germinales. Se pueden eliminar los ficheros intermedios de ficheros sam, fixmate y sorted, dejando los bam refinados.

\section{Recalibración de la calidad de base}
Cada posición de la secuencia tiene su calidad de base. Las diferentes tecnologías NGS tienen sus sesgos dependiendo del contexto, por lo que es importante hacer la recalibración para corregir empíricamente esos sesgos. La recalibración de bases no es lo mismo que la recalibración de variantes (eso viene después con los VCF).

La recalibración de bases consta de dos pasos. Primero, BaseRecalibrator parsea las lecturas y crea una tabla en la que se asigna a cada lectura el ciclo durante el que se leyó la base, su puntuación y la puntuación de las bases que van antes y después. EL modelo computa cuántas veces hay un cambio en la referencia según dbSNP, excluyendo los loci de alta variabilidad. Después, la herramienta ApplyBQSR parsea las lecturas y, utilizando la matriz de antes, recalibra las puntuaciones. Así, las calidades son más similares a las puntuaciones empíricas, pudiendo utilizarse para el variant calling.

\section{Llamada de variantes somáticas}
Las variantes son cambios permanentes en el ADN de un organismo que pueden deberse a errores durante la replicación, durante la recombinación durante la formación de gametos y por factores externos como radiación, viruses, transposones, rayos ultravioleta, etc. El genoma entre humanos es idéntico en un 99,9\%. El 0,1\% restante es la fuente de variabilidad, permitiendo los mecanismos de evolución, las diferencias fenotípicas entre individuos y en la respuesta a enfermedades y fármacos. 

\begin{figure}[htbp]
\centering
\includegraphics[width = \textwidth]{figs/variants-size.png}
\end{figure}

En función de la posición de la secuencia, puede haber variantes intergénicas (entre genes) en secuencias reguladoras o upstream o downstream de algún gen en concreto. Dentro de genes, puede haber variantes en las regiones UTR, en los exones, en los intrones o en regiones de splicing. 

Para realizar la llamada de variantes, se utiliza MuTect2. Es similar a GATK, pero permitiendo mayor variabilidad de frecuencia alélica (no solo 0, 0,5 y 1 como GATK), además de evitar las variantes germinales. Hay distintos modos: tumor con tejido normal (este es el óptimo), solo tumor y mitocondrial. Se crea un panel de normales para poder inferir qué mutaciones son exclusivamente somáticas. 
\begin{lstlisting}[language=bash]
#Modo Tumor-only 
gatk Mutect2 -R REFERENCE/hg19_chr17.fa -I alignment/tumour_refined.bam -O tumour_only_somatic.vcf

#Modo tumor with matched normal
gatk Mutect2 -R REFERENCE/hg19_chr17.fa -I alignmetn/tumour_refined.bam -I alignment/normal_refined.bam -normal Normal -O out/tumour_matched_somatic.vcf
\end{lstlisting}
 
Después del filtrado, obtenemos las siguientes variantes somáticas:
\begin{lstlisting}[language=bash]
grep "^chr17" out/tumour_matched_somatic.vcf | wc -l #193
grep "^chr17" out/tumour_only_somatic.vcf | wc -l #461
\end{lstlisting}
Cuando solo se mide el tumor, el número es casi el doble. Al dar el normal, el algoritmo sabe qué variantes son germinales por estar ya presentes en el tejido y los descarta. Al medir solo el tumor, algunas variantes germinales se excluyen, pero otras se cuelan, por lo que el número de variantes es mayor. 

Las frecuencias alélicas vienen en la columna FORMAT bajo AF. Estas frecuencias van del 0 al 1 en todo el rango. En las variantes germinales, esta información aparece en INFO y es de 0, 0,5 o 1. 

El fichero de solo tumor, tiene más variantes porque no puede excluir todas las variantes germinales. No obstante, el número de mutaciones sigue siendo mayor que la suma de tumor matched y las mutaciones germinales calculadas en la parte anterior. Esto se debe a que, al calcular las variantes germinales, el algoritmo solo se queda con aquellas que tengan una frecuencia alélica de 0,5 o de 1, no con todo el rango como el que detecta tumor only. Por ejemplo, por contaminación, tumor only puede detectar una variante germinal que no esté al 0,5 o 1 y que, por tanto, esté excluida del análisis de variantes germinales. El fichero de tumor only siempre va a  tener más variantes.

%25/11 - Dido
\chapter{Anotación de variantes}
La anotación de variantes consiste en enriquecer los archivos VCF con información adicional útil para su priorización. Esta información proviene tanto de bases de datos como de cálculos basados en la posición genómica de las variantes.

Dado el gran número de variantes en un archivo VCF, es impráctico evaluarlas manualmente. Como no todas tienen impacto relevante en el fenotipo de estudio, se aplican filtros para reducir el conjunto a las variantes potencialmente relevantes. Esto incluye descartar variantes comunes, benignas o de significancia incierta (VUS, variants of unknown significance).

En una anotación estándar, se incluye información sobre:
\begin{itemize}
\item Tipo de variante (SNV, indel, etc.)
\item Localización genómica
\item Consecuencias en la secuencia
\item Predicción del impacto funcional
\item Frecuencia poblacional
\item Asociación con patologías conocidas
\end{itemize}

\section{Nomenclatura de variantes}
Las variantes se identifican por el cromosoma, la coordenada genómica, el alelo de referencia y el alelo alternativo. Existen dos genomas de referencia principales: hg19 y hg38. El más reciente, hg38, es el recomendado para investigaciones actuales. Para convertir coordenadas entre estas referencias, se puede utilizar la herramienta Liftover disponible en UCSC.

El impacto en el ADN codificante y en la proteína se describe según la nomenclatura HGVS:
\begin{itemize}
\item g. genomic reference sequence
\item c. coding DNA reference sequence
\item m. mitochondrial DNA reference sequence
\item n. non-coding DNA reference sequence
\item r. RNA reference sequence (transcript)
\item p. protein reference sequence
\end{itemize}

Para asegurar la uniformidad, se recomienda seguir las \href{https://hgvs-nomenclature.org/stable/recommendations/general/}{guías estandarizadas para la nomenclatura HGVS}.

Las mutaciones pueden tener efectos diferentes en los transcritos de un mismo gen debido al splicing alternativo. Algunas enfermedades surgen por mutaciones que afectan transcritos minoritarios mientras que el principal permanece intacto. Para gestionar esta complejidad, se usan identificadores únicos, como los de Ensembl:
\begin{itemize}
\item Gene Identifiers: ENSG 
\item Transcript Identifiers: ENST
\end{itemize}

Además, las bases de datos como \href{https://www.genecards.org/}{GeneCards} o \href{https://www.genenames.org/}{GeneNames} ayudan a identificar el nombre oficial del gen, ya que un mismo gen puede tener varios nombres comunes.

Ensembl proporciona información detallada sobre cada gen, sus transcritos y biotipos, junto con equivalencias entre diferentes bases de datos. Por ejemplo:
\begin{itemize}
\item RefSeq incluye principalmente el transcrito principal
\item MANE es una anotación colaborativa que indica el transcrito principal
\end{itemize}

Para proteínas, la base de datos por referencia es \href{https://www.uniprot.org/}{Uniprot},  que incluye información sobre función, localización subcelular, relación con enfermedades y efectos de mutaciones.
Además, APPRIS (proyecto de Gencode) identifica transcritos principales y sus isoformas, categorizados como MANE, canonical o APPRIS principal.

\section{Consecuencias en la secuencia}
El impacto de una variante depende de su ubicación dentro del gen o región reguladora. Las consecuencias están codificadas y documentadas en \href{https://www.ensembl.org/info/genome/variation/prediction/predicted_data.html}{Ensembl}. 

\begin{figure}[htbp]
\centering
\includegraphics[width = \textwidth]{figs/consequences.png}
\end{figure}

\begin{itemize}
\item Mutaciones puntuales:
\begin{itemize}
\item Missense: cambian un aminoácido. Pueden ser conservativas o no conservativas en función de si mantienen la polaridad.
\item Nonsense: introducen un codón de parada
\item Silentes: no afectan al aminoácido
\end{itemize}
\item Mutaciones inframe: no alteran el marco de lectura
\item Mutaciones frameshift: cambian el marco de lectura, afectando significativamente la proteína.
\end{itemize}

Las consecuencias se clasifican según su impacto:
\begin{itemize}
\item Alto: frameshift, nonsense
\item Moderado: missense
\item Bajo: silentes
\item Modificador: sin efecto directo conocido
\end{itemize}

Esta clasificación facilita priorizar variantes según su potencial efecto dañino.

\section{Predicción del impacto funcional}
Existen numerosos algoritmos para predecir el impacto de variantes en la función y estructura de proteínas. Estos se agrupan en:
\begin{itemize}
\item Predictores para variantes missense
\item Predictores para variantes que afectan el splicing
\item Predictores basados en la conservación evolutiva
\end{itemize}

\subsection{Predictores para variantes missense}
El hecho de que haya tantos predictores hace que haya varias escalas para predecir el impacto de un cambio. Cada software tiene su escala y sus criterios para ver si una variante es benigna o no. Por ello, se realiza la anotación con varios y se busca el consenso entre ellos. Algunos predictores son:
\begin{itemize}
\item \textbf{Sift:} evalúa el efecto funcional basándose en homología y propiedades de aminoácidos. Score: 0-1 (más bajo = mayor impacto); < 0,05 deleterious
\item \textbf{PolyPhen:} basado en estructura y función proteica. Score: 0-1 (más alto = mayor impacto); benigno < $\sim$ 0,435 < dañino
\item \textbf{Revel:}  integra 13 herramientas, incluidas SIFT y PolyPhen, para una predicción más robusta.
\item \textbf{ClinPred:} utiliza machine learning para identificar variantes relevantes en enfermedades, incorporando frecuencias alélicas de gnomAD.
\item \textbf{AlphaMissense:} se trata de una adaptación de AlphaFold ajustada a bases de datos de frecuencias de poblaciones de variantes humanas y primates para predecir la patogenicidad de variantes sin sentido combinando el contexto estructural y la conservación evolutiva. Genera predicciones para todas las posibles sustituciones de aminoácidos en humanos y clasificación del 89\% de las variantes sin sentido como probablemente benignas o patógenas.
\end{itemize}

Como no hay consenso, a la hora de realizar la prioridad hay que tener en cuenta su score. Estos predictores tienen una precisión y especificidad moderada, por lo que se deben utilizar en conjunto y ver el consenso.

\subsection{Predictores para variantes de splicing}
En cuanto a los predictores de splicing, uno muy notable es \textbf{SpliceAI}. Se trata de una herramienta basada en deep-learning que identifica las variantes de splicing y predice el efecto. La puntuación delta de una variante oscila entre 0 y 1 y puede interpretarse como la probabilidad de que la variante altere el splicing. En el artículo se ofrece una caracterización detallada de los valores de corte de 0,2 (alta recuperación), 0,5 (recomendado) y 0,8 (alta precisión).

\section{Frecuencias poblacionales}
Las frecuencias poblacionales ayudan a filtrar variantes comunes. Si una variante aparece en más del 1\% de la población, se considera un polimorfismo y generalmente no se asocia a enfermedades graves. Estas variantes pueden causar diferencias desde el color de pelo a la respuesta frente a fármacos. Las bases de datos clave son:
\begin{itemize}
\item gnomAD: es la más completa, con datos de 807,162 individuos divididos entre exomas y genomas completos, y estratificados por población y sexo. También incluye una versión con individuos sanos. Las variantes se pueden buscar mediante las coordenadas de hg38.
\item Proyecto 1000 Genomas
\item dbSNP
\item CIBERER: servidor de variantes en la población española con 2.100 exomas. Permite obtener información mucho más concreta de la población
\end{itemize}

\section{Asociación con enfermedades}
Muchas bases de datos incluyen información aportada por mutaciones in silico o in vivo que guardan relación con el desarrollo de una enfermedad. Algunas de ellas son de pago (como HGMD), pero hay otras buenas de libre acceso.
\begin{itemize}
\item \textbf{ClinVar}: recoge una serie de mutaciones (principalmente pequeñas SNV e indels, aunque hay algunas CNV), dando información sobre la clasificación clínica de la variante (benigna, probablemente benigna, patogénica, probablemente patogénica, respuesta a fármacos, factor protector para una condición, factor de riesgo, etc.), la enfermedad que causa y estatus de revisión. La mayor parte de las variantes son de significado incierto.
\item \textbf{OMIM}: es una base de datos de genes humanos y trastornos y rasgos genéticos con un enfoque particular en la relación gen-fenotipo. La búsqueda se realiza por enfermedad para ver todas las variantes asociadas a la misma.
\item \textbf{COSMIC}: es una base de datos en línea que recoge las mutaciones somáticas descubiertas en el cáncer humano. Recopila datos de publicaciones científicas y de estudios experimentales a gran escala.
\item \textbf{DisGeNET}
\end{itemize}

\section{Herramientas de anotación}
Las herramientas de anotación más comunes son:
\begin{itemize}
\item \textbf{Variant Effect Predictor (VEP):} se trata de una herramienta de Ensembl, siendo la más utilizada y completa. Proporciona anotaciones para los distintos tipos de alteraciones en las diferentes localizaciones genómicas. Tiene una versión en línea de comandos, pero también una \href{https://www.ensembl.org/info/docs/tools/vep/index.html}{versión web}. Permite incluir los identificadores de gen, versión del transcrito, UniProt y HGVS, además de muchos otros predictores e información. 
\item \textbf{ANNOVAR}
\item \textbf{Variant Effect}
\item \textbf{Predictor}
\item \textbf{VarAFT}
\item \textbf{SnpEff}
\end{itemize}

Utilizando VEP con los datos generados en la llamada de variantes somáticas con tejido normal pareado, los resultados son los siguientes (muchas columnas de la tabla están ocultas, como la puntuación de los distintos predictores):
\begin{figure}[htbp]
\centering
\includegraphics[width = \textwidth]{figs/vep-result.png}
\end{figure}

Para priorizar, se tendría en cuenta el score de AlphaMissense, SpliceAI, ClinPred, Revel, Sift y PolyPhen, buscando un consenso entre todos. Además, se puede buscar el identificador en ClinVar por si se hubiese recogido una asociación.

%26/11 - Dido
La priorización es un paso posterior, y en ocasiones nos podemos quedar en la descripción de lo que hay en las muestras sin profundizar cuál puede ser la mutación clave. En un análisis genómico descriptivo, nos quedaríamos en el VCF filtrado. 

Aunque nosotros hayamos utilizado el servidor web, de forma profesional se utiliza por línea de comando. En cualquier caso, podemos descargarnos el VCF ya anotado. Este fichero incluye la leyenda de lo que significa la información en el campo INFO. El encabezado muestra también información sobre las distintas herramientas utilizadas con los distintos límites de los scores y el comando para la línea de comando, incluso si se ha generado por el servidor web. EL VCF ya tenía información previa en el campo INFO, pero los detalles nuevos comienzan con CSQ y está separada por barras verticales. Como esto es algo difícil de leer bien, cada uno tiene su código para separar esa información en columnas más legibles; ningún bioinformático trabaja desde la web.

En resumen: En el archivo de input que se carga, había 193 variantes. Los genes y transcritos afectados por las variantes son 230 y 1299 respectivamente. Hay 22 variantes que caen en regiones reguladoras. La consecuencia más abundante son variantes intrónicas. Dentro de secuencias codificantes, la más común es missense. Las variantes que caen en una región codificante de un gen son 0. Los identificadores HGVS indican la notación estándar de las variantes. Las herramientas de predicción no siempre tienen mucho consenso. En caso de no encontrarlo para una variante, se debe valorar. Podríamos tener en cuenta primero REVEL, AlphaMissense y ClinPred. Sift y PolyPhen por sí mismos no son suficientes. Para ver si hay polimorfismos en nuestros datos, se mira la columna Existing Variant, indicando si esa variante está ya descrita en alguna base de datos, o la frecuencia alélica en gnomADe AF (exomas; valores mayores a 0,01).

Para datasets grandes y análisis de muchas muestras, no se utiliza el servidor web, si no el programa por línea de comandos. La instalación puede ser algo compleja al tener que descargar ficheros caché y una base de datos local para que la consulta de datos sea más rápida y no dependa de internet. También hay que descargar algunos plugins para facilitar y mejorar la anotación. Finalmente, se pueden realizar anotaciones customizadas de ficheros VCF, BED, GTF o BigWig.

\begin{figure}[htbp]
\centering
\includegraphics[width = \textwidth]{figs/ensembl-vep-commandline.png}
\caption{Ejemplo de un comando con pocas anotaciones en Ensembl. Está la ruta a la herramienta, el input y output en formato vcf comprimido o no. La caché son una serie de archivos que permite hacer las consultas en modo offline para que vaya más rápido. Los plugins están en el repositorio de GitHub y se pueden descargar para utilizarlos. Después hay una serie de opciones básicas: sobreescribir archivos en caso de existir previamente, paralelizar, número de variantes que se procesan a la vez, genoma de referencia, formato, coficiación MANE del tránscrito, que utilice Sift, PolyPhen, etc. Después se especifican los plugins y las bases de datos customizadas. En este caso, se accede a la base de datos gnomAD4g en formato VCF y se pide la anotación exacta, obteniendo la frecuencia alélica. Como se quiere utilizar la versión 4.1, se debe realizar una consulta customizada. En otros casos hay plugins existentes, como en caso de pLI, en el que no es necesario especificar qué datos debe buscar. De igual forma se especifica el plugin de AlphaMissense y se crea una consulta customizada para ClinVar.}
\end{figure}

\chapter{Priorización de variantes}
Una vez anotados los VCF, nos podemos limitar a la descripción de las variantes de un paciente o una cohorte, o realizar un análisis más detallado filtrando variantes y priorizando. Este sería el último paso, y no en todos los casos se realiza. 

\section{Visualización en IGV}
Al ver una variante interesante, se recomienda visualizarla en IGV para verificar la variante. Se pueden encontrar falsos positivos (artefactos) debido a errores de secuenciación o análisis, al igual que falsos negativos por regiones de baja cobertura o variantes de baja frecuencia. 

Cuando una variante está en los extremos de las lecturas, especialmente si está al final, puede deberse a errores en la secuenciación. Si las demás lecturas tienen esa base en el centro y no cuentan con la variante, podemos tratarlo como artefacto y falso positivo. También puede darse que una variante solo se dé en lecturas que vayan en el mismo sentido. En esos casos se trata de un sesgo y causa un falso positivo. Finalmente, si las lecturas se alinean en regiones que no están muy bien representadas en la referencia y hay muchas sustituciones, probablemente se deba a un alineamiento parálogo (las lecturas tendrían que haber alineado en otro sitio muy parecido). Estos errores no son muy frecuentes por los pasos de filtrado y refinamiento, pero pueden ocurrir. 

\section{Priorización}
La priorización se realiza utilizando un conjunto de evidencias de relevancia basadas en las anotaciones. Las anotaciones utilizadas en la priorización varían con la patología o condición en estudio. Los criterios pueden variar en función del objetivo (variantes raras, variantes comunes en la población, etc.).

Algunas evidencias para variantes clínicamente relevantes son:
\begin{itemize}
\item \textbf{Significancia clínica conocida:} ClinVar muestra si la variante es patogénica, un factor de riesgo, respuesta a fármacos.
\item \textbf{Impacto funcional:} puede ser algo o moderado en función de la consecuencia en la secuencia. Hay varios predictores del impacto funcional.
\item \textbf{Relevancia en patología:} se buscan genes implicados en procesos que están involucrados en enfermedades, o variantes frecuentes/alteraciones recurrentes de un gen o de la enfermedad
\end{itemize}

Algunas evidencias para variantes no relevantes clínicamente son variantes en tránscritos no relevantes o muy poco soportados, polimorfismos y variantes con una frecuencia poblacional mayor al 1\% salvo que esté asociado con predisposición, prognosis, respuesta a fármacos, etc.

Para el diagnóstico existen los \textbf{criterios ACMG}. Lo creó el American College of Medical Genetics and Genomics, y son normas y directrices para la interpretación de variantes de secuencia.
Se desarrollaron principalmente como un recurso educativo para los genetistas de laboratorio clínico para ayudarles a proporcionar servicios de laboratorio clínico de calidad. Las directrices del ACMG incluyen 28 criterios. Durante la interpretación de variantes, éstas se clasifican en cinco niveles: Patogénicas (P), Probablemente patogénicas (LP), Significación incierta (VUS), Probablemente benignas (LB) y Benignas (B), en función de los criterios aplicados.

\begin{figure}[htbp]
\centering
\includegraphics[width = 0.8\textwidth]{figs/acmg-guideline-table.jpg}
\end{figure}

Con la tabla anterior, se realiza un balance entre las opciones que apliquen. Cuando hay criterios de benignos y malignos, normalmente se clasifica como variante de significado incierto. 

%28/11 - Dido
\chapter{Caracterización de cohortes}
Cuando se secuencia un conjunto de muestras, se puede realizar para el diagnóstico o para hacer un análisis holístico y global (exploración de los datos). En este contexto, se pueden realizar varios pasos para ver si hay puntos en común entre los datos o si se describe alguna tendencia.

\section{Carga mutacional tumoral (TMB)}
La TMB es una medida cuantitativa del \textbf{número total de mutaciones por megabase} de ADN en el genoma de un tumor. Sirve como indicador de la \textbf{inestabilidad genómica} y a menudo se asocia con el potencial del tumor para producir neoantígenos que desencadenen una respuesta inmunitaria. El TMB se utiliza habitualmente como \textbf{biomarcador} para predecir la eficacia de la inmunoterapia, como los inhibidores de puntos de control inmunitarios, ya que un mayor TMB suele correlacionarse con mejores resultados terapéuticos. La TMB se suele calcular con mutaciones puntuales, y se pueden excluir las mutaciones sinónimas.

Hay muchas herramientas para calcular el TMB, como puede ser \href{https://github.com/anitalu724/MutScape}{MutScape}. Contiene varios sets de herramientas, como la detección de genes significativamente mutados, anotación de mutaciones asociados a cáncer o las estadísticas de la carga mutacional. Estas herramientas utilizan los ficheros en formato VCF o MAF (para esto último, se puede convertir un VCF con la herramienta \href{https://github.com/mskcc/vcf2maf}{VCF2MAF}. También calcula las mutational signatures, inestabilidad cromosómica, anotación de mutaciones accionables por fármacos, etc. 

A continuación se muestra un gráfico o plot de una cohorte de muestras tumorales divididas por el tumor primario. Se representa la carga mutacional. Testículo tiene una carga mutacional muy baja (1 mutación por megabase), pero el útero tiene mucha. Esto puede servir para poder identificar el tumor primario hipermutado. Es un buen primer enfoque para caracterizar la cohorte.

%FIXME añadir plot

\section{Oncoplot}
Un oncoplot es una representación visual utilizada habitualmente en genómica del cáncer para resumir y mostrar el panorama mutacional de una cohorte de muestras tumorales. Suele mostrar mutaciones, CNV y otras alteraciones en genes clave relacionados con el cáncer de varios pacientes, utilizando un formato similar a un mapa de calor. Cada fila representa un gen, cada columna representa una muestra y los colores o símbolos indican tipos específicos de alteraciones.

%FIXME añadir plot

Estos oncoplots se generan y trabajan con la herramienta \href{https://bioconductor.org/packages/devel/bioc/vignettes/maftools/inst/doc/maftools.html}{Maftools}. 

\href{https://www.cbioportal.org/}{Cbioportal} es una plataforma donde se han subido los genomas de muchos estudios de cáncer para poder visualizarlos a través de oncoplots. Se pueden seleccionar unos genes para buscar en muestras sacadas del atlas de cáncer pangenómico (TCGA). 

\section{Mutational signatures}
Las firmas mutacionales son \textbf{patrones únicos de mutaciones} en el ADN que reflejan los procesos subyacentes que causan alteraciones genéticas en un tumor. Estos procesos pueden incluir la exposición a factores ambientales (por ejemplo, radiación UV, fumar), deficiencias en la reparación del ADN o actividades enzimáticas. El análisis de las firmas mutacionales ayuda a identificar la etiología de las mutaciones, descubrir mecanismos de desarrollo tumoral y orientar las decisiones terapéuticas vinculando firmas específicas a posibles vulnerabilidades u opciones de tratamiento.

\href{https://cancer.sanger.ac.uk/signatures/}{COSMIC} ha generado las firmas mutacionales utilizando análisis de gran escala. Las colecciones de formas están clasificadas en función de las mutaciones que se estén analizando (mutaciones puntuales, dobletes, indels, etc). La mayoría de etiologías son desconocidas, pero sí se caracteriza una huella. La herramienta oficial es \href{https://github.com/AlexandrovLab/SigProfilerExtractor}{SigProfilerExtractor} para generar el análisis.

%FIXME añadir plot

Un patrón se puede descomponer en distintas firmas. Se puede inferir el mecanismo de acción que causó el tumor mediante el ProfilerExtractor y \href{https://github.com/AlexandrovLab/SigProfilerMatrixGenerator}{ProfilerMatrixGenerator}. 

Una vez calculadas las firmas de cada una de las muestras (aunque se computen en conjunto), se realiza la descomposición para que la composición de firmas de cada una de las muestras. Así, se puede ver si un mutor tiene la huella de haber sido tratado con quimioterapia con platino, si está asociado al tabaco, etc. 

%FIXME añadir plot de heatmap

\section{Otros aspectos relevantes}
La \textbf{inestabilidad cromosómica (CIN)} se refiere al aumento de la tasa de cambios cromosómicos, incluyendo ganancias, pérdidas y reordenamientos de cromosomas dentro de una célula. La CIN contribuye a la heterogeneidad tumoral, la progresión y la resistencia a las terapias al crear diversidad genética y promover la adaptación al estrés o a los tratamientos.

Los \textbf{neoantígenos} son péptidos de nivel que se presentan en la superficie de las células tumorales como resultado de mutaciones suáticas. Estos antígenos únicos son reconocidos por el sistema inmunitario y pueden desencadenar una respuesta inmunitaria antitumoral. Los neoantígenos son un punto clave en la inmunoterapia del cáncer, sobre todo para desarrollar vacunas personalizadas contra el cáncer e inhibidores de puntos de control inmunitario.

La \textbf{inestabilidad de microsatélites (MSI)} es una enfermedad caracterizada por la acumulación de mutaciones en secuencias repetitivas de ADN denominadas microsatélites debido a defectos en el sistema de reparación de errores de emparejamiento del ADN (MMR). La MSI suele asociarse a ciertos tipos de cáncer, como el colorrectal, el endometrial y el gástrico, y sirve como biomarcador de la respuesta a la inmunoterapia, en particular a los inhibidores de puntos de control inmunitarios.

\chapter{Copy Number Variants (CNV)}
La variación del número de copias (CNV) es un fenómeno en el que se repiten secciones del genoma
y el número de repeticiones en el genoma varía de un individuo a otro.
La variación del número de copias es un tipo de variación estructural: en concreto, es un tipo de duplicación o que afecta a un número considerable de pares de bases (genes enteros).
Aproximadamente dos tercios de todo el genoma humano pueden estar compuestos por repeticiones y el 4,8-9,5\% del genoma humano pueden clasificarse como variaciones del número de copias.
Cada vez hay más pruebas de que las CNV desempeñan un papel importante en las enfermedades humanas.

Las variaciones del número de copias se estudiaron originalmente mediante \textbf{técnicas citogenéticas}, que son técnicas que permiten observar la estructura física del cromosoma.
Una de estas técnicas es la \textbf{hibridación fluorescente in situ (FISH)}, que consiste en insertar sondas fluorescentes que requieren un alto grado de complementariedad en el genoma para unirse. Al microscopio se podían ver bandas y patrones en los cromosomas para ver si habían surgido deleciones, inserciones o translocaciones. Pequeñas CNVs no se veían.
La \textbf{hibridación genómica comparada (CGH)} también se utilizaba habitualmente para detectar variaciones en el número de copias mediante la visualización de fluoróforos y la posterior comparación de la longitud de los cromosomas, con mayor resolución. Compara la longitud esperada con la observada por fluorescencia. Con secuenciación, la detección de CNVs mejoró considerablemente. 

\section{Llamada de variantes de número de copias}
Se han desarrollado muchos algoritmos para realizar la llamada de variantes en el número de copias.
La base de estos algoritmos varía en función de la tecnología NGS utilizada para secuenciar los datos.
La secuenciación del genoma completo (WGS) a menudo utiliza estrategias de mapeo de profundidad de lectura o de extremo pareado, mientras que la secuenciación del exoma (WES) se basa en la normalización de la cobertura y el análisis de regiones específicas (si una región en lugar de tener una cobertura de 200x tiene una cobertura de 600x, se puede inferir que está triplicada; pero depende mucho de la preparación de la librería). 
Entre los principales retos a los que se enfrenta la llamada de VNC se encuentran la distinción entre variantes verdaderas y artefactos técnicos, el manejo de regiones de baja calidad y la detección fiable de puntos de rotura (breakpoints).
Una interpretación adecuada requiere algoritmos robustos y la validación con técnicas complementarias de laboratorio (como MLPA), ya que los algoritmos, sobre todo en el caso de exomas, no funcionan del todo bien.

Algunos programas para la llamada de variantes en el número de copias en \textbf{WGS} son:
\begin{itemize}
\item \href{https://github.com/Illumina/manta}{Manta}: Diseñado para detectar variantes estructurales (VS) y CNV a partir de datos WGS. Utiliza un enfoque basado en gráficos, ofreciendo una alta sensibilidad a duplicaciones, deleciones y reordenamientos complejos. Funciona bien con datos de alta cobertura y es particularmente adecuado para CNVs grandes.
\item \href{https://github.com/dellytools/delly}{Delly}: Se especializa en la detección de variantes estructurales, incluidas duplicaciones, deleciones, inserciones y translocaciones. Aprovecha las lecturas divididas y discordantes, lo que proporciona una gran precisión en conjuntos de datos WGS.
\item \href{https://genomebiology.biomedcentral.com/articles/10.1186/gb-2014-15-6-r84}{Lumpy}: Un llamador de variantes estructurales que combina lecturas divididas, lecturas discordantes y profundidad de cobertura. Eficaz para datos WGS, capaz de detectar CNV de varios tamaños y tipos.
\end{itemize}

Para \textbf{WGS}, se utilizan:
\begin{itemize}
\item \href{https://cnvkit.readthedocs.io/en/stable/}{CNVkit}: Diseñado específicamente para datos WES, pero también compatible con paneles específicos. Normaliza la profundidad de lectura a la vez que tiene en cuenta las características únicas de los diseños de captura de exomas. Alta precisión para CNV pequeñas y medianas en regiones codificantes.
\item \href{https://github.com/vplagnol/ExomeDepth}{ExomeDepth}: Paquete R adaptado para WES, que compara la profundidad de lectura entre muestras con un panel de controles. Muy adecuado para aplicaciones clínicas, ya que ofrece una alta sensibilidad y especificidad en las regiones capturadas.
\item \href{https://www.nature.com/articles/s41588-023-01449-0}{GATK gCNV}: Parte de la suite GATK, desarrollada para la detección de CNV en WES. Utiliza un modelo bayesiano para integrar datos de múltiples muestras y mejorar la precisión.
\end{itemize}

Una base de datos que recoge la frecuencia poblacional de CNVs es \href{https://dgv.tcag.ca/dgv/app/about}{DGV o Database of Genomic Variants}. Resumen de la variación estructural en el genoma humano (alteraciones genómicas que afectan a segmentos de ADN de más de 50 pb). El contenido de la base de datos sólo representa la variación estructural identificada en muestras de control sanas, por lo que proporciona un catálogo útil de datos de control para los estudios que pretenden correlacionar la variación genómica con los datos fenotípicos. La base de datos se actualiza continuamente con nuevos datos procedentes de estudios de investigación revisados por expertos.

\chapter{Snakemake y pipeline management}
Todos los pasos vistos anteriormente se deben automatizar, ya que es inviable realizar los pasos individuales para una gran cantidad de muestras.

Snakemake es un potente y flexible sistema de gestión de flujos de trabajo diseñado para crear pipelines de análisis de datos reproducibles y escalables utilizando un lenguaje basado en Python.
Utiliza una sintaxis declarativa para definir flujos de trabajo, donde cada paso (o «regla») especifica la entrada, la salida y los comandos para procesar los datos.
Snakemake determina automáticamente las dependencias entre los pasos y los ejecuta de manera eficiente, ya sea en una máquina local, un clúster o un entorno en la nube.

Para un proceso de análisis de variantes somáticas y de línea germinal, Snakemake proporciona la estructura para integrar múltiples herramientas y scripts en un flujo de trabajo cohesivo.
Garantiza un orden de ejecución adecuado, gestiona archivos intermedios y admite puntos de comprobación y pasos condicionales para canalizaciones dinámicas.

Los ficheros necesarios para un workflow en Snakemake son:
\begin{itemize}
\item \textbf{Snakefile:} es el archivo central que define el flujo de trabajo. Contiene reglas que especifican cómo se transforman los archivos de entrada en archivos de salida, comandos para herramientas de llamada de variantes (BWA, GATK, MuTect2, etc) y dependencias entre pasos (por ejemplo, alineación $\rightarrow$ llamada de variantes $\rightarrow$ anotación).

\begin{figure}[h!]
\centering
\includegraphics[width = 0.36\textwidth]{figs/snakemake-rule.png}
\end{figure}

\item \textbf{Fichero de configuración (config.yaml):} almacena parámetros personalizables para la pipeline, tales como rutas de archivos para los datos de entrada (FastQ, BAM, etc) y rutas del genoma de referencia, ajustes específicos de la herramienta (por ejemplo, umbrales de mutación o ploidía). Permite la reutilización y simplifica la adaptación del proceso a nuevos conjuntos de datos.

\begin{figure}[htbp]
\centering
\includegraphics[width = 0.5\textwidth]{figs/snakemake-config.png}
\end{figure}

\item \textbf{Entornos o módulos:} garantizan un entorno de software coherente y reproducible. Cuenta de archivos de entorno conda (environment.yaml) o definiciones de módulos para herramientas y dependencias necesarias. 

\begin{figure}[htbp]
\centering
\includegraphics[width = 0.3\textwidth]{figs/snakemake-env.png}
\end{figure}
\end{itemize}

Snakemake permite crear un gráfico con todos los pasos de la pipeline para verificar si los pasos están ordenados de forma correcta.
%29/11 - Lucía Sánchez
\part{Genome/Phenome Analysis}
\chapter{Genome-Wide Association Studies (GWAS)}
\section{Introducción a GWAS y características}
Los \textbf{estudios de asociación a nivel del genoma} (GWAS, por sus siglas en inglés) son un enfoque utilizado para identificar variantes genéticas asociadas a rasgos o enfermedades específicas en una población. Estos estudios analizan la asociación entre \textbf{variantes genéticas comunes} y fenotipos mediante el genotipado de grandes cantidades de SNPs (Single-Nucleotide Polymorphisms) en múltiples individuos.

El gráfico ilustra la relación entre \textbf{frecuencia alélica} y \textbf{penetrancia}. La penetrancia mide el porcentaje de individuos portadores de una variante genética que desarrollan un fenotipo o enfermedad. Por otro lado, la frecuencia alélica indica la proporción en la población de un alelo determinado que causa un fenotipo.
\begin{itemize}
\item \textbf{Enfermedades mendelianas:} tienen alta penetrancia (con una mutación se desarrolla la enfermedad) pero baja frecuencia alélica. Un ejemplo es la talasemia, una enfermedad autosómica recesiva que afecta la síntesis de las cadenas alfa y beta de la hemoglobina, provocando anemia severa.
\item \textbf{Enfermedades complejas:} tienen baja penetrancia pero alta frecuencia alélica. GWAS se enfoca en estas variantes comunes que influyen parcialmente en el riesgo de enfermedades complejas como las enfermedades cardiovasculares, que involucran factores genéticos y ambientales.
\end{itemize}

\begin{figure}[htbp]
\centering
\includegraphics[width = \textwidth]{figs/gwas.png}
\end{figure}

Las variantes estudiadas en GWAS son:
\begin{itemize}
\item Single-Nucleotide Variant (SNV): Cambios de una sola base en el ADN causados por errores durante la meiosis o daño en el ADN de las células germinales.
\item Single-Nucleotide Polymorphism (SNP): SNVs presentes en al menos un 1\% de la población.
\end{itemize}

GWAS analiza grandes cantidades de SNPs (entre 500,000 y 1,000,000 por muestra). Esto es posible gracias a plataformas tecnológicas como Illumina y Affymetrix.

Las herramientas utilizadas en GWAS son:
\begin{itemize}
\item PLINK: Software especializado en la manipulación, resumen y limpieza de datos genéticos.
\item R: Utilizado para análisis estadístico y visualización.
\item Bases de datos:
\begin{itemize}
\item dbSNP: Para obtener información sobre variantes conocidas.
\item GWAS Catalog: Repositorio de estudios GWAS publicados.
\end{itemize}
\end{itemize}

\section{Realizar un GWAS}
El primer paso es \textbf{seleccionar una población de estudio}. Esto depende de la pregunta experimental y hay que tener un tamaño muestral suficiente para asegurar potencia estadística. Si el estudio es dicotómico, habría que tener casos y controles para ver la asociación entre presencia y ausencia. Si por el contrario el estudio es cuantitativo, hay que tener medidas cuantitativas. 

A las personas se las \textbf{genotipa} mediante Whole-genome Sequencing (WGS), whole-exome sequencing (WES) o microarrays (análisis de SNPs concretos para analizar variantes preseleccionadas). 

Una vez secuenciados los datos, hay que \textbf{procesarlos}. En algunos casos hay que anonimizar los datos, ver si hay relaciones familiares entre muestras, sexo, información fenotípica, etc. También es necesario realizar control de calidad. La imputación permite predecir variantes no genotipadas mediante patrones de asociación conocidos. Por último, se realiza un \textbf{test de asociación} para analizar la relación entre variantes genéticas y el fenotipo.

\subsection{Control de calidad}
\subsubsection{Missingness}
En el control de calidad, se mira el missingness o la ausencia tanto por SNP como por individuo. Se eliminan los SNPs o individuos con altos porcentajes de datos ausentes. Los valores recomendados son tener al menos un 95\% de información por muestra y un 95-99\% de información por SNP (call rate).

\begin{figure}[htbp]
\centering
\includegraphics[width = 0.8\textwidth]{figs/missingness.png}
\end{figure}

\subsubsection{Discrepancia por sexo}
Se verifica la concordancia entre el sexo genotípico (tasa de homocigosis en el cromosoma X) y el sexo declarado.

%Para estudiar la discrepancia por sexo, se mira la diferencia entre el sexo asignado y el determinado basado en la información genotípica. Se determina mediante la computación de la tasa de homocigosis de los SNP del cromosoma X.

\subsubsection{Minor Allele Frequency (MAF)}
El Minor Allele Frequency (MAF) se define como la frecuencia del alelo menos frecuente en cada locus. Los GWAS se centran en variantes comunes asociadas a enfermedades en la población. Las variantes raras tienen baja potencia estadística.
Las variantes con un MAF muy bajo también se ven afectadas más fácilmente por errores de genotipado. Se utilizan los siguientes límites: 1-5\% para GWAS de unos cientos o mil individuos y más bajo (0,1\%) para tamaños muestrales más grandes, como UK Biobank.

\subsubsection{Hardy-Weinberg Equilibrium (HWE)}
El equilibrio de Hardy-Weinberg o ley de Hardy-Weinberg establece que en un apareamiento aleatorio tanto las frecuencias alélicas como genotípicas de una población permanecen invariables. Para que este equilibrio se dé, se deben cumplir los siguientes supuestos: apareamiento aleatorio, alelos femeninos y masculinos independientes, frecuencias alélicas idénticas entre machos y hembras, tamaño poblacional grande (infinito), no hay efecto de migración, mutación o selección natural. Para calcular las frecuencias genotípicas, se utiliza la siguiente fórmula:
$$P(G_i) = \sum_{j=1}^6 P(G_i|MT_j) \cdot P(MT_j)$$

\begin{figure}[htbp]
\centering
\includegraphics[width = \textwidth]{figs/hwe.png}
\end{figure}

Para asimilar esto, vamos a realizar un ejercicio en el que calculamos el equilibrio Hardy-Weinberg:
\begin{table}[h!]
\centering
\begin{tabular}{|l|c|ccc|}
\hline
\textbf{Mating types} & \textbf{Frequency} & \multicolumn{3}{c|}{\textbf{Frequency of zygotes}} \\ 
\cline{3-5}
                      &                    & \textbf{AA} & \textbf{AB} & \textbf{BB} \\ \hline
MT1: AA x AA          & $g_0g_0 = g_0^2$   & 1           & -           & -           \\
MT2: AA x AB          & $g_0g_1+g_1g_0 = 2g_0g_1$ & 0.5         & 0.5         & -           \\
MT3: AA x BB          & $g_0g_2 + g_2g_0= 2g_0g_2$ & -           & 1           & -           \\
MT4: AB x AB          & $g_1g_1 = g_1^2$   & 0.25        & 0.5         & 0.25        \\
MT5: AB x BB          & $g_1g_2+g_2g_1 = 2g_1g_2$ & -           & 0.5         & 0.5         \\
MT6: BB x BB          & $g_2g_2 = g_2^2$   & -           & -           & 1           \\ \hline
\end{tabular}
\caption{Tabla de frecuencias de tipos de apareamiento y cigotos.}
\label{tab:zygotes}
\end{table}

En base a los resultados de la tabla \ref{tab:zygotes}, las frecuencias genotípicas son:
$$q^2 = P(AA) = 1 \cdot g_0^2 + \frac{2g_0g_1}{2} + \frac{g_1^2}{4} = g_0^2 + g_0g_1 + \frac{g_1^2}{4} = (g_0 + \frac{g_1}{2})^2$$
$$p^2 = P(BB) = \frac{g_1^2}{4} + \frac{2g_1g_2}{2} + g_2^2 = \frac{g_1^2}{4} + g_1g_2 + g_2^2 = (g_2 + \frac{g_1}{2})^2$$
$$2pq = P(AB) = \frac{2g_0g_1}{2} + 1 \cdot 2g_0g_2 + \frac{g_1^2}{2} + \frac{2g_1g_2}{2} = g_0g_1 + 2g_0g_2 + \frac{g_1^2}{2} + g_1g_2 = 2(g_2 + \frac{g_1}{2})(g_0 + \frac{g_1}{2})$$

Para testar las proporciones HWE, se utiliza el test del chi cuadrado.

\begin{figure}[htbp]
\centering
\includegraphics[width = \textwidth]{figs/chi-cuadrado.png}
\end{figure}

Mide lo que difiere los resultados observados con los resultados esperados. El problema es que con los GWAS, este test no es del todo preciso, por lo que se emplea el exact test.

Una vez calculado el HWE y con el test se ve cómo difieren los resultados, se ve si se está violando la ley de HW, es decir, si las frecuencias genotípicas son significativamente diferentes de las esperadas. En GWAS, se asume que desviaciones de HWE se deben a errores del genotipado. En el caso de estudios binarios, el límite del HWE es menos estricto en casos que en controles, ya que la violación de la ley puede indicar una asociación genética real con riesgo a enfermedad. Para estudios cuantitativos, se emplea un p-valor menor a 1e-6. 

\subsubsection{Heterocigosidad}
La heterocigosidad indica la proporción de loci heterocigotos en un individuo, es decir, se refiere a la presencia de los dos alelos en un SNP de un individuo. Se recomienda eliminar todos los individuos que se desvíen $\pm 3 SD$ de la media:
$$HeterozygosityRate_ind = \frac{NonMissingCounts - HomozygousGenotypeCount}{NonMissingCounts}$$
Un alto nivel de heterozigosidad se puede deber a una calidad baja de las muestras o contaminación, y unos niveles bajos a inbreeding o una relación entre las muestras.

\subsubsection{Relatedness}
Relatedness es el último paso del control de calidad. En los GWAS más comunes, se asume que no hay asociación entre los participantes del estudio. El grado de relatedness se puede definir como número de alelos compartidos entre los individuos dos a dos. Se mide mediante identity by descent (IBD), que es la proporción de los genomas de dos individuos compartiendo alelos heredados de un ancestro común. 

\begin{figure}[htbp]
\centering
\includegraphics[width = \textwidth]{figs/ibd.png}
\end{figure}

Se diferencian Identity-by-state (IBS) de Identity-by-descent (IBD). En IBS, los alelos compartidos entre individuos son en un locus particular debido a evolución convergente, ancestros comunes o eventos mutacionales similares y se computa como sin información sobre herencia, mientras que en IBD los alelos compartidos entre individuos en un locus particular se debe a un ancestro comun y se debe estimar la probabilidad de heredad la misma copia de un alelo.

En estudios de población estándar, se recomienda eliminar uno de los individuos con un IBD mayor de 0,2. El desequilibrio de ligamiento hace referencia a la herencia conjunta de genes en diferentes loci en el mismo cromosoma en una población concreta. Los SNP están en LD cuando la frecuencia de asociación de sus alelos es superior a la esperada si los loci fueran independientes y estuvieran asociados al azar. 

\section{Práctica: Proyecto HapMap internacional}
El objetivo es elaborar un mapa de haplotipos del genoma humano. La información está disponible gratuitamente en conjuntos de datos públicos. Comenzó con una reunión, celebrada del 27 al 29 de octubre de 2002, y alcanzó su objetivo de completar el mapa en tres años. Se trata de una colaboración entre investigadores de centros académicos, grupos de investigación biomédica sin ánimo de lucro y empresas privadas de Japón, Reino Unido, Canadá, China, Nigeria y Estados Unidos. El HapMap identifica entre 250.000 y 500.000 SNP marcados (casi tanta información cartográfica como los 10 millones de SNP). Cuenta con muestras procedentes de Yoruba, Japón, China y Estados Unidos (residentes en Utah con ascendencia del norte y oeste de Europa).

Los haplotipos son un conjunto de alelos de un cromosomas que se han heredado conjuntamente de un mismo progenitor al estar localizados de forma próxima en el cromosoma. Se puede limitar a un solo gen o a múltiples. Los \textbf{TagSNP} son SNPs representativos en una región del genoma con un alto linkage disequilibrium. 

\begin{figure}[htbp]
\centering
\includegraphics[width = \textwidth]{figs/tagsnp.png}
\end{figure}

El Proyecto Internacional HapMap nació para desarrollar un mapa de haplotipos del genoma humano. 

Durante las prácticas de esta parte de la asignatura, haremos uso de los datos del HapMap para determinar asociaciones entre los SNPs de este estudio y la variable de resultado, en lo que se conoce como estudios de asociación de genoma completo (GWAS).

Como ya hemos visto en clase, el control de calidad es el primer paso en los GWAS. Este proceso es crucial para eliminar las muestras de baja calidad, la contaminación, deshacerse de los errores generados durante el SNP calling o controlar la subestructura de la población, entre otras cosas. Esto es esencial para asegurar que nuestros datos tienen suficiente calidad para realizar las pruebas de asociación.

Como recordatorio, el control de calidad se divide en algunos pasos:
\begin{enumerate}
\item Control for missingness
\item Sex Discrepancy
\item Minor allele frequency
\item Hardy-Weinberg equilibrium
\item Heterozygosity
\item Relatedness
\item Population substructure
\end{enumerate}

En este pipeline, controlaremos los seis primeros pasos. Para ello se utilizará principalmente PLINK, una herramienta que permite estudiar las características de los datos y limpiarlos de forma sencilla y eficaz. También se utilizará R para trazar algunos resultados y ayudar en la determinación de los umbrales (librerías ggplot2 y dplyr).

\subsection{Missingness por individuo y por SNP}
La falta de datos (missingness) se refiere al grado de datos no disponibles a nivel de SNP o de individuo y está directamente asociada con la calidad de los datos. Una buena práctica consiste en eliminar los SNP/individuos con una elevada proporción de omisión.

Para determinar esta proporción, podemos utilizar `--missing` de PLINK. Este flag genera dos archivos que muestran la proporción de SNPs perdidos por individuo y la proporción de individuos perdidos por SNP, respectivamente. 

En este paso, se crean los ficheros plink.lmiss con la información de missigness de los SNP y plink.imiss con la información de missigness de los individuos.

\begin{lstlisting}[language=bash]
plink --bfile HapMap_3_r3_1 --missing --out plink
\end{lstlisting}

Como dice el informe, tenemos 1457897 variantes y 165 personas (80 hombres / 85 mujeres).

\subsection{Estudio de Missingness de SNP}

\begin{lstlisting}[language=R]
snpmiss <- read.table(file="plink.lmiss", header=TRUE)

kable(head(snpmiss), caption = "SNP missingness information") %>%
  kable_styling(bootstrap_options = c("striped", "hover", "condensed", "responsive"), full_width = FALSE) 
\end{lstlisting}

Una vez cargados los datos, los visualizamos:

\begin{lstlisting}[language=R]
p <- ggplot(snpmiss, aes(x=F_MISS)) +
  geom_histogram(color="black", fill="#E69F00", binwidth=.0025) +
  ggtitle('Histogram SNP missingness') +
  ylab('Frequency') +
  geom_vline(xintercept = .02, linetype="dotted",
               color = "black", linewidth=.9)
p 
\end{lstlisting}

\begin{figure}[htbp]
\centering
\includegraphics[width = 0.8\textwidth]{figs/hist-snpmiss.png}
\end{figure}

Imaginemos que decidimos fijar un umbral en 0,02. Esto significa que estamos eliminando SNPs con más del 2\% de su información faltante.
\begin{lstlisting}[language=R]
sum(snpmiss$F_MISS > 0.02)
\end{lstlisting}

En total estamos eliminando 27454 valores. Para ver la cantidad de individuos que deben tener una ausencia información para ese SNP para que se elimine, se realiza el siguiente código y el resultado son 4 personas.
\begin{lstlisting}[language=R]
deleted_snp <- snpmiss[snpmiss$F_MISS > 0.02, ]
min(deleted_snp$N_MISS)
\end{lstlisting}

\subsubsection{Estudio de missingness de individuos}
De forma similar a como hemos hecho con el SNP missingness, queremos detectar el missingness individual. Para ello, cargar el archivo que contiene esta información y, representar los individuos falta en un histograma:
\begin{lstlisting}[language=R]
indmiss <- read.table("plink.imiss", header = TRUE)

p <- ggplot(indmiss, aes(F_MISS)) +
  geom_histogram(color="black", fill="#E69F00", binwidth=.0005)

p
\end{lstlisting}

\begin{figure}[htbp]
\centering
\includegraphics[width = 0.8\textwidth]{figs/hist-indmiss.png}
\end{figure}

Decidimos fijar de nuevo el umbral en 0,02. Esto significa que se eliminan los individuos con más de un 2\% de omisión en sus SNP, que en este caso es una persona.
\begin{lstlisting}[language=R]
sum(indmiss$F_MISS > 0.02)
\end{lstlisting}

Para ver la cantidad de SNPs que deben estar ausentes en un individuo para eliminarlo, se realiza el siguiente cálculo y el resultado son 29584.
\begin{lstlisting}[language=R]
deleted_ind <- indmiss[indmiss$F_MISS > 0.02, ]
min(deleted_ind$N_MISS)
\end{lstlisting}

\subsubsection{Filtrar SNPs e individuos}
Después de representar los resultados, concluimos que el 2\% de missingness es un buen umbral tanto para SNPs como para individuos. Por lo tanto, debemos utilizar `--geno` y `--mind` para eliminar ese porcentaje.

\begin{lstlisting}[language=bash]
# Delete SNPs with missingness >0.02.
plink --bfile HapMap_3_r3_1 --geno 0.02 --make-bed --out HapMap_3_r3_2

# Delete individuals with missingness >0.02.
plink --bfile HapMap_3_r3_2 --mind 0.02 --make-bed --out HapMap_3_r3_3
\end{lstlisting}

Tras analizar el output, vemos que no se ha eliminado ningún individuo. Esto se debe a que, como primero se eliminan los SNP, se recalcula el missingness para los individuos y el que antes sí eliminábamos, ya no cumple con la condición (los SNPs de los que no tenía información son aquellos que se han eliminado por tener poca información de individuos).

%02/12 - Lucía Sánchez
A continuación miramos la discrepancia entre sexos. La discrepancia de sexo se refiere a la diferencia entre el sexo asignado y el determinado. Puede estudiarse con `--check-sex`, que genera un documento con 6 columnas:
\begin{enumerate}
\item FID: family ID
\item IID: individual ID
\item PEDSEX: sex from the pedigree file (1 = male, 2 = female)
\item SNPSEX: sex determined by X chromosome
\item STATUS: problem/ok
\item F: the actual X chromosome inbreeding (homozygosity) estimate. This estimate allows to determine the sex of the individuals, being F < 0.2 assigned to females, and F > 0.8, to males.
\end{enumerate}

Primero utilizamos ese flag para determinar el número de personas con sexo masculino y femenino:
\begin{lstlisting}[language=bash]
plink --bfile HapMap_3_r3_3 --check-sex --out plink
\end{lstlisting}

Ese fichero se carga en R y se genera un histograma con F estimado:
\begin{lstlisting}[language=R]
sex <- read.table("plink.sexcheck", header = TRUE)
p <- ggplot(sex, aes(F)) +
  geom_histogram(color="black", fill="#E69F00", binwidth=.005)
p
\end{lstlisting}

Para ver el número de hombres y mujeres predichos:
\begin{lstlisting}[language=R]
#males
sum(sex$SNPSEX == 1)

#females
sum(sex$SNPSEX == 2)
\end{lstlisting}
Obtenemos 81 hombres y 84 mujeres. Ahora queremos ver si hay alguna discordancia entre el sexo predicho y el computado:
\begin{lstlisting}[language=R]
discordance <- sex[sex$PEDSEX != sex$SNPSEX,]
discordance
\end{lstlisting}

Vemos que hay un individuo que discrepa, por lo que queremos filtrarlo. Generamos un fichero .txt con el FID e IID de la persona problemática.
\begin{lstlisting}[language=bash]
grep "PROBLEM" plink.sexcheck | awk '{print $1, $2}' > sex_discrepancy.txt
\end{lstlisting}

Posteriormente utilizamos el flag --remove para eliminar los individuos en el fichero txt.
\begin{lstlisting}[language=bash]
plink --bfile HapMap_3_r3_3 --remove sex_discrepancy.txt --make-bed --out HapMap_3_r3_4
\end{lstlisting}

Ahora analizamos minor allele frequency (MAF). El MAF se refiere a la frecuencia del alelo menos frecuente en un locus. Debemos eliminar los SNP con un MAF bajo porque la potencia estadística de los GWAS no permite detectar asociaciones si la frecuencia del alelo es demasiado baja. Generamos un fichero .txt con los SNPs autosomales y después eliminamos aquellos con el MAF más pequeño. 
\begin{lstlisting}[language=bash]
# Select autosomal SNPs (from chromosomes 1 to 22).
awk '{ if ($1 >= 1 && $1 < 23) print $2}' HapMap_3_r3_4.bim > snp_1_22.txt
#Remove unlisted variants
plink --bfile HapMap_3_r3_4 --extract snp_1_22.txt --make-bed --out HapMap_3_r3_5
\end{lstlisting}

A continuación utilizamos el flag --freq para computar el MAF en los cromosomas autosomales.
\begin{lstlisting}[language=bash]
plink --bfile HapMap_3_r3_5 --freq --out plink
\end{lstlisting}

Cargamos el fichero generado y representamos la distribución de MAF en un histograma, estableciendo un límite del 5\%.
\begin{lstlisting}[language=R]
maf_freq <- read.table("plink.frq", header = TRUE)
p <- ggplot(maf_freq, aes(MAF)) +
	geom_histogram(color="black", fill="#E69F00", binwidth=.005) +
	geom_vline(xintercept = .05, linetype="dotted", color = "black", linewidth=.9)
p
\end{lstlisting}

Estamos eliminando 1073226 SNPs y quedándonos con 325318.
\begin{lstlisting}[language=R]
#deleting
sum(maf_freq$MAF >= 0.05)
#retaining
sum(maf_freq$MAF < 0.05)
\end{lstlisting}

Eliminamos los SNPs.
\begin{lstlisting}[language=bash]
plink --bfile HapMap_3_r3_5 --maf 0.05 --make-bed --out HapMap_3_r3_6
\end{lstlisting}

El siguiente paso es eliminar los SNPs que no estén en equilibrio Hardy-Weinberg. El equilibrio de Hardy-Weinberg (HWE) establece que, en un apareamiento aleatorio, las frecuencias alélicas y genotípicas permanecen constantes o estables en una población si no se introducen factores perturbadores. `--hardy` escribe una lista de recuentos de genotipos y estadísticas de la prueba exacta de equilibrio Hardy-Weinberg.
\begin{lstlisting}[language=bash]
plink --bfile HapMap_3_r3_6 --hardy --out plink
\end{lstlisting}

Posteriormente elegimos los SNPs con un p-valor HWE < 0,0001.
\begin{lstlisting}[language=bash]
awk '{ if ($9 <0.00001) print $0 }' plink.hwe > plinkzoomhwe.hwe
\end{lstlisting}

Y ahora creamos los histogramas:
\begin{lstlisting}[language=R]
hwe <- read.table("plink.hwe", header = TRUE)
hwe
p <- ggplot(hwe, aes(P)) +
  geom_histogram(color="black", fill="#E69F00", binwidth=.05)
p

hwe_zoom <- read.table("plinkzoomhwe.hwe", header = FALSE)
hwe_zoom
p <- ggplot(hwe_zoom, aes(V9)) +
  geom_histogram(color="black", fill="#E69F00")
p
\end{lstlisting}

\section{Consideraciones de GWAS}
En cuanto al diseño del estudio, pueden diferenciarse los estudios basados en población, en familia o en poblaciones aisladas.

\subsection{Population-based GWAS}
Se trata de un estudio de asociación genética con individuos no emparentados. El estudio más común es uno de casos y controles, siendo los casos personas con presencia de un fenotipo y los controles ausencia del mismo. Los individuos pueden seleccionarse activamente y los controles se pueden emparejar con los casos (con respecto al sexo, factores de riesgo, ...). Se realiza un reclutamiento activo. Estos estudios tienen buena potencia y son rentables si la frecuencia de la enfermedad en la población es baja (<20\%). Si la frecuencia de los casos es mayor que la frecuencia basada en población, se debe ajustar por covariantes durante el análisis estadístico. Los casos y controles que no se genotipen conjuntamente deben tener una corrección (batch correction) para ajustar por covariables.

Los estudios de casos y controles se pueden clasificar en estudios retrospectivos y prospectivos. En los \textbf{estudios retrospectivos}, los sujetos se seleccionan en función de su estado de enfermedad. Se suele utilizar con enfermedades raras, ya que no es viable escoger participantes y esperar a que generen la enfermedad. Se obtienen los datos genéticos y ambientales.

En los \textbf{estudios prospectivos}, se establece una cohorte y se realiza un genotipado base de todos los sujetos. Se realiza un seguimiento de los individuos y se observa si hay un desarrollo de la enfermedad. Aquellos que la desarrollen pasan a ser controles, y aquellos que no serán los controles. 

Las ventajas es que estos estudios son muy rentable para asociaciones a gran escala, pero tiene la desventaja de obtener subgrupos poblacionales, generando asociaciones falsas debido a las subpoblaciones.

\subsection{Family-based GWAS}
Este estudio utiliza sujetos recogidos en familias, y se utilizó frecuentemente en los inicios de GWAS. En el caso del trío caso-padres, se genotipa a la descendencia afectada y a los padres. Se compara entre el número de alelos marcadores transmitidos de padres a descendientes con el número de alelos no transmitidos. Se puede tralizar la prueba de desequilibrio de transmisión (TDT) o prueba de asociación para identificar el vínculo genético entre un marcador genético (SNP) y un rasgo (fenotipo). Examina la segregación de un alelo dentro de una familia. La ventaja es que es robusto frente a estratificación poblacional y con enfermedades de baja prevalencia (< 1\%). Además, se estudian los efectos de un alelo en un fenotipo individual de sus efectos indirectos en miembros familiares cercanos. No obstante, se requiere un tamaño muestral más grande que el GWAS poblacional para alcanzar la misma potencia estadística y es menos eficiente en el caso de enfermedades de aparición tardía.

\subsection{Poblaciones aisladas}
Las poblaciones aisladas son grupos separado de sus poblaciones vecinas por barreras (geográficas, culturales o lingüísticas) y que tienen un flujo genético mínimo desde ellas. Estas poblaciones aisladas han permanecido aisladas durante un periodo prolongado, teniendo un flujo genético restringido con las poblaciones vecinas. Estos estudios tienen una mayor precisión de imputación que otras pruebas con un desequilibrio de ligamiento de largo alcance. 
Los descubrimientos en poblaciones aisladas son muy difíciles de replicar en otras poblaciones. Así, variantes funcionales raras pueden estar presente en mayor frecuencia en poblaciones aisladas, habiendo así una potencia aumentada para estudios de asociación de estas variantes.

\subsection{Subestructura poblacional - práctica}
En esta práctica estudiaremos relatedness y la estratificación poblacional. La estratificación poblacional puede ser la principal fuente de confusión. Ejemplo: estudio de casos y controles, en el que las diferencias genotípicas entre casos y controles se deben a los distintos orígenes de la población (casos: europeos, controles: asiáticos) y no a un efecto sobre el riesgo de enfermedad. La confusión se debe a que la subestructura de la población no está distribuida por igual entre los grupos de casos y controles. Así, una señal de asociación surgirá no por una asociación entre un fenotipo y un SNP, sino por diferencias de frecuencia alélica entre las poblaciones que comprenden los casos y los controles. Esto tiene dos soluciones: eliminar los individuos de ascendencia divergente o establecer la ascendencia como covariable/efecto aleatorio en modelos mixtos.

Los métodos para identificación de individuos con diferencias a larga escala en ascendencia es mediante una PCA (principal component analysis) o MDS (multidimensional scaling). MDS calcula la proporción de alelos compartidos entre cada par de individuos para identificar la variación genética para cada individuo. Los resultados se muestran en un gráfico para explorar la distribución de individuos en los datos. Por ejemplo, estudio genético que incluya sujetos de Asia y Europa. El análisis MDS revelaría que los asiáticos son genéticamente más parecidos entre sí que a los europeos.

%04/12 - Lucía Sánchez
\chapter{Análisis estadístico o de asociación}
La imputación permite predecir otros SNPs no secuenciados debido a que se heredan conjuntamente, ampliando así la información de los SNPs. Los softwares que se pueden utilizar son:
\begin{itemize}
\item \textbf{IMPUTE2 y MACH}: utilizan un modelo de Markov oculto (HMM) para estimar los genotipos que faltan, mediante la inferencia de haplotipos y el uso de parámetros genéticos previamente especificados, como las tasas de mutación y de recombinación
\item \textbf{BEAGLE}: no necesita tales parámetros. Estima los valores que faltan mediante el uso de haplotipos agrupados localmente con algoritmos HMM y de maximización de expectativas (EM).
\end{itemize}

El test de asociación se realiza después de la imputación. El test de asociación que se realice depende del fenotipo (binario o continuo), el control de covariantes (edad, sexo) y la estructura poblacional (estratificación u homogeneidad), pudiendo ser regresiones lineales, regresiones lineales múltiples, modelos lineares mixtos, regresiones logísticas, etc. Los tipos de análisis que se pueden encontrar en función de la dominancia son:
\begin{itemize}
\item \textbf{Modelo dominante:} La presencia del alelo B aumenta el riesgo de enfermedad en la misma medida para los genotipos BB y AB, en comparación con el riesgo de referencia para los AA.
\item \textbf{Modelo codominante o aditivo:} Cada copia adicional del alelo B aumenta el riesgo de enfermedad de forma aditiva, o por el contrario, aumenta el efecto protector.
\item \textbf{Modelo recesivo:} Se necesitan dos copias del alelo B para expresar la característica fenotípica relacionada con este alelo.
\end{itemize}

\section{Tipos de test de asociación}
\subsection{Modelos lineares}
Se describen mediante la fórmula:
$$y = \beta_0 + \beta_1 \cdot X_{1i} + \epsilon_1$$

Siendo $y$ la variable dependiente (respuesta/outcome), $\beta_0$ el término constante del modelo, $\beta_1$ el coeficiente de regresión, $X_{1i}$ la variable independiente o predictora y $\epsilon_1$ el término del error. $i$ representa el número de observaciones o muestras.

Los cambios que se producen en el outcome (y) se puede modelar como una función lineal de la variable independiente. Así, un cambio en la variable independiente produce un cambio en la variable dependiente, siendo ésta numérica. 

Un ejemplo: Se desea estimar si el tabaco (medido como el número de cigarrillos fumados al mes) influye en el volumen residual pulmonar (VR: volumen de aire que queda en los pulmones tras una espiración máxima). Si existe una asociación entre el tabaco y el VR, el número de cigarrillos (variable independiente) se asociaría con una reducción del VR (variable dependiente)

En este caso, la variable predictora va a ser categórica, representando los distintos SNPs. Normalmente, el estimador (la pendiente) suele estar relacionado con el p-valor. 

\subsection{Modelos de regresión lineal múltiple}
Se incluyen otros predictores o covariables. Los términos en la variable dependiente se modelan con todos los predictores independientes. Una manera de regular la estratificación de la población es metiéndolo como covariable.

Siguiendo con el ejemplo previo, nos damos cuenta de que el volumen residual puede verse afectado no sólo por el número de cigarrillos fumados al mes, sino también por la edad y el sexo. Ahora tendremos tres variables independientes para probar la dependiente: tabaco, sexo y edad.

\subsection{Modelos lineares mixtos}
Estos modelos sirven para modelar estructuras de datos más complejas. Las variables predictoras se dividen en dos:
\begin{itemize}
\item \textbf{Efecto fijo:} son las variable que se espera que tengan un efecto en la variable respuesta. Un ejemplo sería los SNP y el entorno o las covariables clínicas.
\item \textbf{Efectos aleatorios:} no nos interesa su impacto en la variable de respuesta, pero sabemos que pueden estar influyendo en los patrones que observamos, por lo que queremos separarlos del modelo para ver cómo verdaderamente el efecto fijo afecta al outcome. Aquí se contarían variables categóricas que queremos controlar.
\end{itemize}

Un ejemplo sería un estudio de la deprivación del sueño. Se restringe el tiempo de sueño de 18 individuos y la reacción de su organismo durante 10 días. El objetivo es determinar cómo cambia la reacción de los individuso durante su deprivación de sueño. Si se mide como regresión lineal simple, la variabilidad va cambiando. Si se clusteriza cada dato por individuo, se puede ver que la recta que se traza es más ajustada a los datos, demostrando que hay una estructura compleja que no se puede determinar por un modelo lineal.

\begin{table}[htbp]
\centering
\begin{tabular}{p{7cm}|p{7cm}}
Modelo de regresión lineal & Modelos mixtos \\ \hline
Suponen una relación lineal entre las variables dependientes e independientes. & Relajan el supuesto de independencia entre las observaciones. \\
\\
Adecuado para analizar datos con una estructura simple (se supone que las observaciones son independientes entre sí). & Adecuado para analizar datos con una estructura agrupada (las observaciones están anidadas dentro de grupos / mediciones repetidas en los mismos sujetos). \\
\\
Los modelos de regresión lineal suelen incluir efectos fijos: parámetros asociados a las variables predictoras. Estos efectos son constantes en todos los niveles de cualquier variable de agrupación. & Los modelos mixtos incluyen efectos fijos (como los modelos lineales) y efectos aleatorios (captan la variabilidad a distintos niveles).\\
\\
Tareas de regresión simple: se desea predecir una variable de resultado continua a partir de una o más variables predictoras. & Se utilizan en análisis de datos longitudinales, análisis de medidas repetidas y modelización jerárquica. Son apropiados cuando los datos no son independientes y se desea tener en cuenta las correlaciones dentro del grupo o la variabilidad específica de los sujetos.
\end{tabular}
\end{table}

\subsection{Regresión logística}
La regresión logística sigue la fórmula:
$$y = \frac{e^{(\beta_0 + \beta_1X_{1i)}}}{1 + e^{(\beta_0 + \beta_1X{1i})}}$$

Los cambios en la variable dependiente (y) pueden modelizarse como una función logística de las variables independientes. Hay dos variantes:
El modelo de regresión logística múltiple es como la lineal múltiple, pero logística, es decir, se añaden varias variables independientes. El modelo multinomial cuenta con una variable dependiente con más de dos categorías.

Siguiendo con el ejemplo, nos interesa detectar si existe una asociación entre el número de cigarrillos: (variable independiente continua) y el volumen residual inferior a 1 litro (variable dependiente binaria: sí/no). En la regresión logística múltiple, se añadiría sexo como covariable, mientras que en la regresión logística multinomial la variable dependiente tendría más de dos categorías.

\section{Visualización de datos}
El plot se denomina como Manhattan plot. Las asociaciones significativas se muestran en la parte superior del gráfico. El eje x va por orden en los distintos cromosomas. 

\begin{figure}[htbp]
\centering
\includegraphics[width = \textwidth]{figs/manhattan-plot.png}
\end{figure}

\section{Nivel de significancia de GWAS}
A la hora de mirar el nivel de significancia de GWAS, nos encontramos con el problema del testeo múltiple. El p-valor representa la probabilidad de obtener resultados tan extremos como los observados si la hipótesis nula es correcta. La hipótesis nula dice que no hay una relación significativa entre el predictor y la variable respuesta. El nivel de significancia se suele poner en 0,05. 

Por ejemplo, estamos probando (hipótesis nula) que no hay una relación significativa entre el predictor (SNP, edad y sexo) con la variable respuesta (volumen residual). Si el p-valor es 0,03, si no hubiera una relación entre esas variables (y la hipótesis nula fuera cierta), la probabilidad de tener resultados tan extremos como los observados sería de 0,03. 

Con un nivel de significancia de 0,05, todavía hay una probabilidad de que no haya una relación significativa, aunque sea pequeña. Esto no es un problema cuando se realiza un solo análisis, pero cuando se realizan muchos, se incrementa el número de falsos positivos. Por ello, se debe controlar el testo múltiple y ajustar el p-valor. Esto se puede realizar de varias formas:
\begin{itemize}
\item \textbf{Corrección de Bonferroni:} divide el nivel de significancia por el número de análisis realizados para ver el nivel de significancia que se debe comprobar en cada análisis individual. No obstante, es algo restrictivo, incrementando los falsos negativos.
\item \textbf{False Discovery Rate (FDR):} se describió por Benjamini y Hochberg en 1995, y monitoriza el número de falsos positivos en relación al número de resultados positivos, siendo así menos estricto.
\end{itemize}

%09/12 - Fátima
\chapter{Epidemiología molecular: introducción a la inferencia causal}
Los biobancos son grandes estudios poblacionales que siguen a personas aparentemente sanas. En el portal de \href{https://portal.gdc.cancer.gov/analysis_page?app=}{TGCA} se han creado algunas herramientas que permiten construir una cohorte: obtener los pacientes con un tipo de cáncer para sacar cierta información. Se pueden ver los tipos de datos disponibles y aplicando distintos filtros. Como bioinformáticos, no nos gusta acceder a las bases de datos mediante front-end, ya que esto solo sirve para consultas pequeñas. Nosotros accederemos desde R (script en la carpeta de prácticas, \texttt{tcga.R}).

Para prácticamente todos los biobancos existen APIs para acceder y un paquete de R para poder minar las bases de datos. En este caso, como queremos acceder a TGCA, utilizaremos el paquete de TCGAretriever.

 \section{Correlación vs causalidad}
 La correlación no implica causalidad. Hay muchos ejemplos en los que hay variables confusoras que hacen que haya correlación entre ambos eventos, pero no causalidad: ventas de helados y ataques de tiburones (verano como variable confusora), consumo de chocolate y premios Nobel en un país (país con gran inversión y poder adquisitivo). 
 
Hay veces que es complicado ver si algo es causal o no. La ciencia ha tratado de investigar la causalidad en variables asociadas en modelos animales y ensayos clínicos. En ratones, se puede simular la situación mediante knock-outs para demostrar causalidad de forma directa, pero la experimentación con animales es complicada y de aquí a unos años puede estar incluso prohibida. En el caso de los ensayos clínicos, se pueden buscar individuos con características que se quieren simular, se puede ver el outcome. No obstante, son muy caros y muy largos. Otra opción es mediante modelos digitales que permitan simular \textit{in silico} la asociación y causalidad con distintos fenotipos. Ahora que hay tantos biobancos y se invierte tanto dinero en ellos, se pueden obtener muchos datos observacionales. 

\section{Inferencia causal}
La inferencia causal nos permite sacar conclusiones causales a partir de los datos observacionales de los que a priori solo podemos sacar asociaciones. Para ver causalidad habría que comparar a un mismo individuo en dos situaciones, pero es complicado obtener esos estados (no puedes comparar el efecto de una aspirina en una persona comparando su estado tomándosela y no tomándosela, ya que no se puede realizar ambas). 

El efecto causal de un tratamiento en un mismo individuo es la diferencia entre el valor del outcome si el individuo se trata y el valor del outcome si no se trata. No obstante, es imposible obtener esos dos outcomes. 

Se hizo un estudio en el que se miró si el consumo de alcohol tiene relación con sufrir un infarto. Idealmente, se buscaba tener datos de las distintas personas sin consumo de alcohol y con consumo, midiendo el tiempo hasta que sufre un infarto. Cada una de las dos columnas tiene una distribución (función de probabilidad). Con esta información, lo que tenemos en un estudio observacional es para una persona, una variable que dice si toma o no alcohol y el tiempo de supervivencia en su grupo correspondiente. Esto es un problema de valores perdidos: sabiendo la distribución de los datos (de estudios observacionales anteriores), podemos simular y rellenar la tabla con los valores faltantes (el tiempo hasta sufrir un infarto para cada persona en el grupo faltante).

\subsection{Neyman-Rubin Causal Model}
El teorema de Rubin dice que si la variable es totalmente independiente del outcome que se mide (si tomar alcohol y tener un infarto es independiente), no se necesita para cada persona los dos valores, ya que se puede sustraer la media de las dos variables y se obtiene una estimación muy buena que correspondería a tener para cada individuo los dos valores y extraer la media. Hay variantes genéticas que predisponen al consumo de alcohol, y variantes genéticas que predisponen al infarto. Si las variantes son diferentes, se puede clasificar a cada persona en su grupo.

\section{Genes como variables instrumentales}
De forma aleatoria, nosotros tenemos un genotipo u otro, por lo que se puede considerar como variable instrumental. Esto sirve para la \textbf{mendelian randomization}, la cual se basa en el supuesto de que las variantes genéticas aportan una fuente de variación exógena en la exposición.

Hay muchos supuestos:
\begin{enumerate}
\item La variante genética tiene que estar asociada con el trait asociado
\item La variante genética no puede estar asociada con ninguna variable confusora
\item La variante genética no puede estar asociada directamente con el outcome
\end{enumerate}

Esto se puede realizar en R con el paquete \texttt{MendelianRandomization}.

betaX y betaXse son vectores numéricos que describen las asociaciones de las variantes genéticas con la exposición. 
betaX son los coeficientes beta de los análisis de regresión univariable de la exposición/tratamiento sobre cada variante genética. 
betaXse son los errores estándar.
betaY y betaYse son vectores numéricos que describen las asociaciones de las variantes genéticas con el resultado: 
betaY son los coeficientes beta de los análisis de regresión del resultado en cada variante genética 
betaYse son los errores estándar
Correlación es una matriz con las correlaciones con signo entre las variantes. Si no se proporciona una matriz de correlación, se asume que las variantes no están correlacionadas.
exposition es una cadena de caracteres que indica el nombre del factor de riesgo, por ejemplo, LDL-colesterol
outcome es una cadena de caracteres que indica el nombre del resultado, por ejemplo, cardiopatía coronaria. 
snps es un vector de caracteres que contiene los nombres de las distintas variantes genéticas (SNP) del conjunto de datos, por ejemplo rs12785878. 

No es necesario nombrar la exposición, el resultado o los SNPs, pero estos nombres se utilizan en las funciones gráficas y pueden ser útiles para realizar un seguimiento de los distintos análisis.

%11/12 - Jorge de la Barrera jdelabarrera@cnic.es
\chapter{Genetic Susceptibility (Polygenic Risk Scores)}
\section{La búsqueda de variantes causantes de enfermedad}
Las enfermedades monogénicas tienen pocas variantes, pero con mucho efecto. Las enfermedades poligénicas van al revés, teniendo muchas variantes pero con poco efecto. Entre medias están las enfermedades oligogénicas. 

\begin{figure}[htbp]
\centering
\includegraphics[width = 0.8 \textwidth]{figs/Imagen1.jpg}
\end{figure}

Los trastornos monogénicos están causados por una única variante de alta penetrancia en un gen. 
Son poco frecuentes en la población y se asocian a un alto riesgo relativo de desarrollar la enfermedad. 
Por ejemplo, la enfermedad de Huntington es un trastorno monogénico.

Los trastornos poligenéticos están influidos por los efectos combinados de múltiples SNP comunes, cada uno con un pequeño efecto sobre el riesgo de enfermedad. 
Por ejemplo, la diabetes tipo 2 es un trastorno poligénico.

\section{Polygenic Risk Score (PRS)}
El PRS (polygenic risk score) es un número que estima la predisposición genética de un individuo a una enfermedad o rasgo específico. Proporciona una medida de la predisposición genética. Esta estimación se basa en su perfil genotípico y en datos estadísticamente ponderados de GWAS relevantes. Se calculan combinando información de miles de variantes genéticas, cada una con tamaños de efecto pequeños.

\begin{figure}[htbp]
\centering
\includegraphics[width = 0.8 \textwidth]{figs/Imagen2.png}
\end{figure}

Las enfermedades monogénicas suelen ser raras, pero las poligénicas comunes. Si nosotros somos capaces de tener en un score el riesgo, se pueden estratificar los individuos según su riesgo, creando predicciones más ajustadas y políticas de prevención más acorde. Así, los PRS permiten realizar una prevención más precisa, mejorar el diagnóstico, y mejorar los tratamientos. En resumen:
\begin{itemize}
\item \textbf{Prevención selectiva:}
Estratificar a los individuos según su riesgo genético de desarrollar diversas enfermedades comunes.
Optimizar el uso del cribado y los tratamientos preventivos.
Informar a los programas de cribado de la población identificando a los individuos con mayor riesgo.

\item \textbf{Diagnóstico temprano:}
Facilitar el diagnóstico de enfermedades complejas, especialmente en conjunción con otros factores clínicos.
Afinar las estimaciones de riesgo en familias con antecedentes de determinadas enfermedades, especialmente cuando se combinan con pruebas de riesgo monogénico.

\item \textbf{Intervenciones personalizadas:}
Mejorar la atención personalizada a los pacientes.
Orientar las decisiones terapéuticas al proporcionar información sobre la probabilidad de que un individuo responda a terapias específicas.
Informar sobre el pronóstico al ofrecer información sobre la posible evolución de la enfermedad y sus resultados.
\end{itemize}

\begin{figure}[htbp]
\centering
\includegraphics[width = 0.8 \textwidth]{figs/Imagen3.png}
\end{figure}

\subsection{Entendiendo los PRSs}
Las PRS son puntuaciones genéticas que \textbf{combinan los efectos} de muchas variantes genéticas (SNPs) para evaluar la \textbf{predisposición de un individuo} a un rasgo o enfermedad específica.

Las PRS se desarrollan \textbf{utilizando datos de estudios de asociación de genoma completo (GWAS)} . 	Estos estudios identifican los SNP asociados a un rasgo o enfermedad concretos mediante el análisis de los genomas de grandes poblaciones. Su fórmula general es:
$$PRS = \sum (efecto del SNP \cdot alelo portado)$$.

Esta fórmula calcula la PRS sumando el número de alelos de riesgo que porta una persona en cada SNP relevante, ponderado por el tamaño del efecto de cada alelo.

Entre los usos del PRS se encuentran:
\begin{itemize}
\item \textbf{Predicción de riesgos genéticos}: Las PRS pueden estimar el riesgo genético de un individuo a lo largo de su vida de desarrollar determinadas enfermedades, como cardiopatías, diabetes o cáncer.
\item \textbf{Complementación de factores clínicos y ambientales}: Las PRS pueden utilizarse con otros factores de riesgo, como los antecedentes familiares, el estilo de vida y los indicadores clínicos, para ofrecer una imagen más completa del riesgo global de una persona.
\item \textbf{Mejorar el diagnóstico y el tratamiento personalizado}: Las PRS pueden ayudar a afinar el diagnóstico, sobre todo en el caso de enfermedades con síntomas que se solapan. También pueden orientar las decisiones de tratamiento, ya que las personas con diferentes PRS pueden responder de forma diferente a los mismos medicamentos.
\end{itemize}

Consideraciones importantes:
\begin{itemize}
\item Los PRS son estimaciones de riesgo, no diagnósticos. Indican una predisposición, no la certeza de desarrollar una enfermedad.
\item Las PRSs actuales suelen explicar sólo una pequeña fracción de la variación de un rasgo o enfermedad. Esto se debe a que los factores ambientales y las interacciones gen-ambiente también desempeñan un papel importante.
\item La precisión de las PRSs depende de factores como el tamaño y la diversidad de los GWAS utilizados para desarrollarlas.
\item La mayoría de las PRSs se han desarrollado utilizando datos de individuos de ascendencia europea, lo que limita su precisión y aplicabilidad a otras poblaciones.
\item Las PRSs evolucionan rápidamente a medida que avanza la investigación y se dispone de más datos.
\end{itemize}

\subsection{Fundamentos teóricos del PRS}
El PRS se basa en dos fundamentos. El primero es el \textbf{principio de la herencia poligénica}.
La herencia poligénica se refiere al concepto de que múltiples genes contribuyen a un rasgo o enfermedad.
El segundo es la \textbf{heredabilidad y su relación con la PRS}.
La heredabilidad es la proporción de variación fenotípica en una población que puede atribuirse a factores genéticos. La heredabilidad SNP se refiere a la proporción de varianza fenotípica explicada por los SNP. El poder predictivo de una PRS tenderá hacia la heredabilidad SNP de un rasgo a medida que aumente el tamaño de las muestras GWAS.

\subsection{Cálculo del PRS}
Los estudios GWAS pueden sobrestimar los efectos de los SNP, especialmente en el caso de las variantes con asociaciones más débiles o las identificadas en muestras de pequeño tamaño. Además, puede haber redundancia causada por SNP en desequilibrio de ligamiento (LD). Por ello, es necesario ajustar el PRS. Sin ajustes, la PRS puede funcionar mal en poblaciones nuevas, lo que da lugar a predicciones inexactas.

«Al analizar millones de variantes genéticas (SNP), ¿cómo decidimos cuáles importan y cuánto importan?».
Tanto P+T como Shrinkage son enfoques para simplificar y mejorar las puntuaciones de riesgo poligénico (PRS) abordando retos como la redundancia causada por SNPs en desequilibrio de ligamiento (LD) y tamaños de efecto ruidosos debido a la limitada potencia estadística de los GWAS.

\subsubsection{P+T}
\textbf{Pruning} hace referencia a eliminar los SNP que estén altamente correlacionados (en LD) con otros, manteniendo sólo el más significativo. 
Primero se utiliza un umbral de desequilibrio de ligamiento (LD) para reducir la redundancia entre los SNP retenidos. Por ejemplo, se utiliza un umbral LD de $r^2 > 0,2$. Además, se analiza LD entre SNPs en la población de referencia: \\
SNP1 y SNP2: $r^2=0,8$ (LD alto).\\
SNP3: $r^2=0,05$ (LD bajo con otros).\\
Se conserva sólo un SNP por bloque LD: \\
SNP2 se mantiene del par SNP1-SNP2 porque tiene un p-valor más pequeño. \\
SNP3 se mantiene porque está en LD bajo con los otros.

\begin{table}[htbp]
\centering
\begin{tabular}{l l l}
& $\beta$ & p \\ \hline
\st{SNP1} & \st{0,12} & \st{$1 \cdot 10^{-6}$} \\
SNP2 & 0,09 & $3 \cdot 10^{-8}$ \\
SNP3 & 0,07 & $0,02$ \\
\st{SNP4} & \st{0,05} & \st{$0,04$} \\
\st{SNP5} & \st{0,02} & \st{$0,1$} \\
\end{tabular}
\end{table}

 \textbf{Thresholding} descarta los SNP con asociaciones débiles (valores p altos) para centrarse en los SNP con efectos estadísticamente significativos.
Primero se establece un umbral de valor p para filtrar los SNP en función de su importancia estadística.
Aquí, seleccionamos el umbral de p<0,05. Así, retenemos los SNPs SNP1, SNP2 y SNP3, quedando SNP4 y SNP5 excluidos.

\begin{table}[htbp]
\centering
\begin{tabular}{l l l}
& $\beta$ & p \\ \hline
SNP1 & 0,12 & $1 \cdot 10^{-6}$ \\
SNP2 & 0,09 & $3 \cdot 10^{-8}$ \\
SNP3 & 0,07 & $0,02$ \\
\st{SNP4} & \st{0,05} & \st{$0,04$} \\
\st{SNP5} & \st{0,02} & \st{$0,1$} \\
\end{tabular}
\end{table}
 
Estos dos métodos permiten simplificar el modelo, ya que se centra en los SNP más significativos e independientes. Además, es eficaz, al reducir la complejidad computacional mediante el pruning de SNPs correlacionados. No obstante, puede descartar SNP débilmente asociados que contribuyen colectivamente al rasgo. Los umbrales de LD también pueden variar entre poblaciones, lo que reduce la generalizabilidad de la PRS.
 
 \subsubsection{Shrinkage}
 Shrinkage es necesario por:
\begin{itemize}
\item \textbf{Datos ruidosos:} los tamaños del efecto ($\beta$) estimados a partir de GWAS suelen ser ruidosos, especialmente para SNPs con asociaciones débiles.
\item \textbf{Poligenicidad:} La mayoría de los rasgos son poligénicos, lo que significa que muchos SNPs contribuyen con pequeños efectos en lugar de unos pocos SNPs con grandes efectos.
\item \textbf{Estructura LD:} Los SNP están correlacionados debido al desequilibrio de ligamiento (LD), que puede inflar las señales de los SNP no causales.
\end{itemize}
 
La contracción (shrinkage) se refiere a un conjunto de técnicas de regularización utilizadas para reducir la magnitud de los coeficientes de regresión en modelos estadísticos. 

Función en las puntuaciones de riesgo poligénico (PRS):
\begin{itemize}
\item Optimiza los tamaños del efecto SNP: Mejora la estimación de los tamaños de efecto de los polimorfismos de nucleótido único (SNP).
\item Mejora el rendimiento predictivo: Aumenta la precisión y robustez de las PRS refinando las estimaciones de los coeficientes.
\end{itemize}

\paragraph{Bayesiano}
La \textbf{reducción bayesiana} aplica un enfoque bayesiano para estimar los tamaños del efecto genético ($\beta$) de los SNP en un modelo poligénico.
En este enfoque se considera un prior (distribución a priori) para los tamaños del efecto que refleja nuestras suposiciones (por ejemplo, la mayoría de los SNP tienen efectos pequeños o nulos).
Las estimaciones del tamaño del efecto se actualizan utilizando los datos disponibles (pruebas) para obtener una distribución posterior.
Este método reduce las estimaciones hacia cero, en particular para los SNP con poco apoyo estadístico, evitando el sobreajuste.

\paragraph{Semi-Bayesiano}
En los métodos bayesianos, es necesario especificar una distribución a priori. Pero, ¿y si no sabemos cómo debe ser la distribución a priori? Los \textbf{métodos bayesianos empíricos} estiman la prioridad directamente a partir de los propios datos, lo que hace que el enfoque esté más orientado a los datos. Combinan las ventajas de los métodos bayesianos con una mayor flexibilidad.

\paragraph{LDPred}
\textbf{LDPred} es un método de PRS que mejora la precisión modelando explícitamente el desequilibrio de ligamiento (LD), la asociación no aleatoria de alelos en diferentes loci genéticos. Los métodos tradicionales, como la poda y el umbral, a menudo ignoran el LD o lo tratan de forma inadecuada, lo que conduce a una menor precisión. Utiliza estadística bayesiana con un modelo a priori de mezcla normal para modelar los efectos genéticos, asumiendo que una pequeña proporción de variantes tienen efectos sustanciales mientras que la mayoría tienen un impacto mínimo. Calcula los efectos medios posteriores de las variantes combinando el previo con la información LD de un panel de referencia. Esto tiene en cuenta la correlación entre variantes y pondera con precisión sus contribuciones a la PRS.
Requiere estadísticas de resumen GWAS e información LD de un panel de referencia con patrones LD similares a los datos de entrenamiento. Se recomienda un panel de referencia de al menos 1.000 individuos. 

Como datos de entrada requiere de:
\begin{itemize}
\item \textbf{Summary statistics de GWAS:} suelen incluir el ID del SNP, el tamaño del efecto (beta u odds ratio), el valor p y el tamaño de la muestra.
\item \textbf{Panel de referencia LD:} Este panel proporciona información sobre la correlación entre SNPs y permite al algoritmo LDpred tener en cuenta con precisión el desequilibrio de ligamiento (LD).
\end{itemize}

Los parámetros son el radio LD, el cual determina el número de SNPs a cada lado de un SNP dado que considera al calcular la media posterior del tamaño del efecto, y la fracción de variantes causales ($\rho$), que refleja la proporción de SNPs que se supone que tienen un efecto distinto de cero sobre el rasgo o la enfermedad.

Un radio de LD grande podría considerar erróneamente loci no ligados como ligados debido a errores en la estimación de LD, lo que llevaría a estimaciones de efecto imprecisas y problemas de convergencia. 
Un radio LD pequeño puede no tener en cuenta el LD entre loci genuinamente ligados.
Valor por defecto de M/3000 donde M es el número total de SNPs $\sim$ ventana de LD de 2Mb de media 

La fracción de variantes causales refleja la proporción de SNP que se supone que tienen un efecto distinto de cero sobre el rasgo o la enfermedad. Es desconocida y puede variar significativamente entre diferentes fenotipos. Es común probar un rango de valores $\rho$ cuando se construye una PRS con LDpred, de forma similar a como se exploran los umbrales del valor p en el método de poda y umbral (P+T).
El valor $\rho$ óptimo suele determinarse evaluando la precisión predictiva de la PRS en un conjunto de datos de validación independiente.

\subsection{PRS en la práctica - ejemplo de un estudio}
Aquí analizamos el ejemplo del estudio con el artículo llamado "Genome-wide polygenic scores for common diseases identify individuals with risk equivalent to monogenic mutations". De forma general, las variantes comunes están implicadas en enfermedades comunes y tienen poco efecto. No obstante, hay algunos ejemplos de variantes comunes de alto impacto que influyen en enfermedades comunes.

En las enfermedades comunes, las mutaciones raras confieren un riesgo varias veces mayor a los portadores heterocigotos. 
\begin{table}[htbp]
\begin{tabular}{l l l l}
Mutation & Affected population & Increment of risk & Disease \\ \hline
Familial hypercol. & 0,4\% & x3 & CAD \\
HNF1A Glu508Lys & 0,01\% (0,7\% Latinos) & x5 & Type2 diabetes
\end{tabular}
\end{table}

Esto es muy relevante para los portadores, pero la mayoría de la población de riesgo sigue sin identificarse.

Los objetivos del estudio son crear una puntuación poligénica de todo el genoma (GPS) para cinco enfermedades complejas comunes para poder identificar individuos con un riesgo clínicamente significativo (comparable a los niveles conferidos por mutaciones monogénicas raras) y que esté basada en estadísticas de resumen e imputación de grandes GWAS recientes (principalmente de ascendencia europea).

\begin{figure}[htbp]
\centering
\includegraphics[width = 0.8 \textwidth]{figs/Imagen4.png}
\end{figure}

\newpage
Primero se cogieron estudios de GWAS que asocien el genotipo con un trait/enfermedad. Un panel de individuos permite calcular los bloques LD. Así, se crean 31 polygenic scores para cada enfermedad, 24 para el P+T y 7 para LDPred. Con la cohorte de Biobank se valida y se prueba en otra subcohorte. 

\begin{figure}[htbp]
\centering
\includegraphics[width = 0.8 \textwidth]{figs/Imagen5.png}
\end{figure}

Para todas las enfermedades se calculó la predicción (el área bajo la curva) utilizando las variantes significativas en GWAS, la estrategia P+T y el algoritmo LDPred. Este método intenta repartir el efecto o la imagen duplicada/triplicada de los SNPs que se encuentran en el mismo LD, y se compara el dataset de validación y en el dataset de prueba. El área bajo la curva de ambos datasets son similares, lo que significa que los modelos funcionan bien para las poblaciones estudiadas.

\begin{figure}[htbp]
\centering
\includegraphics[width = 0.8 \textwidth]{figs/Imagen6.png}
\end{figure}

\newpage
En cuanto al riesgo de CAD acorde al GPS, se normalizó la distribución y se clasificó según el incremento del riesgo. El 8\% de la población tiene un riesgo 3x, el 2,3\% un riesgo 4x y un 0,5\% de la población un riesgo del 5x. El boxplot muestra que los casos tienen un mayor riesgo, y los controles un score menor.

\begin{figure}[htbp]
\centering
\includegraphics[width = \textwidth]{figs/Imagen7.png}
\end{figure}

\begin{figure}[htbp]
\centering
\includegraphics[width = 0.8 \textwidth]{figs/Imagen8.png}
\end{figure}

\newpage
En CAD hay un predictor para individuos con 3 veces más riesgo, que representan un 8\% de la población. También se muestran otros factores de riesgo y su nivel de significancia. Por ejemplo, fumar e hipertensión son factores de riesgo con un p-valor significativo. 
En la comparación entre individuos con un GPS alto y aquellos con puntajes promedio, las personas con un GPS alto tienden a tener otros factores de riesgo (por ejemplo, antecedentes familiares o niveles elevados de biomarcadores). Esto refuerza la utilidad del GPS como complemento a otras medidas de riesgo tradicionales.

\begin{figure}[htbp]
\centering
\includegraphics[width = 0.8 \textwidth]{figs/Imagen9.png}
\end{figure}

En conclusión:
\begin{itemize}
\item La posibilidad de identificar a los individuos con un riesgo genético significativamente mayor, en una amplia gama de enfermedades comunes y a cualquier edad, plantea una serie de oportunidades y retos para la medicina clínica.
\item Cuando se disponga de estrategias eficaces de prevención o detección precoz, las cuestiones clave serán la asignación de atención y recursos a individuos con distintos niveles de riesgo genético y la integración de la estratificación del riesgo genético con otros factores de riesgo, incluidas las mutaciones monogénicas raras y los factores clínicos y ambientales.
\item Cuando estas estrategias no existan o no sean óptimas, la identificación de los individuos de alto riesgo facilitará el diseño de estudios eficaces de la historia natural para descubrir marcadores precoces de la aparición de la enfermedad y de ensayos clínicos para probar estrategias de prevención.
\item El riesgo asociado a una puntuación poligénica elevada puede no reflejar un único mecanismo subyacente de patogenicidad. Prácticamente todas las enfermedades muestran un resultado final (la patología), pero lo que lleva a ella no se muestra.
\item La utilidad de los conocimientos y los posibles perjuicios para el individuo pueden variar en función de la enfermedad y la etapa de la vida. Será importante considerar cómo evaluar los riesgos absolutos y relativos y cómo comunicar estos riesgos para atender mejor a cada paciente; por ejemplo, para fomentar la adopción de modificaciones en el estilo de vida o el cribado de la enfermedad.
\item Las puntuaciones de riesgo poligénico aquí descritas se derivaron y probaron en individuos de ascendencia principalmente europea, por lo que no tendrán un poder predictivo óptimo para otros grupos étnicos.
\end{itemize}


\end{document}
