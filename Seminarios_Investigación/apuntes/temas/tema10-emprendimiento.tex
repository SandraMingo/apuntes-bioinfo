%28/11 - UAM Emprende Jorge + Rocío
\section{Cómo ser un emprendedor en bioinformática}
\textbf{Jorge Álvarez Cercadillo - UAM Emprende}

UAM Emprende es un servicio de la universidad a investigadores. El objetivo es hacer un producto. 

El emprendimiento parece complicado con un local, unos empleados y demás, pero el primer paso es desarrollar el servicio para después poder obtener un retorno. El emprendimiento científico puede ser llevar a la práctica la investigación, una patente, un dispositivo médico, algoritmos de análisis, etc. 

El perfil del emprendedor es una persona con un máster en su mayoría. El mayor emprendimiento se encuentra en el ámbito de la salud, especialmente en tecnologías como inteligencia artificial, apps, biotech y tecnologías médicas. 

UAM Emprende es una incubadora de proyectos. Todos los años se hace una llamada a profesores principalmente  con un miniproducto al que quiera dar forma y con un plan de trabajo poder dar en un par de años un servicio. 

Uno de los primeros proyectos que se desarrolló fue \textbf{Diagnosis and Discovery for rare diseases} por parte de Pablo Mínguez de la IIS-Fundación Jiménez Díaz. Existen alrededor de 7.000 enfermedades raras, la mayoría de origen genético. El proceso diagnóstico actual resuelve solo el 50 \% de los casos. El diagnóstico de enfermedades raras es un mercado en crecimiento mundial. Así, se ofrece muna solución integrada orientada a mejorar y acelerar el diagnóstico genético de enfermedades raras mediante análisis avanzados, priorización de variantes y descubrimiento de nuevos genes implicados. 

Otro proyecto fue \textbf{EDx Solutions} que propuso un nuevo dispositivo de EMG (electromiografía) avanzado basado en algoritmos de análisis, algoritmos de visualización y un nuevo biomarcador clave. Comparado con líderes como Natus, Cadwell o Nihon Kohden, destaca por el acceso total a los datos, análisis avanzado de señal, visualización y patrones, informes inteligentes y precisión para medicina personalizada. 

Uno de los desafíos en el emprendimiento científico es que el desarrollo de los productos tarda varios años, y hasta que el producto no esté terminado no hay clientes que lo compren. Por ello, se debe acudir a los inversores. Una alternativa es por CrowdEquity, que es contactar con inversores privados como familia o amigos para entre todos poder financiar un proyecto. Una plataforma es \textit{CapitalCell}.