%21/11 - healthdata-ri@salud.madrid.org
\section{HealthData\@ MAD-R\&I: Hacia un espacio de datos de salud para investigación e innovación en la Comunidad de Madrid}
\textbf{Hospital Universitario La Paz}

Hay un reglamento del espacio europeo de datos sanitarios del año 2024 que da plazo hasta 2027 para crear y ofrecer a toda la comunidad investigadora dentro y fuera del sistema sanitario información que sirva para contestar preguntas de interés social. El compromiso es ofrecer un catálogo con unos metadatos que irá creciendo. Esa disposición tiene que permitir que cualquiera a través de unos permisos y autorizaciones solicite la información.

El uso de datos secundarios es un elemento clave para la mejora de la calidad asistencial. Hay muchos tipos de datos (escritos, imágenes, codificaciones) que se infrautilizan en el uso secundario. El uso primario de los datos es utilizarlos para la propia asistencia, por lo que es necesario crear entornos que los recojan, puesto que los datos entre distintos hospitales y comunidades autónomas no se cruzan. El uso secundario, solo un 3\% de la información que se genera revierte en investigación e innovación. 

La infrautilización de lo que se genera en consulta es una pena, ya que tiene un gran potencial. Esto se debe a la fragmentación e interoperabilidad. Hay diferentes programas informáticos de historia clínica para los distintos hospitales (públicos). Además, los datos tienen una baja calidad (errores de registro, duplicidades, datos faltantes). Lo más problemático es la gobernanza de los datos, ya que hay que poner a mucha gente de acuerdo con intereses muy distintos. 

Los datos se encuentran en los servidores de AWS de Amazon. La responsabilidad de los datos de salud estaba en el gerente del hospital, pero se ha migrado a la viceconsejería de salud para que sea más fácil y accesible. 

HealthData busca dar un espacio de datos de salud regional para uso secundario de investigación e innovación. Un lago de datos pueden ser los datos crudos sin organizar, mientras que un espacio de datos los tiene trabajados, homogeneizados y combinados. Este espacio debe ser federado (los datos no salen de ahí), tener múltiples fuentes de datos, datos anonimizados o pseudonimizados y con beneficios sociales. 

Como todo proyecto grando europeo, se divide en varias fases: coordinación del proyecto, modelo de gobernanza, evolución e integración de las plataformas, interoperabilidad, comunicación y divulgación, escalabilidad y sostenibilidad. 

Un modelo de gobernanza es el marco que define cómo se organizan, regulan y supervisan los usos secundarios de los datos. Establece las reglas, los roles y las garantías. Se propone el siguiente modelo de gobernanza:
\begin{itemize}
\item Comité de Dirección del Espacio
\item Oficina Técnica del Espacio de Datos
\item Comité de Acceso y Evaluación de Proyectos
\item Comité Ético y Científico Asesor
\end{itemize}