%07/11 - Fernando Casares
\section{A baroque feedback mechanism controlling organ size precision}
\textbf{Fernando Casares - CABD (Centro Andaluz de Biología del Desarrollo)}

Las relaciones del desarrollo tienen que estar ajustadas para que al final aparezca el mismo producto final pese a las variaciones en las condiciones del desarrollo. Para CH Waddington, la constancia del desarrollo pedía la existencia de mecanismos de control. 

Los ojos son unos órganos pequeños con un tejido preciso. En Drosophila melanogaster, como en cualquier insecto, los ojos son compuestos y están formados por miles de unidades visuales diminutas llamadas omatidios, que generan una imagen en mosaico, y los ocelos son ojos simples que detectan la luz ambiental. Este ojo compuesto es caro de crear y mantener. Hay algo en los mecanismos de control que generan un ojo de tamaño grande, pese a mutaciones que pueda tener. 

El ojo es una capa monoestratificada que forma un complejo mayor. A medida que el ojo crece, se produce la diferenciación en forma de ola. El proceso de desarrollo tiene muchas no linealidades, siendo así muy susceptible al ruido. Esto se puede representar en un crecimiento exponencial. Cualquier variación en el proceso puede generar desviaciones muy grandes. Por tanto, el control es evidente a múltiples niveles. 

La fluctuación simétrica es la variación entre un órgano izquierdo y derecho. 

Los genes G, Rn, R, Dpp y Hh son abreviaciones comunes para Glass (G), Rotund (Rn), Rough (R), Decapentaplegic (Dpp) y Hedgehog (Hh), todos ellos con roles cruciales en la formación y diferenciación del ojo compuesto.
\begin{enumerate}
\item Rotund (Rn) define dominios en el disco (destino “ojo”).
\item Hh (desde células ya diferenciadas) impulsa la hendidura morfogenética.
\item Dpp señal secundaria que promueve diferenciación.
\item Rough (R) especifica tipos de fotorreceptores.
\item Glass (G) activa el programa final de diferenciación visual.
\end{enumerate}

Cuando se elimina Dpp, el ojo es más pequeño, pero más variable que en comparación con un control. Al bloquear la muerte celular (apoptosis) genéticamente, el ojo no solo recupera su tamaño, sino que incluso es más grande. Es la falta de Dpp la que manda a las células a apoptosis, pero son células viables. No obstante, la precisión no se recupera (la distribución de tamaños es más grande). Todas las manipulaciones en las que se toca la vía de Dpp o se bloquea la apoptosis, la precisión se pierde.  

Para formalizar esto matemáticamente, se buscaba crear un modelo informado cuantitativamente para poder hacer predicciones sobre la precisión del sistema. El proceso se puede definir viendo cómo van cambiando las células progenitoras, en diferenciación y diferenciadas. Las curvas de crecimiento se ajustan muy bien a las observaciones experimentales.

A escala celular, un proceso caro (que produce células que se deben matar) debe ser intrínsicamente ruidoso o variable.