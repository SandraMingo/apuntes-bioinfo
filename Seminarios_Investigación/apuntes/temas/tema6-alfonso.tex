%24/10 - Alfonso Valencia
\section{Los límites de computación en bioinformática}
\textbf{Alfonso Valencia - BSC-CNS}

\subsection{Computación}
Un superordenador es un ordenador con grandes prestaciones. En el caso del BSC, son dos ordenadores: uno de general purpose y otro de uso acelerado. Esto es parte de una red europea de ordenadores grandes. 

Ahora mismo hay 12 factorías de inteligencia artificial financiadas en la Unión Europea. Estas factorías pretenden dar soporte al desarrollo de modelos, incluidos los conjuntos de datos, los cuales contienen datos médicos, imagen médica, datos genómicos, trayectorias de dinámica molecular y librerías de pequeñas moléculas. 

\subsection{Gemelos digitales en el BSC}
El gemelo digital corre a la vez que la factoría y simula todo. Esto sirve para organizar la forma más rápida de reestablecer la cadena de fabricación en caso de que falle un robot. Hay otros gemelos digitales: un gemelo digital de la Tierra para predecir el cambio climático, gemelos digitales de las ciudades, gemelos digitales de humanos de partes concretas. Se pueden hacer gemelos digitales de modelos moleculares, simulaciones de órganos, de tumores. 

Se toman datos de single cell para tener la información de señalización celular y sus genes y proteínas activadas. Con esto se pueden simular tumores y organoides. La FDA ha empezado a aprobar medicamentos para humanos en los que parte del dossier se incluyen simulaciones, por lo que ha aumentado el interés en estas técnicas. 

El problema de los gemelos digitales en humanos es que es un sistema multiescala, son simulaciones a gran escala y la definición de parámetros y estructura interna en la señalización celular. 

\subsection{Acceso a datos}
La promesa del futuro es poder acceder a los datos, que estén disponibles para poder inferir el mecanismo celular y patológico. El mayor experimento que puede tener la humanidad es en la práctica clínica con los sujetos que van al hospital y se someten a tratamientos, pruebas, imágenes, etc. Todos los datos que se generan en la atención primaria recientemente se deben poder utilizar para la investigación según una normativa europea. Hay varias iniciativas en Europa para que los datos sean accesibles sin que salgan desde donde están: Elixir, Eucaim, IMPaCT. 

Hay un sistema federado probado en distintos sitios en España que permiten acceder a unas herramientas y potencia computacional con unos datos sin que salgan del sitio que se conecta. El problema a nivel europeo es que la burocracia limita todo y no fue posible acceder a los datos debido a que las autoridades de los datos no lo permitieron. No obstante, el siguiente problema será la (falta de) interoperabilidad de los datos de los distintos sitios. A futuro se considera la ontología de OMOP o una curación del texto escrito por médicos a la terminología de la ontología.

\subsection{Inteligencia artificial}
Las aplicaciones más interesantes que están surgiendo en la inteligencia artificial son en el campo de la biomedicina al ser el campo más complejo. Hay modelos de lenguaje específicos para la medicina en la que se extraen los conceptos del texto del historial médico. Con esto se busca sacar unos guidelines para finales de año. 

El mayor logro de la IA fue el modelado molecular de proteínas. Esto permite no solo crear nuevos fármacos a unas velocidades desorbitantes, si no también explorar la evolución de las funciones proteicas y explorar el espacio de secuencias. Además, esto permite generar datos sintéticos que se pueden utilizar para modelar. No obstante, ahora entra la pregunta, ¿cómo sabemos que los datos sintéticos son correctos? Con las proteínas era fácil saberlo por cristalografía de experimentos previos. 

El sesgo más importante que se encuentra es relativo al sexo y género. La gran mayoría de los datos se han obtenido de hombres o ratones machos, y en el caso de los humanos suele ser hombres blancos de un cierto rango de edad. Esto es terrible porque se ha estudiado que hombres y mujeres tienen pathways distintos para el dolor y las patologías. No obstante, también hay otros sesgos como los técnicos.

La mayor parte de las interacciones por la web son por agentes. Los agentes de IA son sistemas de software autónomos que utilizan IA para conseguir objetivos específicos y realizar tareas por los usuarios. Pueden interaccionar con el entorno y tomar acciones. 
