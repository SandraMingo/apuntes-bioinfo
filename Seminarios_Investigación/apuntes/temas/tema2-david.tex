%26/09 - David Juan CNB
\section{Comparative primate genomics to understand evolution, health, and disease}
\textbf{David Juan - investigador Centro Nacional de Biotecnología}

Primates are one of the model systems. The closest relatives of humans are chimpanzees and bonobos, although the most commonly used model animal is the mouse. Primates have a higher relevance for disease, immunity, cognition and aging, unique phenotypes absent in rodents like longevity, brain development and cancer resistance, and more similar genomes and transcriptomic and epigenomic regulation. However, working with primates is very difficult due to ethical and legal restrictions, limited availability, high maintenance cost, slower breeding cycles, fewer genetic tools and established inbred lines compared to mice and smaller sample sizes reduce statistical power. 

In 2023, we had 20 long-read and 40 short-read assemblies, in addition to some RNA-seq data for almost all these species. Epigenomic studies exist only for great apes and a few other primates. 
Most studies are focused on the evolution of the human brain, why humans are so intelligent. That being said, omic data per tissue is scarse, mostly concentrated on the brain. 

In 2023, 17 new long-read assemblies were published together with 233 primate short-read genomes (assemblies and several individuals).

The group is focusing on great apes to understand human mutational processes. To study aberrant mutational processes in tumors, they compare them to normal germline mutations extracted from genomic population data.  

Populations of great apes are better models for understanding the accumulation of somatic mutations in human cells than human population. In fact, gorillas and chimpanzees are more correlated to human tumors than humans themselves.

